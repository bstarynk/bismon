% file appendix-bm.tex, which is \input from bismon-chariot-doc.tex
% see https://github.com/bstarynk/bismon for more about Bismon
\section{Appendix}
\label{sec:appendix}

\subsection{Building \texttt{bismon} from its source code}
\label{subsec:building-bismon}

We focus here on how to build \texttt{bismon} from its source code on
\href{http://debian.org/}{Debian}-like distributions running on
x86-64 computers, such as
\href{https://www.debian.org/releases/stable/}{Debian Buster 10.6}, or
\href{https://ubuntu.com/}{Ubuntu 20.04}, or
\href{https://linuxmint.com/}{Linux Mint 20}. Familiarity with the
command line is required\footnote{For example, the reader is expected
  of being able to build \href{https://www.gnu.org/software/make/}{GNU
    make} or \href{https://www.gnu.org/software/bash/}{GNU bash} from
  their source code.}, with \texttt{root} access (e.g. using
\texttt{sudo}).

The reader is expected to be authorized (by his/her management, if
that build is done professionally) to build \texttt{bismon} from its
source code and probably also some recent
\href{https://gcc.gnu.org/}{GCC} cross-compiler on his/her Linux workstation
and should budget several days of work for that.


\subsection{Prerequisites for building \texttt{bismon}}
\label{subsec:prereq-bismon}

The \texttt{bismon} source code is on
\href{https://github.com/bstarynk/bismon/}{\texttt{\textbf{github.com/bstarynk/bismon/}}}
and the reader is expected to be capable of getting that source code
on his/her Linux workstation. A possible command to retrieve that code
might be \texttt{git clone https://github.com/bstarynk/bismon.git} ;
you'll then obtain a \emph{fresh} \texttt{bismon/} subdirectory
containing the source code. About 100Mbytes of disk space (for less
than 2000 inodes) is required.

A recent \texttt{libonion} library\footnote{This is an open source
  library for web HTTP and HTTPS service. It is LGPL licensed.}
(version 0.8 at least) is required. Fetch \texttt{libonion}'s source
code from
\href{https://github.com/davidmoreno/onion}{\texttt{\textbf{github.com/davidmoreno/onion/}}}
and follow it build instructions: probably \texttt{mkdir \_build} then
\texttt{cd \_build} then \texttt{cmake ..} then \texttt{make} and at
last \texttt{sudo make install}. That \texttt{libonion} library needs
less than 25Mbytes of disk space,
\href{https://cmake.org}{\texttt{cmake}} and several libraries (in
particular support for \texttt{openssl}, \texttt{gcrypt},
\texttt{systemd}, \texttt{sqlite3}, \texttt{lzma}, \texttt{libicu},
\texttt{libpam}) to be built. Check and inspect your
\texttt{onion/version.h} header file\footnote{You might use
  \href{https://man7.org/linux/man-pages/man1/locate.1.html}{\texttt{locate(1)}}
  or
  \href{https://man7.org/linux/man-pages/man1/find.1.html}{\texttt{find(1)}}
  to find files on your Linux box. On \emph{my} Linux machine, that
  header file is in \texttt{/usr/local/include/onion/version.h} and
  comes from \emph{libonion}
  \href{https://github.com/davidmoreno/onion/commit/43128b03199518d4878074c311ff71ff0018aea8}{git
    commit \texttt{43128b031995}}....}, it should have some
\texttt{ONION\_VERSION} close to \texttt{0.8.150} at least.
