% file appendix-bm.tex, which is \input from bismon-chariot-doc.tex
% see https://github.com/bstarynk/bismon for more about Bismon
\section{Appendix}
\label{sec:appendix}

\subsection{Building \texttt{bismon} from its source code}
\label{subsec:building-bismon}

We focus here on how to build \texttt{bismon} from its source code on
\href{http://debian.org/}{Debian}-like distributions running on
x86-64 computers, such as
\href{https://www.debian.org/releases/stable/}{Debian Buster 10.6}, or
\href{https://ubuntu.com/}{Ubuntu 20.04}, or
\href{https://linuxmint.com/}{Linux Mint 20}. Familiarity with the
command line is required\footnote{For example, the reader is expected
  of being able to build \href{https://www.gnu.org/software/make/}{GNU
    make} or \href{https://www.gnu.org/software/bash/}{GNU bash} from
  their source code.}, with \texttt{root} access (e.g. using
\texttt{sudo}).


\subsection{Prerequisites for building \texttt{bismon}}
\label{subsec:prereq-bismon}

The \texttt{bismon} source code is on
\href{https://github.com/bstarynk/bismon/}{\texttt{\textbf{github.com/bstarynk/bismon/}}}
and the reader is expected to be capable of getting that source code
on his/her Linux workstation. A possible command might be \texttt{git
  clone https://github.com/bstarynk/bismon.git} to obtain a
\emph{fresh} \texttt{bismon/} subdirectory containing the source
code. About 100Mbytes of disk space (for less than 2000 inodes) is
required.

A recent \texttt{libonion} library\footnote{This is an open source
library for web HTTP and HTTPS service. It is LGPL licensed.} is
required. Fetch the source code from
\href{https://github.com/davidmoreno/onion}{\texttt{\textbf{github.com/davidmoreno/onion/}}}. That
library needs less than 25Mbytes of disk space, \texttt{cmake} and
several libraries (in particular support for \texttt{openssl},
\texttt{gcrypt}, \texttt{systemd}, \texttt{sqlite3}, \texttt{lzma},
\texttt{libicu}, \texttt{libpam}) to be built.
