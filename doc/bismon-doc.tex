% File bismon/doc/bismon-doc.tex on github.com/bstarynk/bismon
% by <basile.starynkevitch@cea.fr> ... Please read both
% bismon/README.md and bismon/dependencies/README-dependencies.md
% before processing this LaTeX document on Debian related Linux
% distributions using the "make doc" command from bismon/ source
% directory.
\documentclass[11pt,a4paper,svgnames]{article}
\usepackage[T1]{fontenc}
% inputenc is not needed with lualatex
% \usepackage[utf8x]{inputenc}
\usepackage{alltt}
% https://tex.stackexchange.com/a/342804/42406
% IMPORTANT! see also: https://tex.stackexchange.com/a/500963/42406
%\usepackage{textcomp}
\usepackage{moreverb}
\usepackage{fancyvrb}
\usepackage{fancyhdr}
\usepackage{fancybox}
% https://tex.stackexchange.com/a/226497/42406
\usepackage[title]{appendix}
% libertine, see https://tex.stackexchange.com/a/9868/42406
%\usepackage{libertine}
%\usepackage{epsfig}
\usepackage{graphicx}
\usepackage{float}
\usepackage{xcolor}
\usepackage{moreverb}
\usepackage{multirow}
\usepackage{boxedminipage}
\usepackage[square]{natbib}
% https://tex.stackexchange.com/a/16992/42406
\usepackage{mathabx}
%\usepackage{charter}
%\usepackage{inconsolata}
\usepackage{hevea}
\usepackage{listings}
\usepackage{relsize}
\usepackage{verbatimbox}
\usepackage{verbatim}
%\usepackage{filecontents}
\usepackage{catchfile}
\usepackage{lastpage}
\usepackage{stmaryrd}
\usepackage{ucs}
%\usepackage{stix}
\usepackage{newunicodechar}
% bigfoot enables \verb in footnotes
\usepackage{bigfoot}
\usepackage{makeidx}
\usepackage{times}
% https://tex.stackexchange.com/a/413066/42406
\usepackage{apptools}
\usepackage[a4paper, margin=2cm]{geometry}
\usepackage{hyperref}

\newcommand{\bmemail}[1]{{\href{mailto:#1}{\texttt{\textbf{#1}}}}}
\newcommand{\bmurl}[1]{{\href{#1}{\texttt{\relsize{-1}{\textbf{#1}}}}}}


% see also http://www.sascha-frank.com/Arrow/latex-arrows.html
% and http://tug.ctan.org/info/symbols/comprehensive/symbols-a4.pdf
% and https://ctan.math.illinois.edu/macros/latex/contrib/newunicodechar/newunicodechar.pdf
%%%% keep in order
%U+21A6 RIGHTWARDS ARROW FROM BAR
\newunicodechar{↦}{$\mapsto$}
%U+21B3 DOWNWARDS ARROW WITH TIP RIGHTWARDS
\newunicodechar{↳}{\rotatebox[origin=c]{180}{$\Lsh$}}
%U+2208 ELEMENT OF
\newunicodechar{∈}{$\in$}
% U+00AB LEFT-POINTING DOUBLE ANGLE QUOTATION MARK
\newunicodechar{«}{\guillemotleft}
% U+00BB RIGHT-POINTING DOUBLE ANGLE QUOTATION MARK
\newunicodechar{»}{\guillemotright}
% U+00B1 PLUS-MINUS SIGN
\newunicodechar{±}{$\pm$}
% U+00B5 MICRO SIGN
\newunicodechar{µ}{$\mu$}


%see https://tex.stackexchange.com/a/51656/42406 
\lstdefinelanguage
   [x64]{Assembler}     % add a "x64" dialect of Assembler
   [x86masm]{Assembler} % based on the "x86masm" dialect
   % with these extra keywords:
   {morekeywords={CDQE,CQO,CMPSQ,CMPXCHG16B,JRCXZ,LODSQ,MOVSXD, %
                  POPFQ,PUSHFQ,SCASQ,STOSQ,IRETQ,RDTSCP,SWAPGS, %
                  rax,rdx,rcx,rbx,rsi,rdi,rsp,rbp, %
                  r8,r8d,r8w,r8b,r9,r9d,r9w,r9b, %
                  r10,r10d,r10w,r10b,r11,r11d,r11w,r11b, %
                  r12,r12d,r12w,r12b,r13,r13d,r13w,r13b, %
                  r14,r14d,r14w,r14b,r15,r15d,r15w,r15b}} % etc.

%% https://latex.org/forum/viewtopic.php?t=3970
%\makeatletter
%\newcommand*\bmexpandableinput[1]{\@@input#1 }
%\makeatother

 % see https://tex.stackexchange.com/a/51349/42406
\hypersetup{
  colorlinks   = true, %Colours links instead of ugly boxes
  urlcolor     = NavyBlue, %Colour for external hyperlinks
  linkcolor    = DarkGreen, %Colour of internal links
  citecolor   = DarkMagenta, %Colour of citations
  frenchlinks = true,
}
% generated files
\input{generated/git-commit.tex}
\input{generated/timestamp.tex}



% LaTeX specific macros that HeVeA should ignore
%% file bismon-latex.tex
% LaTeX specific macros that HeVeA should ignore

\newcommand{\myincludewidthgraphics}[2]{\includegraphics[width=#1]{#2}}


\date{\bmdocdate}

%-\title{{\color{orange}{D1.3v1}}~Specialized Static Analysis tools for more secure and safer IoT software development.
%-  \\
%-  {\large{\emph{Bismon}~\thanks{The \texttt{bismon} name is temporary and
%-    is likely to change later. See also the
%-    \bmurl{http://github.com/bstarynk/bismon} repository; this documentation was generated with its \texttt{build-bismon-doc.sh} script}~ documentation}}}
%-
%-\author{Basile \textsc{Starynkevitch}\thanks{Email
%-    \bmemail{basile@starynkevitch.net} or
%-    \bmemail{basile.starynkevitch@cea.fr}}, %
%-  \\ {\small{CEA, LIST,
%-      France}}}
%-
% the index works again in commit d7070ccc581cf951e2a22bfed8 but we dont know why :-(
\makeindex

\begin{document}

\bibliographystyle{abbrvnat}


\pagestyle{fancy}
\fancyhf{}
\rhead{the \textsc{Bismon} static source code analyzer}
\renewcommand{\footrulewidth}{0.4pt}
% https://tex.stackexchange.com/a/317531/42406
\futurelet\TMPfootrule\def\footrule{{\color{orange}\TMPfootrule}}
\futurelet\TMPheadrule\def\headrule{{\color{orange}\TMPheadrule}}
\fancyfoot[L]{{\raisebox{0.0cm}[1pt][1pt]{\color{LightSlateGrey}{{\relsize{+1}{DRAFT}}}~\relsize{-1.5}{\texttt{\bmgitcommit} on \emph{\bmgitdate}}}}}
\fancyfoot[R]{{Page \thepage ~ {\relsize{-2.5}{of \pageref{LastPage}}}}}
\begin{titlepage}
%\maketitle

  \thispagestyle{empty}

  \begin{center}
    
    {\LARGE documentation of the \textsc{Bismon} static source code analyzer.}

  \end{center}

  \medskip

  \begin{quote}

\begin{relsize}{-0.5}
    This report and the \emph{Bismon} static analyzer has been funded
    by European H2020 projects
    \href{http://www.chariotproject.eu}{\textbf{\texttt{www.chariotproject.eu}}}
    (grant 780075) and
    \href{http://www.decoder-project.eu}{\textbf{\texttt{www.decoder-project.eu}}} (grant 824231).

    The opinions are those of the author(s) only. The content of the publication
      herein is the sole responsibility of the publishers and it does
      not necessarily represent the views expressed by the European
      Commission or its services.
\end{relsize}

\medskip

%\hline

%\medskip

\textbf{author}: 
   \begin{minipage}{8cm}\smallskip 
     Basile \textsc{Starynkevitch} (CEA, LIST) ~ {\relsize{-1.5}{DILS/LSL}} \\
     {\relsize{-1.5}{from its \textsc{Software Security Laboratory}}} \\
     {\relsize{-1}{\bmemail{basile.starynkevitch@cea.fr}}} \\
     {\relsize{-1}{\bmurl{http://starynkevitch.net/Basile/}}} \\
     \smallskip
   \end{minipage}
   
\medskip

  \end{quote}
  
  
%-  \begin{relsize}{-0.1}
%-  \subsection*{Document Summary Information}
%-    \begin{tabular}{|l|l|l|l|}
%-      \hline 
%-    \textbf{Grant Agreement} & 780075 & \textbf{Acronym} & \textsc{Chariot} \\
%-    \hline 
%-      {\textbf{Full title}} & %
%-      \multicolumn{3}{l|}{\textbf{\relsize{+0.5}{C}}ognitive \textbf{\relsize{+0.5}{H}}eterogeneous \textbf{\relsize{+0.5}{A}}rchitecture for Industrial {\relsize{+0.5}{\textbf{IoT}}}}  \\
%-      \hline
%-      \textbf{Start date} & 01/01/2018 & \textbf{Duration} & 36 months \\
%-      \hline
%-    \textbf{Project URL} & \multicolumn{3}{l|}{{\large \href{http://www.chariotproject.eu}{\textbf{\texttt{www.chariotproject.eu}}}}} \\
%-      \hline
%-    \textbf{Deliverable} & \multicolumn{3}{p{9.5cm}|}{{\textbf{D1.3} Specialized Static Analysis tools for more secure and safer IoT software development.}} \\
%-      \hline
%-    \textbf{Workpackage} & \multicolumn{3}{l|}{\textbf{WP1}} \\
%-      \hline
%-    \textbf{Dissemination level} & \multicolumn{3}{l|}{\textbf{Public report} {\large {[DRAFT]}}} \\
%-      \hline
%-    \textbf{Contractual due date} & {31/06/2020} (M30) & \textbf{Actual submission date} & - \\
%-      \hline
%-    \textbf{Nature} & R {\small (report)} & \textbf{dissemination level} & PU  {\small (public)} \\
%-      \hline
%-   \textbf{Lead beneficiary} & \multicolumn{3}{l|}{CEA, LIST} \\
%-   \hline
%-   \textbf{Responsible author} & \multicolumn{3}{l|}{%
%-     \begin{minipage}{8cm}\smallskip 
%-       Basile \textsc{Starynkevitch} (CEA, LIST) ~ {\relsize{-1.5}{DILS/LSL}} \\
%-       {\relsize{-1.5}{from its \textsc{Software Security Laboratory}}} \\
%-       {\relsize{-1}{\bmemail{basile.starynkevitch@cea.fr}}} \\
%-       {\relsize{-1}{\bmurl{http://starynkevitch.net/Basile/}}} \\
%-       \smallskip
%-     \end{minipage}}
%-  \\
%-    \hline
%-    \textbf{Contributions from} &\multicolumn{3}{l|}{ - }
%-    \\
%-      \hline
%-    \textbf{Revision} &\multicolumn{3}{p{9cm}|}{git commit \texttt{\bmgitcommit} on \textit{\bmgitdate}} \\
%-      \hline 
%-    {\emph{{\LaTeX}-generated}} &\multicolumn{3}{p{9cm}|}{{\scriptsize\textit{\bmdoctimestamp}}} \\
%-      \hline
%-    {{\relsize{-1}{\emph{draft downloadable from}}}} &\multicolumn{3}{p{12cm}|}{{\relsize{-0.5}{\bmurl{http://starynkevitch.net/Basile/bismon-doc.pdf}}}} \\
%-      \hline
%-  \end{tabular}
%-  \end{relsize}

\medskip

{\textcolor{red}{\large unverified, unapproved, unchecked draft \textbf{Ful of Mistaks}}}.

\medskip

Some version of this \textbf{DRAFT} report might be downloadable from {\bmurl{http://starynkevitch.net/Basile/bismon-doc.pdf}}

  
This partly generated document and the \emph{Bismon} software itself
are co-developed in an agile and incremental manner, and have exactly
{\bmgitnumbercommits} \href{http://git-scm.com/}{\texttt{git}} commits
on \textit{\bmdoctimestamp}. For more, please see
\bmurl{https://github.com/bstarynk/bismon/commits/} for details. These
commits are too many and too fine grained to be considered as
``versions''. Given the agile and continuous workflow, it is
unreasonable, and practically impossible, to identify any formalized
versions.

\begin{relsize}{-0.5}
This document is co-developed with the \emph{Bismon} software itself,
it was typeset using \LaTeX~on Linux and contains some \emph{generated}
documentation~\footnote{The generated parts are clearly identified as
  such, and are extracted from the \emph{Bismon} system.}, mixed with
hand-written text. During development of \texttt{bismon}, the amount
of generated documentation will grow.  The entire history of
\emph{Bismon} (both the software -including its persistent store- and
this document) is available on
\bmurl{https://github.com/bstarynk/bismon/commits} and has, for this
document of commit id \texttt{\bmgitcommit} (done on \emph{\bmgitdate}) generated on
\textit{\bmdocdate}, exactly {\bmgitnumbercommits} commits (or
elementary changes). Since changes on any file in the \texttt{git}
repository can affect this document, no ``version'' is identifiable.
\end{relsize}
\bigskip

For convenience to the reader, here are the last~\footnote{Obtained by the
  \texttt{git log --name-status -3} command running in \emph{bismon}
  top source directory.} three \texttt{git commit}-s:

\begin{relsize}{-2}
  \input{generated/000-lastgitcommits.tex}
\end{relsize}

There is no notion of any identifiable ``version'' in \emph{bismon},
so also in this report. The work is incremental and the development is
agile.

\bigskip

%-  \subsection*{disclaimer}
%-
%-  \begin{center}
%-    \fbox{\parbox{0.92\textwidth}{ \small The content of the publication
%-        herein is the sole responsibility of the publishers and it does
%-        not necessarily represent the views expressed by the European
%-        Commission or its services.
%-
%-        While the information contained in
%-        the documents is believed to be accurate, the authors(s) or any
%-        other participant in the \textsc{Chariot} consortium make no
%-        warranty of any kind with regard to this material including, but
%-        not limited to the implied warranties of merchantability and
%-        fitness for a particular purpose.
%-
%-        Neither the \textsc{Chariot}
%-        Consortium nor any of its members, their officers, employees or
%-        agents shall be responsible or liable in negligence or otherwise
%-        howsoever in respect of any inaccuracy or omission herein.
%-        
%-        Without derogating from the generality of the foregoing neither
%-        the \textsc{Chariot} Consortium nor any of its members, their
%-        officers, employees or agents shall be liable for any direct or
%-        indirect or consequential loss or damage caused by or arising
%-        from any information advice or inaccuracy or omission herein.  }
%-    }
%-  \end{center}

    \smallskip



 \subsection*{copyright message}

 \begin{center}
   \Ovalbox{
     \parbox{0.95\textwidth}{
     \medskip
     
    
       Copyright ~ \textcopyright ~ 2018 - 2021 CEA {\relsize{-0.2}{(Commissariat à
           l'énergie atomique et aux énergies alternatives)}}.

       \bigskip 
         %
     This deliverable contains original unpublished work except where
     clearly indicated otherwise. Acknowledgement of previously
     published material and of the work of others has been made
     through appropriate citation, quotation or
     both. Reproduction is
     authorised provided the source is acknowledged  {\relsize{-0.5}{(see last page for licensing details)}}.
    

     \medskip
     }
     }
 \end{center}

 \smallskip
 

%-    \subsection*{Executive Summary}
%-
%-    This D1.3\textsuperscript{v1} \textsc{Chariot} deliverable is
%-    {\color{red}{(was)}} a first \textbf{draft} -and \emph{preliminary}-
%-    version (at M12) of D1.3\textsuperscript{v2} scheduled at M30 on
%-    ``Specialized Static Analysis tools for more secure and safer IoT
%-    software development''.
%-
%-  But {\color{red}{\textbf{incremental work has begun on
%-        D1.3\textsuperscript{v2} in early 2019}}}, hence the
%-  D1.3\textsuperscript{v2-} in the header.
%-
%-  It describes the \textsc{Chariot} vision on static source code (mostly
%-  of C and C++ code for IoT firmware and application) analysis. It
%-  proposes a \emph{simple} static analysis \emph{framework} leveraging
%-  on the powerful recent \emph{GCC} [cross-]compiler. A \emph{persistent
%-    monitor} (tentatively named \texttt{bismon}) is being designed and
%-  implemented as a GPLv3+ free software for Linux, using
%-  meta-programming techniques (leveraging on experience gained in the
%-  former GCC MELT project) to \emph{generate} GCC plugins, and able to
%-  keep some intermediate results (of compilation or static analysis)
%-  during the entire life of the IoT project, and giving to the IoT
%-  developers (thru a web interface) a whole-program view of the source
%-  code (as digested by the \emph{GCC} cross-compiler) and of its static
%-  analysis properties. That framework is configurable and scriptable by
%-  static analysis experts, hence permitting different IoT projects to
%-  address various concerns, while keeping the usual IoT development
%-  workflow (running as usual their \emph{GCC} cross-compiler on Linux,
%-  with extra plugin-related compilation flags).  The deliverable has
%-  been structured starting from the identification of the software and
%-  tool users and the document expected audience as well as the vision on
%-  specialized source code analysis towards more secure and safer IoT
%-  software development. The report then describes its strong alignment
%-  to adding capabilities to \emph{GCC} as well as the driving principles of the
%-  tools. Data and their persistence character are also described
%-  including mutable and non-mutable values/types while persistence is
%-  considered to start by loading some previous persisted state, usually
%-  dumping its current state before termination and loading the next
%-  state on the next load-up. The framework for static code analysis is
%-  also defined as part of the \emph{GCC} compilation process. The described
%-  work is also analysed in terms of contributing to other free software
%-  projects.
%-
%-  \medskip

\bigskip

\subsection*{notice}

\begin{quote}
\begin{relsize}{-0.5}
This work \index{h2020@\emph{\textsc{h2020}}} is funded (from start of
2018 to end of 2020) thru the \emph{\textsc{Chariot}} project (see its
web site on \bmurl{http://chariotproject.eu/}) which has received
funding from the European Union’s Horizon 2020 research and innovation
programme under the Grant Agreement No 780075. This work is also
\emph{partly} funded -from 2019 to 2021- by the
\href{http://decoder-project.eu/}{\textsc{Decoder}} H2020 project,
under its Grant Agreement 824231.
\end{relsize}
\end{quote}
\medskip



\hspace{2cm}

\end{titlepage}





\newpage

\tableofcontents

\newpage
\listoffigures

\medskip

\listoftables

\newpage

\rfoot{Page \thepage ~ {\relsize{-2}{of \pageref{LastPage}}}}
\newpage


\subsection*{Glossary of terms and abbreviations used}

\smallskip

\begin{tabular}{|c|p{0.7\textwidth}|}
  \hline
  \begin{minipage}{0.15\textwidth}

    \smallskip
    
    \textbf{Abbreviation / Term}

    \smallskip
    
  \end{minipage} & \textbf{Description} \\
  \hline \emph{binutils} \index{binutils@\emph{binutils}} & GNU free software package containing assembler \index{as@\texttt{as}} \texttt{as}, linker \texttt{ld} \index{ld@\texttt{ld}} and other utilities to operate on object files or binary executables, etc... \bmurl{https://www.gnu.org/software/binutils/} \\
  \hline \index{bismon@\texttt{bismon}} \texttt{bismon} & the free software framework and persistent monitor described here ; source repository on \bmurl{http://github.com/bstarynk/bismon/} \\
  \hline \textsc{Clang} \index{clang@\textsc{Clang}} & The Clang open-source project provides a language front-end and tooling infrastructure for languages in the C language family (C, C++, Objective C/C++, OpenCL, CUDA, and RenderScript) for the LLVM project \bmurl{http://clang.llvm.org/} \\
  \hline \textsc{Frama-C} \index{Frama-C@\textsc{Frama-C}} &  a free sofware extensible platform for analysis of C software \bmurl{http://frama-c.com/} \\
  \hline FSF & Free Software Foundation \bmurl{http://fsf.org/} \\
  \hline GCC \index{GCC} & Gnu Compiler Collection \bmurl{http://gcc.gnu.org/} \\
  \hline \textsc{Gcc Melt} \index{MELT@\textsc{Melt}} \index{GCC MELT@\textsc{Gcc Melt}} & was a (GPLv3+-licensed) GCC plugin and framework providing a DSL \index{DSL} to ease GCC extensions ; it is archived on \bmurl{http://starynkevitch.net/Basile/gcc-melt/}\\
  \hline \emph{Generic} \index{generic@\emph{Generic}} & language-independent abstract syntax tree (internal representation) in GCC \\
  \hline Gimple \index{gimple@\emph{Gimple}} & middle-end internal representation in GCC \\
  \hline GPL\index{GPL} & Gnu General Public Licence (a copylefted free software license) \bmurl{https://www.gnu.org/licenses/gpl.html} \\
  \hline IoT \index{IoT} & Internet of Things \\
  \hline \texttt{libonion} \index{libonion@\texttt{libonion}} & an HTTP server library \bmurl{https://www.coralbits.com/libonion/} \\
  \hline LLVM \index{LLVM} & The LLVM Project is an open-source collection of modular and reusable compiler and toolchain technologies \bmurl{http://www.llvm.org/} \\
  \hline MELT \index{MELT} & the Lisp-like domain specific language used in GCC MELT \\
  \hline Persistence \index{persistence} & From Wikipedia~: ``In computer science, \emph{persistence} refers to the characteristic of state that outlives the process that created it. This is achieved in practice by storing the state as data in computer data storage''. \bmurl{https://en.wikipedia.org/wiki/Persistence\_(computer\_science)} \\
  \hline RTL & (register transfer language) back-end internal representation in GCC \\
  \hline static code analysis \index{static code analysis} \index{code analysis} \index{analysis} & (or static program \index{program!analysis} analysis) ``is the analysis of computer software that is performed without actually executing programs, in contrast with dynamic analysis, which is analysis performed on programs while they are executing.'' (from Wikipedia: \bmurl{https://en.wikipedia.org/wiki/Static\_program\_analysis}). In this D1.3\textsuperscript{v1} report, it means static source code analysis, in practice analysis of C or C++ code for IoT fed to the \emph{GCC} compiler. \\
  \hline SSA \index{SSA} & Static Single Assignment (in GCC, a variant of Gimple) \\
  \hline
\end{tabular}

%%%%%%%%%%%%%%%%%%%%%%%%%%%%%%%%%%%%%%%%%%%%%%%%%%%%%%%%%%%%%%%%
%%%% the chapters
\newpage
% file intro-bm.tex, which is \input from bismon-chariot-doc.tex
\section{Introduction}
\label{sec:intro}

This D1.3~\textsuperscript{v1} \textsc{Chariot} deliverable is a first
\textbf{draft} -and \emph{preliminary}- version of
D1.3~\textsuperscript{v2} that will be formally submitted as the
complete and final deliverable at M30 on ``Specialized Static Analysis
tools for more secure and safer IoT software development''. This
deliverable targets software engineering (and indirectly also software
architects) experts working on IoT software coded in C or C++.

\subsection{Mapping \textsc{Chariot} output}
\label{subsec:mapchariot}

We refer to \textsc{Chariot} \emph{Grant Agreement} (GA). See table \ref{tab:mapchariot} below.


\begin{table}[!htbp]
  \caption{\label{tab:mapchariot} adherence to CHARIOT's GA deliverable and tasks descriptions}
  \medskip
  \begin{relsize}{-1.3}
    \begin{tabular}{|p{0.19\textwidth}|p{0.36\textwidth}|p{0.11\textwidth}|p{0.22\textwidth}|}
      \hline \textbf{\textsc{Chariot} GA component title} &
      \textbf{\textsc{Chariot} GA component outline} &
      \textbf{respective document chapter[s]} & \textbf{justification}
      \\ \hline \multicolumn{4}{|c|}{{\large \textbf{deliverable}}}
      \\
      \hline
%
      \begin{minipage}[t]{0.18\textwidth}
        \smallskip
        
        \textbf{{\relsize{+1.5}{D1.3}}} \\
        Specialized Static Analysis tools for
          more secure and safer IoT software development.
      \end{minipage}
              &
      % 
      {The source code (top level documentation) of the prototype static
      analysis tools developed in task T1.3, including the definition
      of data formats and protocols, updates and adapation of existing
      libraries and software components, the persistent monitor
      outline and documentation, anf features description and
      documentation of the compiler and linker extensions. An initial
      version set (V1) will be compiled by M12 followed by a revised
      version set (v2) in M30.}
      & {this whole document}      
      & {a single deliverable (with two versions of it, a preliminary
          draft one D1.3\textsuperscript{v1} and a final one
          D1.3\textsuperscript{v2}) describes the
          work. §\ref{sec:intro} is an overview and introduces the
          main concepts. §\ref{sec:datapersist} explains
          persistence. §\ref{sec:staticanalys} relates to static
          analysis. §\ref{sec:using} will become a user
          manual. §\ref{sec:miscwork} relates miscellanous work. The
          conclusion is in §\ref{sec:conclusion}.} \\      
      \hline
      \multicolumn{4}{|c|}{{\large \textbf{tasks}}} \\
      \hline
      \multirow{4}{*}{\begin{minipage}{0.19\textwidth}\smallskip
          \textbf{\relsize{+1.5}{T1.3}}          
       Specialized Static Analysis tools for
       more secure and safer IoT software development.
        \end{minipage}
      }
      & \textbf{ST1.3.1} definition of data formats and protocols
      & §\ref{sec:datapersist}; §\ref{subsec:compilinkext}
      & The persistent monitor data and format are described in §\ref{sec:datapersist}.
      The protocol to interact with \textsc{Chariot}'s blockchain is related to §\ref{subsec:compilinkext} and chapter 6 of D1.2 \\
      \cline{2-4}
      & \textbf{ST1.3.2} significant patches to existing free software components
      & §\ref{subsec:contribfree}
      & Section §\ref{subsec:contribfree} describe past work, and why future contributions to \emph{GCC} could be needed. \\
      \cline{2-4}
      & \textbf{ST1.3.3} design and implementation of the persistent monitor
      & §\ref{subsec:chariotvision};  §\ref{subsec:principles-bismon};  §\ref{subsec:multithreadist};  §\ref{sec:datapersist};
      &  §\ref{subsec:chariotvision} gives the \textsc{Chariot} vision of (informal) static analysis;
      §\ref{subsec:principles-bismon} explains the driving principles of our \emph{Bismon} persistent monitor,
      and (in  §\ref{subsec:multithreadist}) its multi-threaded and distributed aspects;
      §\ref{sec:datapersist} explains its persistent data.  \\
      \cline{2-4}
      & \textbf{ST1.3.4} design and implementation of the compiler and linker extension
      & §\ref{sec:staticanalys}; §\ref{subsec:compilinkext}
      & static analysis involves \emph{generated} GCC plugins, as (in this D1.3\textsuperscript{v1} preliminary draft)
      partly explained in §\ref{sec:staticanalys}; compiler and linker extensions
      are (in D1.3\textsuperscript{v1}) drafted in §\ref{subsec:compilinkext}  \\
      \hline
%-      \multirow{4}{*}{%
%-     \begin{minipage}[t]{0.18\textwidth}
%-       \smallskip
%-       
%-       \textbf{{\relsize{+1.5}{T1.3}}} \\
%-       Specialized Static Analysis tools for
%-       more secure and safer IoT software development.
%-
%-       \smallskip
%-      \end{minipage} &
%-      \emph{ST1.3.1} definition of the data format and protocols & §XXX & tobewrittenA \\
%-      \hline
%-      \emph{ST1.3.2} significant patches to existing free software components & §YYYY & tobewrittenB \\
%-      \hline
%-      \emph{ST1.3.3} design and implementation of the persistent monitor & §ZZZZ & hasbeenwritten \\
%-      \hline
%-      \emph{ST1.3.4} design and implementation of the compiler and linker extension & §TTTT & tobewrittenB \\
%-      \hline
       
      
    \end{tabular}
 \end{relsize}
\end{table}

\subsection{Deliverable Overview and Report Structure}
\label{subsec:overview}

This \textsc{Chariot} deliverable D1.3\textsuperscript{v1} is the
\emph{preliminary draft} of a report D1.2\textsuperscript{v2} scheduled at M30 on
\emph{Specialized Static Analysis tools for more secure and safer IoT
  software development} and relates to the work performed in
\textbf{T1.3} \emph{Specialized Static Analysis tools for more secure
  and safer IoT software development}.

The introduction (this §\ref{sec:intro}) describes the
\textsc{Chariot} vision on static source code (mostly of C and C++
code for IoT firmware and application) analysis (see
§\ref{subsec:chariotvision}), proposing a simple static analysis
\emph{framework} leveraging on the powerful recent \emph{GCC}
     [cross-]compiler and explaining the necessity of persistence,
     then gives the driving principles of our \emph{Bismon} persistent
     monitor (in §\ref{subsec:principles-bismon}); and explains its
     multi-threaded and distributed aspects (in
     §\ref{subsec:multithreadist}). The data and its persistence is
     detailed (in §\ref{sec:datapersist}, notably
     §\ref{subsec:dataproc} for the processed data,
     §\ref{subsec:gcvalobj} for the garbage collection,
     §\ref{subsec:persistence} for persistence).  The
     §\ref{sec:staticanalys} needs still to be mostly written and will
     describe (in the D1.3v2) how static analysis works. The
     §\ref{sec:using} will contain the (mostly generated) user
     documentation. The §\ref{sec:miscwork} describes some
     miscellanous work.


Related previous \textsc{Chariot} deliverables include: D1.1 (on
\emph{Classification and use guidelines of relevant standards and
  platforms}), which provides a taxonomy of standards and guidelines
(notably on cybersecurity, at a high and abstract level); but does
mention much source code (except as open source projects such as
IoTivity, FiWire, OM2M, etc). and D1.2 (on \emph{Method for coupling
  preprogrammed private keys on IoT devices with a Blockchain system})
which describes the \textsc{Chariot} blockchain and its \emph{Web API}
(which should be adapted into functions or libraries callable from C
code).

\subsection{Expected audience}
\label{subsec:audience}


The numerous footnotes in this report are for a second reading (and
may be used for forward references). To understand this report
describing a circular and reflexive system, you should read it twice
(skipping footnotes at the first read).

\bigskip

The reader of this document (within \textsc{Chariot}, a software
engineering expert working on IoT software or firmware coded in C or
C++) is expected to:

\begin{itemize}

  \item be fluent in C (cf. \cite{Kernighan:1988:CPL}) and/or C++
    (\cite{Stroustrup:2014:CplusPlus}) programming (notably on Linux
    and/or for embedded products, perhaps for IoT),

  \item be knowing a bit the \index{C11}{C11} standard
    (cf. \cite{C11:std,Memarian:2016:PLDI}) and/or the
    \index{C++11}{C++11} one (\cite{CplusPlus11:std}) and
    understanding well the essential notion of \index{undefined
      behavior}{\emph{undefined behavior}} \footnote{See
      \bmurl{http://blog.llvm.org/2011/05/what-every-c-programmer-should-know.html}
      and \bmurl{https://blog.regehr.org/archives/1520}} in C or C++
    programs,

  \item be a daily advanced user of \index{Linux}{Linux} for software
    development activities using GCC and related developer tools
    (e.g. \textit{binutils}, version control like \texttt{git}, build
    automation like \texttt{make} or \texttt{ninja}, source code
    editor like \texttt{emacs} or \texttt{vim}, the {\LaTeX} text
    formatter~\footnote{See
      \bmurl{https://www.latex-project.org/}. Some knowledge of
            {\LaTeX} is useful to improve or contribute to this
            document.}) on the \emph{command line}.
    
    \item be \emph{easily} able, in principle, to
      \textbf{compile}~\footnote{When compiling IoT software such as
        firmware, it usually is of course some
        \emph{cross}-compilation.}  his/her or other software coded in
      C (or in C++) \textbf{on the command line} (\emph{without} any
      \index{IDE!integrated development environment}IDE
      -\index{integrated development environment}~integrated software
      environment- or \index{SDK!software development kit} SDK
      -\index{software development kit}~software development kit-)
      with a \textbf{\emph{sequence} of \texttt{gcc} (or \texttt{g++})
        commands}~\footnote{In practice, we all use some \emph{build
          automation} tool, such as \texttt{make}, \texttt{ninja} or
        generators for them such as \texttt{cmake}, \emph{autoconf},
        \texttt{meson}, etc... But the reader is expected to be able
        to configure that, e.g. to add more options to \texttt{gcc} or
        to \texttt{g++} (perhaps in his/her \texttt{Makefile}) and is
        able to think in terms of a sequence of elementary
        \texttt{gcc} or \texttt{g++} compilation commands (or, when
        using Clang, \texttt{clang} or \texttt{clang++} commands).}
      \textbf{on Linux}.



\item to be capable of building large free software projects (such as
  the GCC compiler (cf \cite{gcc-internals} \footnote{See
    \bmurl{http://gcc.gnu.org} and notice that many cross-compiler
    forms of \emph{GCC} may need to be compiled from the source code
    of that compiler distributed by the \emph{FSF}, in particular
    because GCC plugin ability is needed within \textsc{Chariot}, or
    because hardware vendors provide only old versions of that
    compiler.}), the Linux kernel, the Qt toolkit and other open
  source projects of perhaps millions of source code lines) and
  smaller ones (e.g. \texttt{libonion} \footnote{see
    \bmurl{https://coralbits.com/libonion/}}) from their \emph{source}
  form.

\item have successfully downloaded and built the \emph{Bismon monitor}
  \index{Bismon} from its source code available on
  \bmurl{https://github.com/bstarynk/bismon}, on his Linux
  workstation.
  
\item have contributed or participated to some free software or open
  source projects and understanding their social (cf
  \cite{Raymond:2001:CathBaz}) and economical (cf
  \cite{Weber:2004:SuccessOpenSource, Tirole:2016:EcoBienCommun,
    Nagle:2018:Contributing, DiCosmo:1998:Holdup,
    Lerner-Tirole:2000:economics-open-source}) implications, the
  practical work flow, the importance of developer communities and of
  business~\footnote{See also Daniel Oberhous' blog February 2019 post
    on
    \url{https://motherboard.vice.com/en_us/article/43zak3/the-internet-was-built-on-the-free-labor-of-open-source-developers-is-that-sustainable}:
    \emph{The Internet Was Built on the Free Labor of Open Source
      Developers. Is That Sustainable?}} support.
  
\item be interested in static source code analysis, so have already
  tried some such tools like \emph{Frama-C} \footnote{See
    \bmurl{http://frama-c.com/}} (cf. \cite{Cuoq:2012:Frama-C}),
  \emph{Clang analyzer} \footnote{See
    \bmurl{https://clang-analyzer.llvm.org/}}, ..., and be aware of
  \index{compiler} compiler concepts and technologies (read
  \cite{Aho:2006:DragonBook}).

\item be familiar with operating systems principles
  (see \cite{Tanenbaum:92:OS,ArpaciDusseau14-Book}) and well
  versed in Linux programming
  (cf. \cite{Mitchell:2001:ALP,Kerrisk:2010:LinuxProgramming} \footnote{look
    into \texttt{man} pages on
    \bmurl{http://man7.org/linux/man-pages/}}).

  \item be interested in various programming languages
    (cf. \cite{Abelson1996:SICP,Scott:2007:PLP,Queinnec:1996:LSP}),
    including domain specific ones.

    \item is aware that \emph{most software projects
      fail}~\footnote{See
      \bmurl{https://www.geneca.com/why-up-to-75-of-software-projects-will-fail/}}
      (for \emph{some} definition of failure; see also \cite{Brooks:1995:MM, Khan:2016-GSEPIM,
      Attarzadeh:2008:proj-man}, etc...), and that obviously includes
      research software projects, which fail even more often, and any
      IoT software in general. I believe that such a high failure rate
      is
      \emph{intrinsic}~\footnote{\href{https://en.wiktionary.org/wiki/IMHO}{IMHO},
        allocation of much more time and efforts, including code
        reviews, on software development is necessary - but sadly it is not
        sufficient - to lower that failure rate. Read about the \emph{Joel Test} on \bmurl{https://www.joelonsoftware.com/2000/08/09/the-joel-test-12-steps-to-better-code/} for more.}  to any non-trivial
      software developed by humans (because of
      \cite{Braun:1956:magical-seven}, of ``leaky
      abstractions''~\footnote{Cf. Spolsky's \emph{Law of leaky
          abstractions} on
        \bmurl{https://www.joelonsoftware.com/2002/11/11/the-law-of-leaky-abstractions/}
        etc... for more.}  and of the
      \href{https://en.wikipedia.org/wiki/Halting\_problem}{\emph{Halting
          problem}}, etc...), and that formal methods approaches are
      still vulnerable to specification~\footnote{A speculative
        example of a tragic specification bug might include something
        inside the Boeing 737 MAX - see
        \bmurl{https://en.wikipedia.org/wiki/Boeing\_737\_MAX} - which
        could have recent crashes related to bugs in specifications,
        and probably developed with the most serious formal methods
        approaches, dictated by DO-178-C -see
        \bmurl{https://en.wikipedia.org/wiki/DO-178C} etc... But in
        mid-2019 this is only a speculation (details are unknown to
        me).} bugs.

\end{itemize}


\bigskip

\subsection{The \textsc{Chariot} vision on specialized static source code analysis for more secure and safer IoT software development}
\label{subsec:chariotvision}
\subsubsection{About static source code analysis and IoT}

There are many existing documents related to improving safety and
security in IoT software (e.g. \cite{Chen:2011:DAS, Medwed:2016:ISC}),
and even more on static source code analysis in general
(cf. \cite{Gomes2009AnOO, GosevaPopstojanova2015OnTC,
  Binkley:2007:SCA} and many others). %

Several conferences are dedicated to static analysis~\footnote{The
  25\textsuperscript{th} Static Analysis Symposium will happen in
  august 2018, see
  \bmurl{http://staticanalysis.org/sas2018/sas2018.html}; most ACM
  SIGPLAN conferences such as POPL, PLDI, ICFP, OOPSLA, LCTES, SPLASH,
  DSL, CGO, SLE... have papers related to static source code
  analysis.}.  All dominant C compilers (notably GCC and Clang, but
also MicroSoft's \emph{Visual C}\texttrademark) are using complex
static source code analysis techniques for optimizations and warnings
purposes (and that is why C compilers are monsters~\footnote{see
  \bmurl{https://softwareengineering.stackexchange.com/a/273711/40065}
  for more.}). It is wisely noticed (in
\cite{GosevaPopstojanova2015OnTC}) that \textbf{state-of-the-art
  static source code analysis tools are \emph{not} very effective in
  detecting \index{vulnerability!security} security
  vulnerabilities}~\footnote{Se we can only hope an \emph{incremental}
  progress in that area. Static source code analysis in
  \textsc{Chariot} won't make miracles.}, so they are not a ``silver
bullet'' \index{silver bullet} (\cite{Brooks:1987:NSB}). Many
taxonomies of software defects \index{defect!software} already exist
(e.g. \cite{Silva:2016:SES, Wagner:2008:DCD, Levine:2009:DDE}
etc....), notably for IoT\index{IoT} (see \cite{Carpent:2018:RRA,
  Ahmad:2018:ModelBasedIoT, Laszlo:2017:Vessedia}); however the
relation between an explicit defect and a source code property is
generally fuzzy, or ill-defined.

Static source code analysis tools can -generally speaking-
be~\footnote{This is a gross simplification! In practice, there is a
  continuous spectrum of source code analyzers, much like there is a
  spectrum between compilers and interpreters (with e.g. bytecode or
  JIT implementations sitting in between compilation and naive
  interpretation).} viewed as being of one of two kinds:

\begin{itemize}
  \item strongly formal methods based, semantically oriented,
    ``sound'' tools (e.g. built above abstract interpretation -cf.
    \cite{Cousot:2014:AIP,CousotCousot77-1}-, model checking -cf.
    \cite{Schlich:2010:MCS, Siddiqui:2018:adv-soft-model-check} and \cite{Jhala:2009:SMC}-, and other
    formal techniques on the whole program... See also
    \cite{Andreasen:2017:SAI}) which can give excellent results but
    require a lot of expertise to be used, and may take a long time to
    run~\footnote{There are cases where those static analyzers need
      weeks of computer time to give interesting results.}. For
    examples, \emph{Frama-C} (cf \cite{Cuoq:2012:Frama-C}),
    \emph{Astrée} (cf \cite{Mine:2015:TIU}), etc... The expert user
    needs either to explicitly annotate the analyzed source code
    (e.g. in ACSL for \emph{Frama-C}, see \cite{Baudin:2018:ACSL,
      Delahaye:2013:CSL, Amin:2017:LAW}), and/or cleverly tune the
    many configuration knobs of the static analyzer, and often
    both. Often, the static analyzer itself has to be extended to be
    able to give interesting results on one particular analyzed source
    code~\footnote{The \emph{Astrée} project can be seen as the
      development of a powerful analyzer tool \emph{specifically}
      suited for the needs of Airbus control command software; it
      implements many complex abstract interpretation lattices wisely
      chosen to fit the relevant traits of the analyzed code. Neither
      \emph{Astrée} nor \emph{Frama-C} can easily -without any
      additional tuning or annotations- and successfully be used on
      most Linux command line utilities (e.g. \texttt{bash},
      \texttt{\emph{coreutils}}, \texttt{\emph{binutils}},
      \texttt{gcc}, ...)  or servers (e.g. \texttt{systemd},
      \texttt{lighttpd}, \texttt{Wayland} or \texttt{Xorg}, or IoT
      frameworks such as MQTT...). But \emph{Frama-C} can be extended
      by additional plugins so is a \emph{framework} for sound static
      analysis.}, when that analyzed code is complex or quite
    large. Many formal static analyzers
    (e.g. \cite{Greenaway:2014:DSS, Vedala:2012:ADP}) focus on
    checking just \emph{one} aspect or property of security or
    safety. Usually, formal and sound static analyzers can practically
    cope only with small sized analyzed programs of at most one or a
    few hundred thousands lines of C code (following some
    \emph{particular} coding style or using some definable
    \emph{subset} of the C language~\footnote{For instance, both
      \emph{Frama-C} and \emph{Astrée} have issues in dealing with
      standard dynamic C memory allocation above \texttt{malloc};
      since they target above all the safety critical real-time
      embedded software market where such allocations are
      \emph{forbidden}.}).  In practice, the formal sound static
    analyzers are able to prove \emph{automatically} some
    \emph{simple} properties of small, highly critical, software
    components (e.g. avoiding the need of \emph{unit testing} at the
    expense of very \emph{costly software development efforts}).

  \item lightweight ``syntax'' oriented ``unsound'' tools, such as
    Coverity Scan~\footnote{See \bmurl{https://scan.coverity.com/}} or
    Clang-Analyzer, or even recent compilers (GCC or Clang) with all
    warnings and link-time optimization~\footnote{Link-time
      optimization (e.g. compiling and \emph{linking} with \texttt{gcc
        -O2 -flto -Wall -Wextra} using GCC) slows down the build time by more than a
      factor of two since the intermediate internal representation (IR) of the
      compiler (e.g. Gimple \index{Gimple} for \emph{GCC}, see \cite{gcc-internals} §12) is kept in object files and
      reload at ``link-time'' which is done by the \emph{compiler}
      working on the whole program's IR, so is rarely used.}  enabled. Of
    course, these simpler approaches give many false positive warnings
    (cf \cite{Nadeem:2012:HFP}), but machine learning techniques (cf
    \cite{Perl:2015:VFP}) using bug databases could help.
    
\end{itemize}
Most static analyzers require some kind of interaction with their user
(cf \cite{Lipford:2014:ICA}), in particular to present partial
analysis results and explanations about them, or complex information
like control flow graphs, derived properties at each statements.

The \href{http://vessedia.eu/}{\textsc{Vessedia}} \index{Vessedia} project is an H2020
IoT-related project~\footnote{The \textsc{Vessedia} project (Verification
  Engineering of Safety and Security Critical Industrial Applications)
  has received funding from the European Union's Horizon 2020 research
  and innovation programme under grant agreement No. 731453, within
  the call H2020-DS-LEIT-2016, started in january 2017, ending in
  december 2019.} which is focusing on a strong formal methods
approach for IoT \emph{software} and insists on a ``single system
formal verification'' approach (so it makes quite weak hypothesis on
the ``systems of systems'' view); it ``aims at enhancing safety and
security of information and communication technology (ICT) and
especially the Internet of Things (IoT). More precisely the aim of
[the VESSEDIA] project consists in making formal methods more
accessible for application domains that want to improve the security
and reliability of their software applications by means of Formal
Methods''~\footnote{Taken from
  \bmurl{https://vessedia.eu/about}}. Most \textsc{Vessedia} partners
(even the industrial ones, cf. \cite{Berkes:2018:Vessedia-approach}) are versed in formal static analysis
techniques (many of them being already trained to use \emph{Frama-C}
several years before, and several of them contributing actively to
that tool.). Some of the major achievements of \textsc{Vessedia}
includes formal (but \emph{fully automatic}) proofs of often
\emph{simple} (and sometimes very complex and very specific)
properties of some basic software components (e.g. lack of undefined
behavior in the memory allocator, or the linked list implementation,
of Contiki). Some automatically proven properties can be very complex,
and the very hard work~\footnote{In terms of human efforts, the
  formalization of the problem, and the annotation of the code and
  guidance of the prover, takes much more time and money (often more
  than a factor of 30x) than the coding of the C code. Within
  \textsc{Chariot} no large effort is explicitly allocated for such
  concrete but difficult tasks.}  is in formalizing these properties
(in terms of C code!)  and then in assisting the formal tool to prove
them.


In contrast, \textsc{Chariot} focuses mainly on a \emph{systems of
  systems} (e.g. networks of systems and systems of networks)
approach, so~\footnote{Taken in october 2018 from \bmurl{https://www.chariotproject.eu/About}, \emph{§Technical Approach}.}
``aims to address how safety-critical-systems should
be securely and appropriately managed and integrated with a fog
network made up of heterogeneous IoT devices and gateways.''. Within
\textsc{Chariot}, static analysis methods have to be ``simple'' and
support its \emph{Open IoT Cloud Platform} thru its \emph{IoT Privacy,
  Security and Safety Supervision Engine} \index{IPSE}~\footnote{Taken
  from \bmurl{https://www.chariotproject.eu/About/}}, and some
industrial \textsc{Chariot} partners, while being IoT network and
hardware experts, are noticing that their favorite IDE (provided by
their main IoT hardware vendor) is running some GCC under the hoods
during the build of their firmware, but are not used to static source
code analysis tools. The \textsc{Chariot} approach to static source
code analysis does \emph{not} require the same level of expertise as needed for
the \emph{Verified in Europe} label pushed by the \textsc{Vessedia}
project.

The \textbf{\textsc{Chariot} approach} to static source analysis
\textbf{leverages on} an \textbf{\emph{existing}} \emph{recent}
\textbf{GCC cross-compiler}~\footnote{The actual version and
  the concrete configuation of \emph{GCC} are important; we want to stick -when
  reasonably possible- to the latest GCC releases, e.g. to
  \href{https://gcc.gnu.org/gcc-8/}{GCC 8} in autumn 2018. In the
  usual case, that \emph{GCC} is a cross-compiler. In the rare case
  where the IoT system runs on an x86-64 device under Linux, that
  \emph{GCC} is not a cross-, but a straight compiler.}. So the IoT
developer following the \textsc{Chariot} methodology would just add
some additional flags to \emph{existing} \texttt{gcc} or \texttt{g++}
cross-compilation commands, and needs simply to change slightly
his/her build automation scripts (e.g. add a few lines to his
\texttt{Makefile}). Such a gentle approach (see figure
\ref{fig:chariotcompil}) has the advantage of not disturbing much the usual
developer workflow and habit, and addresses also the \emph{junior} IoT
software developer. Of course the \emph {details} of compilation
commands would change, the commands shown in the figure \ref{fig:chariotcompil} are grossly
simplified! The compilation and linking processes are
communicating -via some additional \emph{GCC} plugins
(cf. \cite{gcc-internals} §24) doing inter-process communication- with
our \index{persistent monitor} \index{persistent!monitor}
\emph{persistent monitor}, tentatively called \texttt{bismon}
\index{Bismon}. It is preferable (see
\cite{gcc-runtime-library-exception}) to use free software GCC plugins
(or free software generators for them) when compiling proprietary
firmware with the help of these plugins; otherwise, there might
be~\footnote{Of course, I -Basile Starynkevitch- am not a lawyer, and
  you should check any potential licensing issues with your own
  lawyer.} some licensing issues on the obtained proprietary binary
firmware blob, if it was compiled with the help of some hypothetical
\emph{proprietary} GCC plugin.

\begin{figure}[h]
  \begin{center}
    \bmincludewidthgraphics{0.85\textwidth}{chariot-compil-fig}{eps}{svg}
  \end{center}
  \caption{the \textsc{Chariot} compilation of some IoT firmware or
    application {\textbf{(simplified)}}}
  \label{fig:chariotcompil}
\end{figure}

%No \textsc{Chariot} industrial
%partner have any prior extensive experience of static formal-methods
%based source code analysis techniques and tools such as
%\emph{Frama-C}. Most of them don't even build their firmware on a
%Linux workstation (but depend on some proprietary IDE containing an
%obsolete version of \emph{GCC} and have to run that on Windows.).

\subsubsection{The power of an existing compiler: GCC}

\textsc{Chariot} static analysis tools will leverage on the mainstream
\textsc{Gcc}~\footnote{``Gnu Compiler Collection''. See
  \bmurl{http://gcc.gnu.org/} for more. In practice, it is useful to
  build a recent \textsc{Gcc} cross-compiler, fitted for your IoT
  system, from its published source code - see
  \bmurl{https://gcc.gnu.org/mirrors.html} for a list of mirrors.}
compiler (generally used as a
\emph{\index{compiler}\index{cross-compiler}cross-compiler} for IoT
firmware development) Current versions of \emph{GCC} (that is, GCC 8.2
as of September 2018) are capable of quite surprising optimizations
(internally based upon some sophisticated static analysis techniques
and advanced heuristics). But to provide such clever optimizations,
the \textbf{\emph{GCC} compiler has to be quite a large software, of
  more than 5.28 millions lines}~\footnote{Measured by
  \href{https://dwheeler.com/sloccount/}{David Wheeler's
    \texttt{sloccount}} utility} of source code (in
\texttt{gcc-8.2.0}, measured by \texttt{sloccount}). This figure is an
under-estimation~\footnote{The Unix \texttt{wc} utility gives 14.6
  millions lines, including empty ones but excluding generated C++
  code, in 498 megabytes.}, since \emph{GCC} contains a \emph{dozen of
  domain specific languages} and their transpilers to generated C++
code, which are not well recognized or measured by \texttt{sloccount}.


\medskip

We show below several examples of optimizations done by recent
\emph{GCC} compilers. Usually, these optimizations happen in the
middle-end and work on internal intermediate representations which are
mostly not~\footnote{Most of the internal \emph{GCC} representations
  -e.g. \emph{Gimple} or \emph{SSA}- are common to all target systems;
  however, some constants, like the size and alignment of primitive
  data types such as \texttt{long} or pointers, are known at
  preprocessing phase or at early \emph{Gimplification} phase.} target
specific.

\bigskip

\bigskip

{{\raisebox{3pt}{\textcolor{brown}{\rule{0.2\textwidth}{2.0pt}}}} ~ \large{recursive inlining with constant folding}}



\begin{table}[!htbp]
\caption{\label{tab:factinlinecsrc} recursive inlining with constant folding in \emph{GCC} (C source)}
   \medskip
  \begin{center}
    \begin{relsize}{-1.2}
     \begin{tabular}{c}
      \lstinputlisting[language=C]{examples/factinline12.c} \\ 
       \textbf{\emph{C source code}} \\ 
       \hline
     \end{tabular}
    \end{relsize}
  \end{center}
\end{table}

The table \ref{tab:factinlinecsrc} shows a simple example of C code
(file \texttt{factinline12.c}). After preprocessing and parsing, it
becomes quickly expanded in some \emph{Gimple}\index{gimple}
representation (cf. §12 of \cite{gcc-internals}), whose elementary
instructions are arithmetic with two operands, or simple tests with
jumps, or simple calls (in so called \emph{A-normal form}
\index{A-normal form}, where nested calls like \verb+a=f(g(x),y);+ get
transformed into a sequence introducing a temporary $\tau$ such as
$\tau$\verb+=g(x)+ then \verb+a=f(+$\tau$\verb+,y)+, etc...), shown in
table \ref{tab:factinlinegimple}.

\input{generated/002-gimple-factinline.tex}

This is a textual (and quite
\emph{incomplete} since a partial view of some) dump of some in-memory
\emph{internal intermediate representation} during compilation. What
really matters to the \textsc{Chariot} static source code analyzer
framework is the data inside the compilation process \texttt{cc1}, not
its partial textual dump~\footnote{It is possible to pass the
  \texttt{-fdump-tree-all} flag to \texttt{gcc}; then hundreds of
  intermediate textual dump files are emitted, including
  \texttt{factinline12.c.004t.gimple} and
  \texttt{factinline12.c.020t.ssa} and many others for the compilation
  of \texttt{factinline12.c} source file.}. The \emph{Gimple} internal
in-memory representation is declared inside several source files of
\emph{GCC}, including its \texttt{gcc-8.2.0/gcc/gimple.def},
\texttt{gcc-8.2.0/gcc/gimple.h},
\texttt{gcc-8.2.0/gcc/gimple-iterator.h}, etc...


After gimplification, many other optimizations happen. \textbf{The
  \emph{GCC} compiler runs more than two hundred optimization passes
  !}\index{optimization}\index{pass!optimization}. The table
\ref{tab:factinlinessa} shows the ``static single assignment'' form
(SSA, see \cite{pop:ssa} and \cite{gcc-internals} §13), where variables are duplicated so that each SSA
variable gets assigned only \emph{once}. The control flow is reduced
to \index{basic block} \emph{basic blocks} (with a single entry point
at start, and perhaps several conditional exit edges). Then special
$\phi$ nodes introduce places where such a variable may come from two
other ones (after branches).

\input{generated/003-ssaopt-factinline.tex}


At last, many other optimizations happen. And the optimized form in
table \ref{tab:factinlineoptim} shows that \texttt{fact12} just
returns the constant 479001600 (which happens to be the result of
\texttt{fact(12)} computed \emph{at compile-time}).

\newpage

Finally, the generated assembler code has no trace of \texttt{fact}
function, and contains just what is shown in table
\ref{tab:factinlineasm} (where many useless comment lines, giving the
detailed configuration of the cross compiler, have been removed).



\pagebreak



{{\raisebox{3pt}{\textcolor{brown}{\rule{0.2\textwidth}{2.0pt}}}} ~ \large{heap allocation optimization}}

\bigskip

Our second example shows that \emph{GCC} is capable of clever
optimizations around dynamic heap allocation and de-allocation. Its
source code in file \texttt{mallfree.c} is shown in table
\ref{tab:mallfreecsrc}.

\begin{table}[!htbp]
\caption{\label{tab:mallfreecsrc} optimization around heap allocation by \emph{GCC} (C source)}
   \medskip
  \begin{center}
    \begin{relsize}{-1.2}
     \begin{tabular}{c}
      \lstinputlisting[language=C]{examples/mallfree.c} \\ 
       \textbf{\emph{C source code}} \\ 
       \hline
     \end{tabular}
    \end{relsize}
  \end{center}
\end{table}

 The straight \emph{GCC} compiler~\footnote{Similar optimizations also
   happen with a GCC MIPS targetted cross-compiler.} (on Linux/x86-64)
 is optimizing and able to remove the calls to \texttt{malloc} and to
 \texttt{free}, following the \index{as-if rule} \href{https://en.wikipedia.org/wiki/As-if\_rule}{\emph{as-if rule}}.

 %\smallskip

The \emph{Gimple} form shown in table
\ref{tab:mallfreegimple}. Pointer arithmetic has been expanded to
target-specific \emph{address} arithmetic in byte units, knowing that
\texttt{sizeof(int)} is 4.

\include{generated/004-gimplessa-mallfree}

 \smallskip

\begin{table}[H]
\caption{\label{tab:mallfreegimple} optimization around heap
  allocation by \emph{GCC} (generated Gimple)}
   \medskip
  \begin{center}
    \begin{relsize}{-1.5}
     \begin{tabular}{c}
      \lstinputlisting[language=C]{generated/mallfree-gimple.c} \\ 
       \textbf{\emph{Gimple code}} \\ 
       \hline
     \end{tabular}
    \end{relsize}
  \end{center}
\end{table}

\medskip

The \emph{SSA/optimized} form appears in table
\ref{tab:mallfreeoptim}. It shows that the call to \texttt{malloc} and
to \texttt{free} have been optimized away, so the \texttt{weirdsum}
function don't use heap allocation anymore.

\begin{table}[H]
\caption{\label{tab:mallfreeoptim} optimization around heap allocation
  by \emph{GCC} (generated SSA/optimized)}
   \medskip
  \begin{center}
    \begin{relsize}{-1.5}
     \begin{tabular}{c}
      \lstinputlisting[language=C]{generated/mallfree-optimized.c} \\ 
       \textbf{\emph{SSA/optimized code}} \\ 
       \hline
     \end{tabular}
    \end{relsize}
  \end{center}
\end{table}

So the generated x86-64 assembler code in table \ref{tab:mallfreeasm}
contain no calls to \texttt{malloc} or \texttt{free}, hence contains the same code that would be generated from just \texttt{int weirdsum(int x, int y) \{return x+y;\}}.


\smallskip

\begin{table}[t]
\caption{\label{tab:mallfreeasm} optimization around heap allocation
  by \emph{GCC} (generated x86-64 assembly)}
   \medskip
  \begin{center}
    \begin{relsize}{-1.5}
     \begin{tabular}{c}
       %\lstinputlisting{language=[x64]Assembler}{generated/mallfree-tail.s} \\
       \begin{minipage}{0.8\textwidth}
         %\listinginput[4]{1}{generated/mallfree-tail.s}
         \VerbatimInput{generated/mallfree-tail.s}
       \end{minipage} \\
       \textbf{\emph{x86-64 assembly}} \\ 
       \hline
     \end{tabular}
    \end{relsize}
  \end{center}
\end{table}

\medskip

This \texttt{mallfree.c} example could look artificial (because human
developers won't code this way). However, a similar example might
happen in real life after preprocessor expansion and inlining in large
header-mostly libraries. In addition, most genuine C++11 \index{C++11} code
(e.g. using standard container \index{container!C++} templates from \texttt{<map>} or
\texttt{<vector>} standard headers) would produce conceptually similar
code (since many standard constructors and destructors would call
internally the standard \texttt{::operator new} and \texttt{::operator
  delete} operations, which get inlined into calling the system
\texttt{malloc} and \texttt{free} functions).

\newpage
%%%%%%%%%%%%%%%%%%%%%%%%%%%%%%%%%%%%%%%%%%%%%%%%%%%%%%%%%%%%%%%%


\subsubsection{Leveraging simple static source analysis on \emph{GCC}}
\label{subsubsec:leveraging-static-analysis}

By hooking through \textsc{Chariot} specific GCC
\emph{plugins}~\footnote{Notice that GCC plugins work mostly on Linux
  -but not really on Windows-, so this explains the \textsc{Chariot}
  requirement of [cross-]compiling IoT firmware on a Linux
  workstation.}  into usual [cross-] \emph{compilation} processes
(some \texttt{gcc} or \texttt{g++}, etc... such as a
\texttt{mips-linux-gnu-gcc-8} or a \texttt{arm-linux-gnueabi-g++-8},
etc...), IoT software developers will be able to take advantage of all
the numerous optimizations and processing done by \emph{GCC}. However,
a typical firmware build would take many dozens of such compilation
processes, since every translation unit (practically \texttt{*.c} C
source files and \texttt{*.cc} C++ source files of the IoT firmware)
requires its compilation process~\footnote{Technically, a C
  compilation process would be running some \texttt{cc1} internal
  executable started by some \texttt{*gcc*} [cross-] compilation
  command.}. In practice, IoT software developers would use some
\emph{existing} \textbf{build automation} tool (such as \texttt{make},
\texttt{ninja}, \texttt{meson}, \texttt{cmake} etc...)  \index{build
  automation} which is running suitable compilation commands. They
would need to adapt~\footnote{How to adapt cleverly their
  \texttt{Makefile} to take advantage of \textsc{Chariot} provided
  static analysis is of course the responsability of the IoT
  developer. Surely several extra lines are needed, and the
  \texttt{CFLAGS=} line of their \texttt{Makefile} should be
  changed. For other builders such as \texttt{ninja}, etc..., some
  similar configuration changes would be needed.}  and configure their
build process (e.g. by editing their \texttt{Makefile}-s, etc...),
notably to fetch their \emph{GCC} plugin C++ code and compile it into
some \texttt{*.so} shared object to be later \texttt{dlopen}-ed by
some cross-compiler \texttt{cc1} process, and to use these plugins in
their cross-compilation of their IoT firmware. By working in
cooperation with existing \emph{GCC} compilation tools, the IoT
developer don't have to change much his/her current development
practices. However, these various compilation processes are producing
partial new (or updated) internal representations, and need to know
about other translation units. So some \textbf{persistence is needed}
to keep some data (such as the control flow graph, etc...) related to
past~\footnote{Notice that recent \emph{GCC} provide a link-time
  optimization (LTO) ability, if the developer compiles \emph{and
    links} with e.g. \texttt{gcc -O2 -flto}. But LTO is not widely
  used since it slows down a lot the building time, and the plugin
  infrastructure of GCC is not very LTO friendly and would be brittle
  to use. At last, there won't be any practical user interface with
  such an approach. So persistence is practically needed, both without
  LTO and if using LTO.}  compilation processes during perhaps the
entire project development work.

IoT developers need to interact with their static source code analysis
(\emph{GCC} based) tool. In particular, they might need to understand
more the optimizations done by their (\textsc{Chariot} augmented, thru
\emph{GCC} plugins) compiler, and they also need to be able to query
some of the numerous intermediate data related to that static analysis
and compilation. Practically, a small team of IoT developers working
on the same IoT firmware project would interact with the static
analysis infrastructure, that is with a single \emph{persistent
  monitor} process. That \textbf{persistent monitor} (\emph{Bismon})
would be some ``server-like'' program, started probably every working
day in the morning and loading its previous state, and gently stopped
every evening and dumping its current state on disk. It needs to keep
its persistent state in some files. For convenience, a textual format
is preferable~\footnote{This is conceptually similar the the SQL dump
  files of current RDBMS. But of course \emph{Bismon} don't use an SQL
  format, but its own textual format.}. These persistent store files
could (and actually should) be regularly backed up and kept in some
version control system.

Since a single \emph{Bismon} process is used by a small team of IoT
developers, it provides some web interface\index{web interface}: each
IoT developer will interact with the persistent monitor through his/her
web \index{browser} browser~\footnote{We don't aim compatibility with
  all web browsers -including old ones- but just with the latest
  \emph{Chrome} (v.70 or newer) and \emph{Firefox} (v.63 or newer)
  browsers, using HTML5, JavaScript, WebSockets, AJAX
  technologies.}. In addition, a static analysis expert (which could
perhaps be the very senior IoT developer of the team) will configure
the static analysis (also through a web interface).

The figure \ref{fig:bismonit} gives an overall picture: on the top,
both Alice and Bill are working on the same IoT project source code
(and each have a slightly different version of that code, since Alice
might work on routine \texttt{foo\_start} in file \texttt{foo.cc},
while Bill is coding the routine \texttt{bar\_event\_loop} in file
\texttt{bar.c}). Both Alice and Bill (IoT developers in the same team,
working on the same IoT firmware) are compiling with the same
\emph{GCC} cross-compiler (the GCC egg) enhanced by a plugin (the
small green puzzle piece, at right of GCC eggs). They use their
favorite editor or IDE to work on the IoT source code, and run from
time to time a builder (e.g. \texttt{make}). They use a browser (with
a rainbow in the figure) to interact with the monitor and query static
analysis data. The purple dashed arrows represent HTTP exchanges
between their browser and the monitor. The compilation processes,
extended by the GCC plugin, communicate with the monitor (thru the
gray dashed arrows). A static analysis expert (the ``geek'', at left
of the Bismon monitor) is configuring the monitor thru his own web
interface. The monitor is \emph{generating} (orange arrow) the C++
code for GCC plugins (small blue hexagon at right), and the IoT
developer needs to change his/her build procedures to compile and use
that generated GCC plugin. \emph{Bismon} also uses meta-programming
techniques to emit (curved blue arrow to left) internal code (bubble)
- notably C code dynamically loaded by the monitor and JavaScript/HTML
used in browsers. The several ``Tux'' penguins remind that all
this (cross-compilers, builders, the persistent monitor, etc...)
should run on Linux systems. The monitor persists its entire state on
disk, so can restart in the state that it has dumped in its previous
run.

\begin{figure}
  \begin{center}
    \bmincludewidthgraphics{0.85\textwidth}{bismon-monitor-fig}{eps}{svg}
  \end{center}
  \caption{The \emph{Bismon} monitor used by some IoT developer team following the \textsc{Chariot} approach.}
  \label{fig:bismonit}
\end{figure}

\newpage 
%%%%%%%%%%%%%%%%%%%%%%%%%%%%%%%%%%%%%%%%%%%%%%%%%%%%%%%%%%%%%%%%


\subsection{Lessons learned from \textit{GCC MELT}}
\label{subsec:lessonsgccmelt}

Our previous \emph{GCC MELT} project (see \cite{Starynkevitch-DSL2011,
  Starynkevitch-GCCMELTweb, Starynkevitch2007Multistage}) provided a
bootstrapped Lisp-like dialect (called \emph{MELT}) to easily extend
\emph{GCC}. That Lisp-like dialect was translated to
\emph{GCC}-specific C++ code (in a transpiler itself coded in
\emph{MELT}). Then \emph{GCC MELT} gave us the following insights:

\begin{itemize}
  \item the \emph{GCC} software is a huge free software project (with
    several hundreds of contributors, many of them working on GCC at
    least half of their time), which is \emph{continuously}
    evolving. Following that evolution requires a significant effort
    by itself (see for example the mail traffic -of several hundreds
    messages monthly- on
    \href{https://gcc.gnu.org/ml/gcc/}{\texttt{gcc@gcc.gnu.org}} and
    \href{https://gcc.gnu.org/ml/gcc-patches/}{\texttt{gcc-patches@gcc.gnu.org}},
    and the amount of work shown at GCC summits or GNU Tools Cauldron,
    etc...). Switching to \emph{Clang} would also require a lot of
    efforts.


  \item pushing a patch or contribution inside \emph{GCC} is very
    demanding, since the community is quite strict (but that explains
    the quality of GCC).

  \item the \emph{GCC} plugin API~\footnote{including of course the
    API related to internal GCC representations, such as \emph{Gimple}
    or \emph{SSA}.} is not ``carved in stone'', and can evolve
    incompatibly from one release of \emph{GCC} to the
    next. Therefore, the C++ code of a plugin for \texttt{gcc-7} may
    require some (perhaps non-trivial) modifications to be usable on
    \texttt{gcc-8}, etc...

  \item a lot of C++ code (nearly two millions lines) was generated
    within our \emph{GCC MELT} project, and the compilation by
    \texttt{g++} into a shared object of that emitted C++ code was a
    significant bottleneck.

  \item Implementing a generational copying garbage collector above
    \emph{Ggc}~\footnote{This is the internal GCC garbage collector, a
      mark-and-sweep one which can run only \emph{between} GCC
      passes.} was challenging (debugging GC code takes a lot of time).

    \item Describing, in MELT, the interface to the
      many~\footnote{Current \emph{GCC-8} have several thousands
        \emph{GTY}-ed classes.} C++ classes internal to \emph{GCC}
      takes a lot of effort. Some automation would be helpful.

   \item In practice, whole-program static analysis requires a
     persistent system, since the ``link-time optimization'' ability
     of \emph{GCC} is not plugin-friendly and is very rarely used.
\end{itemize}

The final D1.3\textsuperscript{v2} document may add further relevant
items to the above list.

\medskip

\subsection{Driving principles for the  \textit{Bismon} persistent monitor}
\label{subsec:principles-bismon}

To enable whole program static source code analysis (for IoT software
developers coding in C or C++ on a Linux developer workstation), we
are developing \emph{Bismon}~\footnote{Notice \texttt{bismon} is a
  \textbf{temporary name} and could be changed, once we find a better
  name for it. Suggestions for better names are welcome.}, a
persistent monitor (free software, for Linux). It leverages above
existing recent \emph{GCC} [cross-] compilers.

%%%%%%%%%%%%%%%%

\subsubsection{About \emph{Bismon}}
\label{subsubsec:about-bismon}

\textit{Bismon} (a \emph{temporary
  name})\index{Bismon} is a free software (GPLv3+ licensed)\footnote{The
  source code is unreleased but available, and continuously evolving,
  on \bmurl{https://github.com/bstarynk/bismon}} static source code
whole-program analysis framework whose initial domain will be
\emph{Internet of Things} (or \index{IoT}{IoT})\footnote{IoT is viewed
  as the first application domain of \textit{Bismon}, but it is hoped
  that most of \textit{Bismon} could be reused and later extended for
  other domains}. It is designed to work with the \textit{Gcc}
compiler (see \bmurl{gcc.gnu.org}) on a Linux
workstation\footnote{Linux specific features are needed by
  \textit{Bismon}, which is unlikely to be buildable or run under
  other operating systems. My Linux distribution is
  \emph{Debian/Unstable}}. \textit{Bismon} is conceptually the
successor of \textit{GCC MELT} \footnote{The \textit{GCC MELT} web
  pages used to be on \texttt{gcc-melt.org} -a DNS domain relinquished
  in april 2018- and are archived on
  \bmurl{https://starynkevitch.net/Basile/gcc-melt}} (see
\cite{Starynkevitch2007Multistage, Starynkevitch-DSL2011}), but don't
share any code with it while retaining several principles and
approaches of \emph{GCC MELT}.

\textit{Bismon} is \textbf{work in progress}, and many things
described here (this preliminary draft D1.3~\textsuperscript{v1} of a
future report D1.3~\textsuperscript{v2} scheduled for M30 - in 2020)
are not completely implemented in 2018 or could drastically change
later.

\bigskip

\textit{Bismon} is (like \textit{GCC MELT} was) a \textbf{domain
  specific language} implementation, targetted to ease static source
code analysis (above the \textit{GCC} compiler), with the following
features:

\begin{itemize}

  \item \index{persistence}{\textbf{persistence}}, somehow
    \textit{orthogonal persistence}. It is needed in particular
    because compiling some software project (analyzed by
    \textit{Bismon}) is a long-lasting procedure involving
    \textit{several} compiling and linking processes and commands. So
    most of the data handled by \textit{Bismon} can be persisted on
    disk, and reloaded at the next run
    (cf. \cite{Dearle-2010-orthopersist,
      Dearle:2009:OrthogonalPR}). However, some data is temporary by
    nature~\footnote{E.g. data related to a web session, or to a web
      HTTP exchange, or to an ongoing \texttt{gcc} compilation
      process, etc... needs not to be persisted, and would be useless
      when restarting the \textit{Bismon monitor}.} and should not be
    persisted. Such data is called temporary or
    \index{transient}{\textbf{transient}}. But the usual approach is
    to run the \textit{Bismon} program from some initial loaded state
    and have it \index{dump}{\textbf{dump}} its memory state on
    disk~\footnote{Look also into Liam Proven FOSDEM 2018 talk about
      \emph{The circuit less traveled} on
      \bmurl{https://archive.fosdem.org/2018/schedule/event/alternative\_histories/}
      for an interesting point of view regarding persistent systems.}.
    before exiting (and reload that augmented state at the next run),
    and probably more often than that (e.g. twice an hour, or even
    every ten minutes).

  \item \textbf{dynamic typing}, like many scripting languages (such
    as Guile, Python, Lua, etc). Of course the dynamically typed data
    inside the \textit{Bismon monitor} is \index{garbage
      collection}{\textbf{garbage collected}}
    (cf. \cite{Jones:2011:GC-handbook}). The initial GC of the monitor
    is a crude precise garbage collector, but multi-thread compatible (cf. §\ref{subsec:gcvalobj} below);
    it uses a naive stop-the-world mark\&sweep algorithm. That GC
    should be replaced by a better one, such as
    \href{https://www.ravenbrook.com/project/mps/}{Ravenbrook MPS}
    \footnote{\label{fn:initial-gc} Improving that GC is a difficult
      work, and past experience on \emph{GCC MELT} taught us that
      developing and debugging a GC is hard, and is a good
      illustration of
      \href{https://en.wikipedia.org/wiki/Hofstadter's_law}{Hofstadter's
        law} (See \cite{Hofstadter:1979:GEB}). We should consider
      later using MPS from
      \bmurl{https://www.ravenbrook.com/project/mps/} but doing that
      could require recoding or regenerating \emph{a lot} of code,
      since MPS has specific calling conventions and invariants,
      including
      \href{https://en.wikipedia.org/wiki/A-normal_form}{A-normal
        form}. So, switching to MPS in \emph{Bismon} would require at
      least several months of work.}, if good performance and
    scalability is wanted in \emph{Bismon}.

  \item \textbf{homoiconicity} and \textbf{reflection} with
    \textbf{introspection} (cf \cite{Pitrat:1996:FGCS,
      Pitrat:1990:Metaconnaissances, Pitrat:2009:AST,
      Pitrat:2009:ArtifBeings}): all the DSL code is explicitly
    represented as data which can be processed by \textit{Bismon}, and
    the current state is accessible by the DSL.

    \item \textbf{translated} to \emph{C} code; and \textbf{generated}
      \emph{JavaScript} + \emph{HTML} in the \index{browser}{browser}, and generated
      \emph{C++} code of \emph{GCC} plugins \index{plugin}

    \item \textbf{bootstrapped implementation}: \index{bootstrap} (cf. \cite{Pitrat:1996:FGCS, Polito:2014:Bootstrapping-pharo})
      ideally, all of \textit{Bismon} code (including C code dealing
      with data representations, persistent store, etc...) should be
      generated (but that won't happen entirely with the
      \textsc{Chariot} timeframe). However, this ideal has not yet be
      attained, and there is still a significant amount of
      hand-written C code. It is hoped that most of the hand-written C
      code will eventually become replaced by generated C code.
      
    \item ability to \textbf{generate GCC plugins}: the C++ code of
      GCC plugins \index{plugin} performing static analysis of a single translation
      unit should be generated (this was also done in GCC MELT, see \cite{Starynkevitch-DSL2011}).

    \item with \textbf{collaborative web interface}, \index{web interface} used
      by a \emph{small} team of \emph{trusted and well-behaving}
      developers~\footnote{The initial \texttt{bismon} implementation
        had a hand-coded crude GTK interface, nearly unusable. That
        interface is temporarily used to fill the persistent store
        till the web interface (generated by \emph{Bismon}) is
        usable. The GTK interface is already obsolete and should
        disappear at end of 2018. The Web interface (work in
        progress!) is mostly generated - all the HTML and JavaScript
        code is generated (or taken from outside existing projects
        e.g. \href{http://jquery.com/}{\emph{JQuery}} or
        \href{http://codemirror.net/}{\emph{CodeMirror}}), and their
        HTML and JavaScript generators are made of generated C
        code.}. The users of \emph{bismon} are expected to trust each
      other, and to use the \texttt{bismon} tool
      responsibly\footnote{For example, each \emph{bismon} user has
        the technical ability to erase most of the data inside
        \textit{Bismon monitor}, but is expected to not do so. There
        is no mechanism to forbid or warn him doing such bad things.}
      (likewise, developers accessing a \texttt{git} version control
      repository are supposed to act responsibly even if they are
      technically capable of removing most of the source code and its
      history stored in that repository). So protection against
      malicious behavior of \texttt{bismon} users is out of scope.

      Since \textit{Bismon} should be usable by a small team of
      developers (perhaps two or a dozen of them)\footnote{So
        \textit{Bismon}, considered as a Web application, would have
        at most a dozen of \index{browser}{browsers} -and associated
        users- accessing it. Hence, scalability to many HTTP
        connections is not at all a concern (in contrast with most
        usual web applications).}, it is handling some \index{personal
        data}{personal data} (relevant to \index{GDPR}{GDPR}), such as
      the first and last names (or pseudos) of users and their email
      and maintain a password file (used in the Web
      \index{login}{login} form). Compliance to regulations
      (e.g. European GDPR) is out of scope and should be in charge of
      the entities and/or persons using and/or deploying
      \textit{Bismon}. The login form template \footnote{on
        \bmurl{https://github.com/bstarynk/bismon/blob/master/login\_ONIONBM.thtml}}
      could and probably should be adapted on each deployment site (by
      giving there site-specific contacts for the data controller,
      etc...).

      \item \textbf{multi-threaded} - as many ``server'' like
        programs, \emph{Bismon} should be multi-threaded and take
        advantage of current multi-core processors.

      \item with a \textbf{syntax oriented editor} (for our DSLs),
        inspired by the ideas of \textsc{Mentor} in
        \cite{donzeaugouge:inria-mentor}. So the static analysis
        expert is not typing some raw text (in some concrete syntax of
        our DSL) later handled by traditional parsing techniques (as
        in \cite{Aho:2006:DragonBook}) but should interact using a web
        interface to \emph{modify} and \emph{enhance} the persistent
        store (like old Lisp machines or Smalltalk machines did in the
        1980s), partially shown in a web browser (see also
        §\ref{subsec:webinterf} below). That web interface is
        facilitating \emph{refactoring} of DSL code.
\end{itemize}


\subsubsection{About \emph{Bismon} as a domain-specific language}
\label{subsubsec:bismon-dsl}

Notice that \textit{Bismon} is not even thought as a \emph{textual}
domain specific language~\footnote{In contrast, \emph{GCC MELT} was
  textual and had \texttt{*.melt} source \emph{files} edited with
  \texttt{emacs} using its Lisp mode. This made refactoring difficult,
  since automatic move of textual fragments was not realistically
  possible.} (and this is possible because it is persistent). There is
not (and there won't be) any canonical \emph{textual} syntax for
``source code'' of the domain specific language in \textit{Bismon}
~\footnote{This idea is not new: neither Smalltalk
  (cf. \cite{Goldberg:1983:Smalltalk}), nor Common Lisp
  (cf. \cite{Steele:1990:CommonLisp}), are defined as having a textual
  syntax with an associated operational semantics based on it. Even
  the syntax of C is defined \emph{after} preprocessing. What is
  -perhaps informally- defined for Smalltalk and Common Lisp is some
  abstract internal representation of ``source code'' in ``memory''
  and its ``behavior''. In contrast, Scheme has both its textual
  syntax and its semantics well defined in R5RS, see
  \cite{Adams:1998:R5RS}.}. \index{source code}{\emph{Source code}} is
defined (socially) as the preferred form on which developers are
working on programs. For C or C++ or Scheme programs, source code
indeed sits in textual files in practice (even if the C standard don't
speak of files, only of ``translation units'', see \cite{C11:std}),
and the developer can use any source code editor (perhaps an editor
unaware of the syntax of C, of C++, of Scheme) to change these source
files. In contrast, a developer contributing to \textit{Bismon} is
browsing and changing some internal representations thru the
\textit{Bismon} user interface (a Web interface~\footnote{in mid-2018,
  that Web interface was incomplete, and I still had to temporarily
  use some obsolete GTK-based interface that even I find so disgusting
  that I won't explain it, and sometimes even to edit manually some
  \texttt{store2.bmon} data file, cf §~\ref{subsubsec:filestate}.}; see also \cite{Myers:2000:PPF} for a survey)
and interacts with \textit{Bismon} to do that. There is no really any
\index{abstract syntax tree}{abstract syntax \emph{tree}} in
\textit{Bismon}: what the developer is working on is some \emph{graph}
(with circularities), and the entire persistent state of \emph{Bismon}
could be viewed as some large graph in memory.

Conceptually the initial \emph{Bismon} DSL is at first a
dynamically-typed language, semantically similar to Scheme, Python (or
to a lesser degree, to JavaScript: however, it has classes, not
prototypes, with single-inheritance). It is \emph{essentially} (unlike
most Scheme or Python or JavaScript implementations) a
\emph{multi-threaded} language, since the emitted routines can run in
parallel in our agenda machinery (cf. §\ref{subsec:multithreadist}
below). Meta-programming techniques (inspired by Lisp macro systems,
see \cite{Queinnec:1996:LSP}, and largely experimented in \emph{GCC
  MELT}) will ease the extension of that language.

The base \emph{Bismon} DSL is currently implemented~\footnote{See
  notably our hand-written files \texttt{gencode\_BM.c} and
  \texttt{emitcode\_BM.c} in october 2018. Our bootstrap philosophy
  might require replacing later these hand-written files by better,
  \emph{bismon} generated, modules.} as a naive transpiler to C code
(respecting the coding rules of our implementation, in particular of
our garbage collector, see §\ref{subsec:gcvalobj}).

Our initial DSL is designed in terms of code representations as
objects (see §\ref{subsubsec:objects} below) and immutable values (see
§\ref{subsubsec:immutvalues} below). It is not defined by some EBNF
textual syntax. For example, an assign statement $\alpha$ \texttt{=} $\beta$
is represented by an object of class \texttt{basiclo\_assign}
    \index{basiclo assign@\texttt{basiclo\_assign}} with its
first component representing the left hand-side $\alpha$ and the
second component representing the right hand-side $\beta$. Expressions
in our DSL are either objects, or nodes, or scalars (integers,
strings, etc...).


What is transpiled to C are Bismon ``modules'' (for example our
\texttt{webjs\_module} contains code related to emission of
JavaScript), each with a sequence of routines. A module can be
dumpable (into the persistent state, which then contains also the
generated C code) or temporary (then the generated C code is not kept
in that state). A routine or function~\footnote{Technically the
  routine would be in the module's shared object binary; the function
  is a \emph{Bismon} object reifying the code of that routine.}  is an
object of class \texttt{basiclo\_function}
\index{basiclo function@\texttt{basiclo\_function}}
\index{basiclo minifunction@\texttt{basiclo\_minifunction}}
or its subclass
\texttt{basiclo\_minifunction}, etc... A function knows its arguments,
local variables, local numbers, body by various attributes
(e.g. \texttt{arguments}, \texttt{locals}, \texttt{numbers},
\texttt{body} etc...). Its body is a block made of statements.

Statements of our DSL include:

\begin{itemize}
\item assignments, of class \texttt{basiclo\_assign}, as explained above.
  
  \item run statements, of class \texttt{basiclo\_run}, \index{basiclo
    run@\texttt{basiclo\_run}} which ``evaluates'' its single operand
    for side effects (similar to expression statements in C or Go). As
    a special (and rather common) case, that operand can be a ``code
    chunk'' \index{code chunk} (conceptually similar to GCC MELT's
    code chunks, see \cite{Starynkevitch-DSL2011} §3.4.1), that is a
    node of connective \texttt{chunk} providing a ``template'' for
    expansion as C code.

  \item conditional statements, of class \texttt{basiclo\_cond},
    \index{basiclo cond@\texttt{basiclo\_cond}} 
    \index{basiclo when@\texttt{basiclo\_when}} 
    \index{nb conds@\texttt{nb\_conds}} 
      inspired by Lisp's \texttt{cond}. Its components are a sequence
      of when clauses (which are objects of class
      \texttt{basiclo\_when}) followed the ``else'' statements or
      blocks. The \texttt{nb\_conds} attribute in the statement gives
      the number of when clauses.

    \item for loops, we have a \texttt{basiclo\_while} class of
      statements (for ``while'' loops) and a \texttt{basiclo\_loop}
    \index{basiclo while@\texttt{basiclo\_while}} 
    \index{basiclo loop@\texttt{basiclo\_loop}} 
    \index{basiclo exit@\texttt{basiclo\_exit}}
    \index{basiclo return@\texttt{basiclo\_return}} 
      class (for infinite loops). Exiting of loops and blocks are
      using the \texttt{basiclo\_exit} class. Return statements use
      the \texttt{basiclo\_return} class.

      \item failure (inspired by Go's \texttt{panic}) statements are
    \index{basiclo fail@\texttt{basiclo\_fail}} 
        of class \texttt{basiclo\_fail}. Failures are not exceptions,
        but prematurely terminate the tasklet (of the agenda, see
        §\ref{subsec:multithreadist} below) running the function
        containing that statement.

      \item locking of objects use a \texttt{basiclo\_lockobj} class
    \index{basiclo lockobj@\texttt{basiclo\_lockobj}} 
        (mentioning both an object to lock and a sequence of
        sub-statements or blocks). A locked object is unlocked when
        the end of its locking statement is reached, or when the
        currently active routine terminates (on failure or on return).

      \item execution of primitive side-effecting operations with no
    \index{basiclo cexpansion@\texttt{basiclo\_cexpansion}} 
        result happens in C-expansion statements (of class
          \texttt{basiclo\_cexpansion}), inspired by GCC MELT's
          primitives (see \cite{Starynkevitch-DSL2011}) returning
          \texttt{:void}.

      \item etc...
\end{itemize}

Expressions in our DSL are typed (with types like \texttt{value},
\texttt{object}, \texttt{int}, \texttt{string}, etc...)  and include:

\begin{itemize}
\item scalar (integers, constant strings)
  
\item local variables, or arguments, or numerical variables (of the current function).

\item constant objects (mentioned with \texttt{constants} in the function)

\item closure application (represented by a node of connective \texttt{apply}). Often, the closure's connective would be a
  function object.

  \item quotations (like in Lisp, represented by an unary node of
    connective \texttt{exclam})

\item message sending (represented by a node of connective \texttt{send})

\item primitives (inspired by GCC MELT's ones, see
  \cite{Starynkevitch-DSL2011}; the connective of the node is of class
  \texttt{basiclo\_primitive})

  \item builtin objects like \texttt{current\_closure},
    \texttt{current\_closure\_size}, \texttt{current\_module},
    \texttt{current\_routine} are expanded in some ad-hoc
    \index{current closure@\texttt{current\_closure}}
   \index{current closure size@\texttt{current\_closure\_size}}
   \index{current module@\texttt{current\_module}}
   \index{current routine@\texttt{current\_routine}}
    fashion~\footnote{That is: \texttt{current\_closure} $\rightarrow$
      the current closure; \texttt{current\_closure\_size}
      $\rightarrow$ its size; \texttt{current\_module} $\rightarrow$
      the current module; \texttt{current\_routine} $\rightarrow$ the
      current routine respectively.}.

\item etc...
  
\end{itemize}
  

\subsubsection{About \emph{Bismon} as a evolving software system}
\label{subsubsec:bismon-evolving}

So \emph{Bismon} is better thought of as an evolving software
system. We recommend to try it. Notice that \textit{Bismon} is
\textbf{provided as free software} (available
on \bmurl{https://github.com/bstarynk/bismon/} but unreleased in 2018) in \emph{source form only}
and should be \textbf{usable} \emph{only} \textbf{on a Linux/x86-64
  workstation}... (typically, at least 32 gigabytes of RAM and
preferably more, at least 8 cores, several hundreds gigabytes of disk
or SSD).

The \textit{Bismon} system contains \textbf{persistent data} (which is
part of the system itself and should not be considered as ``external''
data; each team using \textit{Bismon} would run its own customized
version of their \textit{Bismon monitor}.), and should be
\textbf{regularily backed up}, and preferably version controlled at
the user site. It is strongly recommended to use
\index{git}{\texttt{git}} \footnote{See \bmurl{http://git-scm.com/}}
or perhaps some other distributed \index{version control}{version
  control} system, to \texttt{git commit} its files several times a
day (probably hourly or even more frequently, as often as a developer
is committing his C++ code), and to backup the files on some external
media or server at least daily. How that is done is outside of the
scope of this document. The \emph{dump facilities} inside
\textit{Bismon} are expected to be used quite often (as often as you
would save your report in a word processor, or your source file in a
source code editor), probably several times per hour. So a developer
team using \textit{Bismon} would probably \texttt{git clone} either
\texttt{git@github.com:bstarynk/bismon.git} thru SSH or
\bmurl{https://github.com/bstarynk/bismon.git}, build it (after
downloading and building required dependencies), and work on that
\texttt{git} repository (and of course back-up it quite often).

We are still growing \emph{Bismon} by feeding it with additional
interactions changing its persistent state. At first, we developed (at
begin of bootstrap\index{bootstrap}) a crude GTK\index{GTK} interface,
shown in figure \ref{fig:bismonscreenshot-cbdcf}, which is a
screenshot made on October 22\textsuperscript{nd} 2018 on git commit
\texttt{cbdcf1ec351c3f2a}, when working on the JavaScript generator
inside \emph{Bismon}. It shows several windows: the large top right
window (named \texttt{new-bismon}) has a command textview (ivory
background, top panel) and a command output (azure background, bottom
panel). The small top left window (named \texttt{bismon values} show
the read-eval-print-loop output (as \texttt{\$a} in two panes). The
mid-sized bottom left window (titled \texttt{bismonob\#1}) shows (in
two text-views of the same GTK text buffer) shows (in top text-view) a
large part of the body of the \texttt{emit\_jsstmt} method for the
\texttt{basiclo\_while} class of \emph{Bismon} and (in bottom
text-view) some components of our \texttt{webjs\_module} object. In
the rear, bottom right, a tiny part of our \texttt{emacs} editor (used
to run \texttt{bismon --gui}...) is visible, and shows a
backtrace~\footnote{Ian L. Taylor's
  \href{https://github.com/ianlancetaylor/libbacktrace}{\texttt{libbacktrace}}
  is used in \emph{bismon} to provide symbolic backtraces.}.


\begin{figure}[h]
  \begin{center}
    \bmincludewidthgraphics{0.98\textwidth}{bismon-cbdcf1ec351c3f2a-screenshot-22oct2018-img}{png}{png}
  \end{center}
  \caption{crude {\relsize{-1}{(soon deprecated)}} GTK interface
    {\relsize{-1}{oct. 22, 2018, git commit
        \texttt{cbdcf1ec351c3f2a}}}}
  \label{fig:bismonscreenshot-cbdcf}
\end{figure}

This crude GTK3 interface~\footnote{It is implemented in 12.5KLOC of C
  code in \texttt{gui\_GTKBM.c}, \texttt{newgui\_GTKBM.c} and
  \texttt{guicode\_BM.c}. We won't debug that code -which crashes
  often- but will remove it once it can be replaced by a web
  interface.} is buggy and not satisfactory. It will and needs to be
replaced by a Web interface. Several lessons have been gained with
this experience:

\begin{itemize}
\item using GTK~\footnote{Since GTK is a free software library, we
  could consider patching its source code, but such a huge effort is
  not reasonable within the timeframe of \textsc{Chariot}, and GTK is
  still evolving a lot, so patching it would require freezing its
  version. \texttt{gtk+-3.24.1} has 1.2 millions lines of source code,
  measured by D.Wheeler \texttt{sloccount} but it depends also on
  other libraries, such as Pango, Glib, etc... so patching GTK source
  for our precise GC is not reasonable at all.}  is in practice
  incompatible with a multi-threaded precise garbage
  collector~\footnote{See also
    \bmurl{https://stackoverflow.com/q/43141659/841108} about GTK and
    Boehm's conservative GC.}, like the one in \emph{Bismon} (cf
  §\ref{subsec:gcvalobj} below), in particular because GTK may have
  several nested event loops, so many local garbage collector pointers
  in internal call frames (which are not accessible from routines
  above).

    \item the model and the C API provided by GTK text views and text
      buffers is not adequate for structured syntactic editing (like
      pionneered in Mentor, see \cite{donzeaugouge:inria-mentor}). It is still too low-level and oriented for plain textual edition.

      \item GTK is not compatible with several X11 displays, so a
        single \emph{bismon} process cannot handle several users each
        having its own screen.
\end{itemize}

So we decided to stop investing efforts on the GTK interface, and give
more priority to a Web interface, which is required once a small team
of \emph{several} IoT developers need to interact with the
\emph{bismon} persistent monitor. The GTK interface is just
temporarily needed to fill the persistent store (till our web
interface is usable). At end of 2018 (or first quarter of 2019) it
will be entirely scrapped, and the static analysis expert (and other
users pof \emph{Bismon}) will interact with \emph{bismon} thru some
Web interface.

Work on the future Web interface has significantly progressed
~\footnote{With the hand-written \texttt{web\_ONIONBM.c} using
  \texttt{libonion}, and the \emph{bismon} module
  \texttt{webjs\_module} translated into the
  \texttt{modules/modbm\_1zCsXG4OTPr\_8PwkDAWr16S.c} emitted C file of
  more than 6.5KLOC in october 2018.}. New users -called
\emph{contributors} \index{contributor}- can be voluntarily registered
and unregistered on the command line~\footnote{Use the option
  \texttt{--contributor=} to add them, \texttt{--remove-contributor=}
  to remove them and \texttt{--add-passwords=} to set their encrypted
  \emph{bismon}-specific password.} into \emph{bismon} in a way
similar to \texttt{git}. When they access any dynamic web page, a
login web form appears (with some GDPR \index{GDPR} related notice) if
no web cookie identifies them. But that Web interface is still
incomplete in October 2018. Several design decisions have been made:
we will use the \texttt{codemirror}~\footnote{See
  \bmurl{http://codemirror.net/} for more.} web framework to show the
analyzed source code of IoT software. The web interface for IoT
developers should be a ``single-page application'' one (so AJAX,
HTML5, CSS3 techniques, with generated JavaScript and HTML
code). \emph{WebSockets} should be used for asynchronous communication
between browser and the \emph{bismon} monitor. The
\href{http://jquery.com/}{\texttt{jquery}},
\href{https://angular.io/}{\texttt{angular}},
\href{https://semantic-ui.com/}{\texttt{semantic-ui}}, etc... web
frameworks are considered as building blocks for that Web interface
and should be installed with inside \emph{bismon}~\footnote{For
  example, the \emph{bismon} source tree has a
  \texttt{webroot/jscript/jquery.js} local file to serve HTTP
  \texttt{GET} requests to an URL like
  \texttt{http://localhost:8086/jscript/jquery.js} handled by the
  \emph{bismon} monitor.} to enable using \emph{bismon} without any
external Internet connection.


IoT developers working with the \emph{Bismon monitor} will use some
Web interface to interact with it.

\subsubsection{About \emph{Bismon} as a static source code analyzer framework}

The \emph{Bismon} persistent monitor will generate the C++ code of GCC
plugins, leveraging on the experience of GCC MELT (see
\cite{Starynkevitch-DSL2011,Starynkevitch2007Multistage,Starynkevitch-GCCMELTweb}). The
C++ code generator will have a design similar to (and share some code
and classes with) our internal initial DSL (cf
§\ref{subsubsec:bismon-dsl}). It is extremely likely that in many
cases, such a generated GCC plugin would just insert its appropriate
passes by using the pass manager (cf \cite{gcc-internals} §9 and
§24.3), and these passes will ``serialize'' internal representations
(either in JSON, or using Google protocol buffer, or using a textual
format close to our dump syntax, see figure \ref{fig:dumpvalsyntax}
below, etc...)  such as \emph{Gimple}-s, \emph{Basic Block}-s and
transmit some form of them to the \emph{Bismon} persistent monitor. In
some simple cases, it is not even necessary to transmit most of that
representation. For instance, a whole program static analysis to help
avoiding stack overflow needs just the size of each call
frame~\footnote{Notice that \emph{GCC} compute these call frame sizes
  (see the \texttt{-fstack-usage} option), and can detect excessively
  big call frame with \texttt{-Wstack-usage=} option.} and the control
flow graph (so only the \emph{Gimple} call statements, ignoring
anything else); with that information (and the control flow graph)
the monitor should be able to estimate an approximation~\footnote{Of
  course dynamic calls, e.g. call thru function pointers, make that
  much more complex and will require manual annotation.} of the
consumed call stack, whole program wide.

Several design decisions have been made regarding the style of the
generated C++ code of GCC plugins: it will use existing scalar data
and \texttt{GTY}-ed classes (see \cite{gcc-internals} §23), to take
advantage of the existing GCC garbage collector
(\emph{Ggc}). Contrarily to GCC MELT, it won't provide a generational
garbage collector (because most of the processing happens in the
monitor, hence performance is less important), so transforming to
A-normal form is not required at translation (to C++) time.

% @@@ to be completed when some static source code analysis of some C or
% C++ code is possible with \emph{Bismon}, using \emph{generated} GCC
% plugins (emitted as C++ code by \emph{Bismon}, and used inside the
% cross-compilation processed started by IoT developers following the
% \textsc{Chariot} approach.


%%%%%%%%%%%%%%%%%%%%%%%%%%%%%%%%%%%%%%%%%%%%%%%%%%%%%%%%%%%%%%%%
\subsection{Multi-threaded and distributed aspects of \textit{Bismon}}
\label{subsec:multithreadist}

The \textit{Bismon monitor} is by itself a multi-threaded
process~\footnote{In contrast of most scripting languages
  implementations such as Python, Ocaml, Ruby, etc..., we try hard to
  avoid any ``global interpreter lock'' and strive to develop a
  genuinely multi-threaded monitor.}.
It uses a \emph{fixed} {\index{thread pool}{\emph{thread pool}}} of
{\index{worker thread}{\emph{worker threads}}} (often active)
\footnote{The number of worker threads is given by the \texttt{--job}
  program argument to \texttt{bismon}. For an 8-cores workstation, it
  is suggested to set it to 5 or 6. It should be at least 2, and at
  most 15. This number of jobs also limits the set of simultaneously
  running external processes, such as \texttt{gcc} processes started
  by \emph{Bismon}.}, and additional (generally idle) threads for web
support and other facilities. The \textit{Bismon monitor} is
occasionally starting some external processes, in particular for the
compilation of generated \emph{GCC} plugins, and for the compilation
into \index{module}{\emph{module}s} -technically ``plugins''- of
dynamically generated \emph{C} code by \textit{Bismon}; later it will
dynamically load (with \texttt{dlopen}) these modules, and thus
\textit{Bismon} can increase its code (but cannot decrease it, even if
some code becomes unused and unreachable); however such modules are
\emph{never} \index{garbage collection}{garbage collected} (so
\texttt{dlclose} is never called). So in practice, it is recommended
to restart \emph{Bismon} every day (to avoid endless growth of its
code segments).


The worker threads of \emph{Bismon} are implementing its
\index{agenda}{\textbf{agenda}} \footnote{Details about the agenda,
  such as the fixed set of available priorities, are subject to
  change. We describe here the current implementation in mid-2018.}
machinery. Conceptually, the agenda is a 5-tuple of first-in first-out
queues of \index{tasklet}{\textbf{tasklets}}, each such FIFO queue is
corresponding to one of the five \index{priority}{priorities} :
\emph{very high}, \emph{high}, \emph{normal}, \emph{low}, \emph{very
  low}. Each agenda worker thread removes one tasklet (choosing the
queue of highest possible priority which is non empty, and picking the
tasklet in front of that queue) and runs that tasklet quickly. A
tasklet should run during a few milliseconds (e.g. with some implicit
kind of non-preemptive scheduling) at most (so cannot do any blocking
IO; so input and output happens outside of the agenda). It may add one
or more tasklets (including itself) to the agenda (either at the
front, or at the end, of a queue of given priority), and it may remove
existing tasklets from the agenda. Of course tasklets run in parallel
since there are several worker threads to run the agenda. The agenda
itself is not persisted as a whole, but tasklets\footnote{Actually
  tasklets are objects (see §\ref{subsubsec:objects} page
  \pageref{subsubsec:objects} below), and to run them, the agenda is
  sending them a message with the predefined selector
  \texttt{\emph{run\_tasklet}}.}  themselves may be
\index{persistent}{persistent} or \index{transient}{transient}.
Tasklets can also be created outside of the agenda (e.g. by incoming
HTTP requests, by completion of external processes, by timers, ...)
and added asynchronously into the agenda.

Outside of the agenda, there is an \emph{idle queue} of delayed todo
closures (a queue of closures to be run, as if it was an idle priority
queue) with some arguments to apply to them. But that \index{idle
  queue}{idle queue} don't contain directly any tasklets. That idle
queue can be filled by external events~\footnote{For example, when an
  external compilation process completes, that queue is filled with
  some closure -provided when starting that compilation- and, as
  arguments, an object with a string buffer containing the output of
  that process, and the integer status of that process.}. Of course the idle
queue is not persisted.

\bigskip

In its final version, the \emph{Bismon system} will involve several
cooperating Linux processes:

\begin{itemize}
\item the \emph{Bismon monitor} itself, with several threads (notably for the agenda mechanism described above)
  
  \item the web \index{browser}{browsers} of developers using that
    particular \emph{Bismon monitor}; each developer probably runs
    his/her own browser. That web browser is expected to follow latest
    Web technologies and standards (HTML5, Javascript6 i.e. EcmaScript
    2016 at least, WebSockets, ...). It should probably be a Firefox
    or a Chrome browser from 2017 or after. The HTML and Javascript is
    dynamically generated by the \emph{Bismon monitor} and should
    provide (to the developer using \emph{Bismon}) some ``single-page
    application'' (cf. \cite{Atkinson:2018:webtabs,
      Queinnec:2004:ContinWeb, Graunke:2003:ModelingWeb})
    feeling~\footnote{So using your browser's backward and forward
      navigation arrows won't work well because in single-page
      applications they \emph{cannot} work reliably}.

    \item the IoT developers using \emph{Bismon} will build their IoT
      firmware as usual; however they will add some extra options (to
      their \texttt{gcc} or \texttt{g++} cross-compilation commands)
      to use some \emph{Bismon} generated GCC plugin in their
      cross-compilation processes. So these cross-compilation
      processes (i.e. \texttt{cc1} started from \texttt{gcc}, or
      \texttt{cc1plus} started from some \texttt{g++}, etc...),
      augmented by generated plugins, are involved.

  \item sometimes \emph{Bismon} would generate some
    \texttt{modules/*.c} file during execution, and fork a (direct)
    compilation of it (technically forking a
    \texttt{./build-bismon-persistent-module.sh} -for persistent
    modules- or a \texttt{./build-bismon-temporary-module.sh} -for
    temporary modules- shell script, which invokes \texttt{make} which
    runs some \texttt{gcc} command) into a ``plugin'' module
    \texttt{modubin/*.so}, which would be \texttt{dlopen}-ed.

  \item \emph{Bismon} should also generate the C++ code of \emph{GCC
    plugins}, to be later compiled then used (with \texttt{gcc} or
    \texttt{g++} option \texttt{-fplugin}). Two kinds of \emph{GCC}
    plugins are considered to be generated:

    \begin{enumerate}
      \item usually, the GCC plugin~\footnote{It is tempting to call
        such plugins \emph{cross-}plugins, since they would be
        \texttt{dlopen}-ed by a cross-compiler.} would be generated to
        assist [cross-] compilation (e.g. of IoT software) by
        developers using \emph{Bismon}. So for an IoT developer
        targeting some RaspberryPi, it could be a GCC plugin
        targeting the \texttt{arm-linux-gnueabi-gcc-8} cross-compiler
        (but the C++ code of that plugin needs to be compiled by the
        native \texttt{gcc} on the host system).
        
      \item But the GCC API is so complex (and under-documented) that
        it is worth extracting it automatically by sometimes
        generating a GCC plugin~\footnote{It is tempting to call such
          plugins \emph{straight-}plugins, since they would be
          \texttt{dlopen}-ed by a straight compiler, not a
          cross-compiler.} to inspect the public headers of
        GCC\footnote{In \emph{GCC MELT}, we tried to describe by
          hand-coded \emph{MELT} code a small part of that GCC API and
          its glue for \emph{MELT}. This approach is exhausting, and
          makes following the evolution of GCC very difficult and
          time-consuming, since new \emph{MELT} code should be written
          or manually adapted at each release of \emph{GCC}. Some
          partial automation is needed to ease that effort of adapting
          to successive \emph{GCC} versions and their non-compatible
          plugins API}. Even when the end-user developer is targetting
        a small IoT chip requiring a cross-compiler (like
        \texttt{arm-linux-gnueabi-gcc-8} above), these GCC inspecting
        plugins are for the native \texttt{gcc} (both
        \cite{Schafmeister:2016:CANDO} and
        \cite{Schafmeister:2015:CLASP} are inspirational for such an
        approach).
        
    \end{enumerate}

    We are considering several ways of providing (to the IoT developer
    using them) such generated C++ code for GCC plugins. We might
    generate (at least for the first common case of GCC plugins
    generated for developers using \emph{Bismon}, and large enough to
    need several~\footnote{By past experience in GCC MELT, we did
      generate C++ files totalizing almost a million lines of C++
      code, and compiling such a large generated C++ code base took
      dozens of minutes, and created a bottleneck.} generated C++
    files) \texttt{*.shar} archives (obtained by Web requests, or
    perhaps some \texttt{wget} or \texttt{curl} command in some
    \texttt{Makefile}) for \emph{GNU sharutils}~\footnote{See
      \bmurl{https://www.gnu.org/software/sharutils/} for more.}
    containing the C++ code and also the \texttt{g++} command
    compiling it. That archive could instead be just a
    \texttt{.tar.gz} file (and the IoT developer would extract it, and
    run \texttt{make} or \texttt{ninja} inside the extracted directory
    to build the shared object GCC plugin binary file), etc... For a
    \emph{small} generated GCC plugin fitting in a single generated
    C++ file of less than a dozen thousands lines, we could simply
    serve in \emph{Bismon} an URL like
    \texttt{http://localhost:8086/genplugin23.c} and require the IoT
    developer to fetch then use that. Other approaches could also be
    considered. The rare second case (GCC plugin code generated to
    inspect the GCC API, running on the same machine as the
    \emph{Bismon} server) could be handled thru external processes
    (similar to compilation of \emph{Bismon} modules). Alternatively,
    we might consider delegating such plugin-enhanced
    cross-compilation processes to the \emph{Bismon} monitor itself,
    etc, etc...
    
\end{itemize}

In principle, the various facets of \emph{Bismon} can run on different
machines as \index{distributed computing}{distributed computing}
(obviously the web browser is not required to run on the same machine
as the \emph{Bismon monitor}, but even the various compilations -of
code generated by \emph{Bismon}, and the cross-compilation of IoT
code- could happen on other machines).

Conceptually, we aim for a \index{multi-tier
  programming}{\textbf{multi-tier programming}} approach (inspired by
Ocsigen~\footnote{See \bmurl{https://ocsigen.org/}} with the high-level
DSL inside \emph{Bismon} generating code: in the \emph{Bismon
  monitor}, as modules; in the \emph{web browser}, as generated
Javascript and HTML; in the \emph{GCC} compiler, as generated GCC plugins.


%%%%%%%%%%%%%%%%%%%%%%%%%%%%%%%%%%%%%%%%%%%%%%%%%%%%%%%%%%%%%%%%
%% Local Variables: ;;
%% compile-command: "cd ..; ./build-bismon-doc.sh" ;;
%% End: ;;
%%%%%%%%%%%%%%%%%%%%%%%%%%%%%%%%%%%%%%%%%%%%%%%%%%%%%%%%%%%%%%%%


\newpage

%% file datapersist-bm.tex, which is \input from bismon-doc.tex
\section{Data and its persistence in \emph{Bismon}}


%%%%%%%%%%%%%%%%%%%%%%%%%%%%%%%%%%%%%%%%%%%%%%%%%%%%%%%%%%%%%%%%
\subsection{data processed in \emph{Bismon}}

The \emph{Bismon monitor} handles various kinds of
\index{data}{data}. A lot of data is \index{immutable}{immutable} (its
content cannot change once the data has been created, for example
strings). But \index{object}{\textbf{objects}} are of course mutable
and can be modified after creation. Since \emph{Bismon} is
multi-threaded and its agenda is running \emph{several} worker threads
in parallel, these mutable objects contain a \index{mutex}{mutex} for
locking purposes.

So the \emph{Bismon monitor} handle
\index{value}{\textbf{values}}~\footnote{To extend \emph{Bismon} to
  handle some new kind of custom data (such as bignums, images, neural
  networks, etc...) processed by external libraries, it is advised to
  define new \emph{payloads} inside objects
  (cf. §\ref{subsubsec:objects} below), without adding some new kind
  of values.} (represented as pointers) : they can be immutable, or objects (and objects are the
only kind of mutable data).

%%%%%%%%%%%%%%%%
\subsubsection{Immutable values}
\label{subsubsec:immutvalues}

They include

\begin{itemize}

\item \textbf{tagged integer} (of 63 bits). The
  \index{integer}{integer} won't change (and integer values don't
  require extra space to keep that integer, since they are encoded in
  the pointer).
  
\item UTF-8 encoded \textbf{string} (the bytes inside such
  \index{string}{strings} don't change).

  \item \textbf{tuple} of objects, that is an ordered (but immutable)
    sequence of object pointers (the size or content of a tuple don't
    change). A given object could appear in several positions in a
    \index{tuple}{tuple}.

  \item \textbf{set} of objects, represented internally as a sorted
    array of objects' [i.e. pointers]. A given object can occur only once in a
    set, and membership (of an object inside a set) is tested
    dichotomically in logarithmic time. Of course, the size and
    content of a \index{set}{set} never change.

  \item \textbf{node}. A \index{node}{node} has an object connective,
    and a (possibly empty, but fixed) sequence of \index{son}{sons}
    (sons are themselves values, so can themselves be integers,
    strings, tuples, sets, sub-nodes). The connective, size and sons
    of a node don't change with time. Since a node is immutable and
    knows all its sons at creation time, circularity (e.g. a node
    having itself as some grand-son) inside it is impossible, and the
    node has a finite depth.

  \item \textbf{closure}. A \index{closure}{closure} is like a node
    (it has a connective and sons), but its connective is interpreted
    as the object giving the \index{routine}{routine} (see
    §\ref{subsubsec:objects} below) to be called when that closure is
    applied, and its sons are considered as closed values.
        
\end{itemize}

The \index{nil}{\textbf{nil}} value is generally not considered as a
value, but as the absence of some value. We might (later) add other
kind of values (perhaps vectors of 64 bits integers, of doubles,
bignums ...), but they should all be immutable. However, it is very
likely that we prefer complex or weird data to sit inside objects, as
payload. There is also a single
\index{unspecified}{\textbf{unspecified}} value (which is non-nil so
cannot be confused with lack of value represented by nil).

Tuples and nodes and closures could contain nil, but sets cannot. A
node or closure connective is a genuine object (so cannot be nil),
even if nodes or closures could have nil sons.

Sets and tuples are sometimes both considered as
\index{sequence}{\textbf{sequences}} and share some common operations.

The immutable values are somehow lightweight. Most of them (strings,
sets, tuples, nodes) internally keep some hash-code (e.g. to
accelerate equality tests on them, or accessing hash tables having
values as keys). The memory overhead for values is therefore small (a
few extra words at most, to keep GC-data type and mark, size hash).

The \index{size}{size} of values (byte length of strings, number of objects in
tuples or sets, number of sons in nodes or closures) can in principle reach
$2^{31} - 1$ but is generally much smaller (often less than a few
dozens) and could be 0.

Mutable values outside of objects (and their payload, see
§\ref{subsubsec:objects} below) cannot exist.

Values (even references to objects, e.g. inside sequences or nodes)
are represented as a machine pointer and fit in a 64 bits word. When
its least significant bit is 1, it is a tagged integer.

Values, including objects, are comparable so sortable. For strings,
nodes, closures, sets, tuples we use a lexicographical order. Values
also have an hashcode to be easily put in hash tables, etc..

%%%%%%%%%%%%%%%%
\subsubsection{Mutable objects}
\label{subsubsec:objects}

\index{object}{Objects} are the only kind of \index{mutable}{mutable}
values, and are somehow heavy (at least severy dozens of machine words
in memory for each object). They can be accessed nearly simultaneously
by several worker threads running different tasklets, so they need a
locking mechanism and contain a (recursive) mutex~\footnote{Each
  object has its mutex initialized with
  \href{http://man7.org/linux/man-pages/man3/pthread\_mutex\_init.3p.html}{\texttt{pthread\_mutex\_init}(3p)}
  with the \texttt{PTHREAD\_MUTEX\_RECURSIVE} attribute, and lockable
  with
  \href{http://man7.org/linux/man-pages/man3/pthread\_mutex\_lock.3p.html}{\texttt{pthread\_mutex\_lock}(3p)}
  etc...} (so in reality only one thread is accessing or modifying
them at a given instant).

\medskip

\emph{Conceptually}, objects contain the following data:

\begin{itemize}
  \item a constant \emph{unique} serial id (of about 128 bits), called
    the \index{objid}{\textbf{objid}}, randomly generated at object
    creation time and never changed after. In many occasions, that
    objid is printed as 24 characters {\small (two glued blocks of 12
    characters each, the first being an underscore \texttt{\_}, the
    second being a digit, the 10 others being alphanumerical with
    significant case)} such as
    \texttt{\_4ggW2XwfXdp\_1XRSvOvZqTC}~\footnote{That objid
      \texttt{\_4ggW2XwfXdp\_1XRSvOvZqTC} is for the predefined object
      \emph{\texttt{the\_system}}, and corresponds to the two 64 bits
      numbers 3577488711679049683, 1649751471969277032 i.e. to 128
      bits hexadecimal \texttt{0x31a5cb0767997fd316e5183916681468}.}
    or \texttt{\_0xbmmxnN8E8\_0ZuEqJmqMNH}. It is expected that objid
    \index{collision}{collisions} never occur in practice, e.g. that
    even thousands of \emph{Bismon monitor} processes (running on many
    distant computers) would in fact never generate the same objid. In
    other words, our objids are as unique as \index{UUID}{UUID}s (from
    \href{https://tools.ietf.org/html/rfc4122}{RFC 4122}) are (but are
    displayed differently, without hyphens). The concrete textual
    ``syntax'' for objid-s (starting with an underscore then a digit,
    etc...) is carefully chosen to be compatible and friendly with
    identifiers in C, C++, JavaScript, Ocaml, etc. The \emph{Bismon}
    runtime maintains a large array of hashtables and mutexes to be
    able to quickly find the object pointer of a given objid (if such
    an object exists in memory). The objid is used to compare (and
    sort) objects. The (32 bits, non-zero) hash code of an object is
    obtained from its objid (but it is cached in the object's memory,
    for performance reasons).

    \item the recursive \textbf{mutex lock} of that object~\footnote{We
      have considered using a pthread \texttt{rwlock} instead of a
      \texttt{mutex}, but that would probably be more heavy and
      perhaps slower, but could be experimented in the future.}. So
      locking (or unlocking) an object really means using that lock on
      pthread mutex operations~\footnote{So accessing without the
        protection of that lock being hold, any data inside an object,
        other than its constant objid, its class, its routine pointer,
        is forbidden and considered as \emph{undefined behavior}}.

  \item a \index{space}{\textbf{space}} number fitting in a single
    byte. The space 0 is for \index{transient}{\emph{transient}}
    objects that are not persisted to disk. The space 1 is for
    \index{predefined}{\emph{predefined}} objects (there are about 60
    of them in Q3 of 2018), which are conceptually created before the
    start of \emph{Bismon monitor} processes and are permanently
    available, even at initial load time of the persistent
    store. Those predefined objects are dumped in file
    \texttt{store1.bmon}, the objects of space 2 (conventionally
    called the \emph{global} space) are dumped and persisted in file
    \texttt{store2.bmon}, etc...

    \item the \index{mtime}{\textbf{mtime}} of an object holds its
      modification time, with a millisecond granularity, since the
      Unix Epoch. \index{touch}{Touching} an object is updating its
      \emph{mtime} to the current time.

    \item the (mutable!) \index{class}{\textbf{class}} of an object is
      an \index{atomic}{\emph{atomic}}~\footnote{Here, ``atomic'' is
        understood in the C or C++ memory sense; so a pointer declared
        \texttt{\_Atomic} in C or \texttt{std::atomic} in C++,
        supposing that they are the same and interoperable. Hence the
        class of an object can be obtained \emph{without} locking that
        object.} pointer to an object (usually another one) describing
      its class, as understood by \emph{Bismon}. It is allowed to
      change dynamically~\footnote{Changing classes is permitted
        within reasonable bounds: the class of all classes should
        remain the \emph{\texttt{class}} predefined object; all
        objects should be instances of the predefined
        \emph{\texttt{object}} or more often of some indirect
        sub-class of it; of course these invariants cannot be proved.}
      the class of any object. Classes describe the behavior (i.e. the
      dictionnary of ``methods''), not the content (i.e. the
      ``attributes'') of objects and enable single-inheritance (every
      class has one super-class).

    \item the \index{attribute}{\emph{attributes}} of an object are
      organized as an hash-table associating attribute or key objects
      to arbitrary non-nil values. An \textbf{attribute} is an
      arbitrary object, and its value is arbitrary (but cannot be
      nil).

    \item the \index{component}{\emph{components}} of an object are
      organized as a vector (whose size can change, grow, or shrink)
      of values. A \textbf{component} inside an object is a value
      (possibly nil).

      \item objects may contain one \textbf{routine} pointer (or nil),
        described by \begin{enumerate}
          
        \item the \index{routine address}{\emph{routine address}}
          inside an object is a function or \index{routine}{routine}
          pointer (in the C sense, possibly and often nil). The
          signature of that function is described by the routine
          signature~\footnote{Perhaps all our routines will keep the
            same signature, and then it would not need to be
            explicitly stored.}

        \item the \index{routine signature}{\emph{routine signature}}
          is (when the routine address is non-nil) describing the
          signature of the routine address above.
      \end{enumerate}

      Notice that routine address and signature can only change when
      a new module~\footnote{The generated C code of modules also
        contains an array of constant objids, ana another array of
        routine objids.} is loaded (or at initial persistent state load
      time), and that can happen only when the agenda is
      inactive. Conceptually they are mostly constant (and do not
      require any locking).
        
      Most (in 2018, all) routines have the same C signature
      \texttt{objrout\_sigBM} corresponding to the predefined object
      \texttt{\emph{function\_sig}}. For an object of objid
      {$\Omega$} of that signature its routine address
      corresponds to the C name \texttt{crout}{$\Omega$}\texttt{\_BM}. For
      instance, to initialize (at load time) the object of oid
      \texttt{\_9CG8SKNs6Ql\_4PiHd8cnydn} the initial loader (or the
      module loader) would \texttt{dlsym} the
      \texttt{crout\_09Hug4WGnPK\_7PpZby8pz84\_BM} C function name.


      \item objects may also have some (nearly arbitrary)
        \index{payload}{\textbf{payload}} - which can contain anything
        that don't fit elsewhere. That payload is a pointer (possibly
        nil) to some client data owned by the object; the payload is
        usually not a value but something else. The garbage collector
        should know all the payload types. In 2018 the following
        payloads are possible (with other specialized payloads,
        e.g. for parsing, loading, dumping and web request and web
        session support, contributors):

        \begin{enumerate}
        \item mutable \index{string buffer}{\emph{string buffer}}.
          \item mutable \index{class}{\emph{class} information} (with its super class, and
            the method ``dictionnary'' associating objects -selectors-
            to closures). The class objects are required to have such
            a payload.
        \item mutable \index{vector}{\emph{vector}} of values (like
          for components).
        \item mutable  doubly \index{linked list}{\emph{linked list}} of non-nil values.
        \item mutable \index{associative table}{\emph{associative
            table}} associating objects to non-nil values (like for
          attributes)
        \item mutable \index{hash set}{\emph{hash set}} of objects.
        \item mutable \index{hash map}{\emph{hash maps}} to associate arbitrary
            non-nil values used as keys to other arbitrary non-nil
            values.
            \item mutable \index{dictionnary}{\emph{string dictionnaries}} associating
              non-empty strings to non-nil values.
          \item etc...
        \end{enumerate}

        Of course, the payload of an object should be initialized (so
        created), accessed, used, modified, changed to another
        payload, cleared (so deleted) only while that object is
        locked, and each payload belongs to only one object, its
        owner.
       
\end{itemize}

\medskip

For convenience, (some) objects can also be (optionally) named, in
some top-level \index{dictionnary}{``dictionnary''} or ``symbol
table'' (which actually contain weak references to named objects). But
the \index{name}{name} of an object is not part of it.

\bigskip

\subsection{garbage collection of values and objects}

The \emph{Bismon monitor} has in 2018 a precise, but naive,
mark\&sweep stop-the-world \index{garbage collector}{garbage
  collector} for values \footnote{See also previous footnote
  \ref{fn:initial-gc} on page \pageref{fn:initial-gc} for possible
  improvements of the GC.}, of course including objects. When the GC
is running, the agenda has been de-activated, and no tasklets are
running.

In contrast to most GC implementations (but inspired by the habits of
\emph{GCC} itself -in its \emph{Ggc} garbage collector- in that area),
the garbage collector of the \emph{Bismon monitor} is \textbf{not}
\emph{directly} triggered in allocation routines (but is started by
the agenda machinery). When allocation routines detect that a
significant amount of memory has been consumed, they set some atomic
flag for wanting GC, and later that flag would be tested (regularily)
by the agenda machinery which runs the GC.  So when the GC is actually
running, the call stacks are conceptually empty~\footnote{But the
  support threads, e.g. for web service with \texttt{libonion}, add
  complication to this scheme. However, ignoring conceptually the call
  stacks don't require us to use A-normal forms in module code, as was
  needed in \emph{GCC MELT}, and facilitate thus the generation of C
  code inside them.}, and no tasklet is active.

The garbage collection roots include:

\begin{itemize}
\item all the \index{tasklet}{tasklets} queued in the (several queues) of the agenda
\item all the \index{predefined}{predefined} objects
  \item all the constants (objects~\footnote{The object of objid
    \texttt{\_1FEnnpEkGdI\_5DAcVDL5XHG} should be designed as
    \texttt{BMK\_1FEnnpEkGdI\_5DAcVDL5XHG} in hand-written C code if
    it is not predefined, and a special utility collects all such
    names and generates a table of all these constants.}) referred by
    both hand-written and generated C code (including constants
    referred by \index{module}{modules}, and objects reifying
    modules).
\item some very few global variables (containing values), so
  conceptually the idle queue of closures, and the queue related to
  external running processes, the hash-set of active web request
  objects, etc.
\end{itemize}

The naive \emph{Bismon monitor} garbage collection~\footnote{Some
  previous experimentation with Boehm's GC in multithreaded settings
  has been unsatisfactory.}  works as follow: a queue of non-visited
objects to be scanned is maintained, with an hash-set of marked
objects. Initially, we visit the GC roots above. Visiting a value
involves marking it (recursively for sequences, nodes, closures, ...)
and if it is a newly marked object absent from the hash-set, adding
that object to the scan queue and to the hash set of marked
objects. We repeatedly extract objects to be scanned from the queue
and visit their content (including their attributes, their components,
their signature and payload and the values inside that payload). When
the scan queue is empty, GC is finished.

%%%%%%%%%%%%%%%%%%%%%%%%%%%%%%%%%%%%%%%%%%%%%%%%%%%%%%%%%%%%%%%%
\subsection{persistence in \emph{Bismon}}

\index{persistence}{\emph{Persistence}} is an essential feature of the
\emph{Bismon monitor}. It always starts by loading some previous
persisted \index{state}{state}, and usually dumps its current state
before termination. On the next run, that new state is loaded, etc....
For convenience and portability, \emph{the state is a set of textual
  files}~\footnote{A gross analogy is the textual dump of some SQL
  database. That dump is the only way to reliably recover the
  database, so it should be done frequently and the backed-up
  \texttt{database.sql} textual dump can be a large file of many
  gigabytes.}. The persistent state should be considered as precious
and as valuable as source code of most software, so it should be
backed-up and probably version-controlled~\footnote{How and when the
  persistent state is dumped, backed up and version controlled is out
  of scope of this report. We strongly recommend doing that
  frequently, at least several times every day and probably a few
  times each hour. If the \textit{Bismon monitor} crashes, you have
  lost everything since the latest dumped persistent store. The
  textual format of the persisted state should be friendly to most
  version control systems and other utilities.} at every site using
the \emph{Bismon} monitor. 

%%%%%%%%%%%%%%%%
\subsubsection{file organization of the persistent state}
\label{subsubsec:filestate}

The persistent state contains both data and ``code''. So it is made of the following \index{file}{files}:

\begin{itemize}
\item data files \texttt{store1.bmon}, \texttt{store2.bmon}
  etc... Each such generated data file describes a (potentially large)
  collection of persistent objects, and mentions also the modules
  required for them. There is one data file per space, so
  \texttt{store1.bmon} is for the space\#1 (containing predefined
  objects), \texttt{store2.bmon} is for the space\#2 (conventionally
  containing ``global'' objects useful on every instance of
  \emph{Bismon}), etc...

  \item code files contain the \emph{generated} C code of persistent
    modules. Since each module is (also) reified by an object
    representing (and generating) that module, the code file paths
    contain objids. For example, our object
    \texttt{\_3BFt4NfJmZC\_7iYi2dwM38B} (it is tentatively named
    \texttt{first\_misc\_module}, of class
    \texttt{\emph{basiclo\_dumpable\_module}}) is emitting its C code
    in the
    {\small{\texttt{modules/modbm\_3BFt4NfJmZC\_7iYi2dwM38B.c}}} file,
    so that code file is part of the persistent state.
\end{itemize}

Ideally, in the future (after end of \textsc{Chariot}), the
\emph{Bismon monitor} should be entirely bootstapped~\footnote{This
  was not completely the case of \emph{GCC MELT}, but almost: about
  80\% of \emph{GCC MELT} at the end of that project was coded in
  \emph{MELT} itself. However, it was tied to a particular version of
  \emph{GCC}.}, so all its files should be generated (including what
is still in 2018 the hand-coded ``runtime'' part of \emph{Bismon} such
as our \texttt{*\_BM.c} files, notably the load and dump machinery in
\texttt{load\_BM.c} and \texttt{dump\_BM.c}, the agenda mechanism in
\texttt{agenda\_BM.c}, miscellanous routines including the support of
module loading in \texttt{misc\_BM.cc}, etc, etc...). Sadly, we are
still quite far from that ideal and this is annoying. Existing
bootstrapped systems~\footnote{Observe that an entire Linux
  distribution is also, when considered as a single system of ten
  billions lines of source code, fully bootstrapped. You could
  regenerate all of it. See \bmurl{http://www.linuxfromscratch.org/}
  for guidance.} such as CAIA {\small (see \cite{Pitrat:blog} - it
  explains that all the 500KLOC of the C code of CAIA are generated)},
Ocaml {\small (cf \cite{Ocaml, Leroy-modular-modules} and many other
  papers by Xavier Leroy and the
  \href{http://gallium.inria.fr/}{Gallium} team at INRIA)}, Self
{\small (cf \cite{Ungar:1987:Self})}, and Clasp {\small (cf
  \cite{Schafmeister:2015:CLASP})} show that it is possible. The major
advantage of generating \emph{all} the code of \emph{Bismon} would be
to deal with internal consistency in some automated and systematic way
and facilitate refactoring~\footnote{In 2018, if we decide painfully
  to change the representation of attribute associations in objects,
  we have to modify a lot of hand-written code and objects
  simultaneously, and that is a difficult and brittle effort of
  refactoring. If all our code was generated, it would be still hard,
  but much less.}. An important insight is that the behavior of a
bootstrapped system can be improved in two ways: the ``source'' of the
code could be improved (in the case of \emph{Bismon}, all the objects
describing some module) and the ``generator'' of the code could also
be improved (cf. \index{partial evaluation}{partial evaluation} and
Futamura projections, e.g. \cite{Futamura:1999:PartialEval}; also,
\cite{Pitrat:2009:ArtifBeings, Pitrat:blog} gives some interesting
perspectives for artificial intelligence with such an approach).

%%%%%%%%%%%%%%%%
\subsubsection{persisting objects}
\label{subsubsec:persistobj}

Obviously, the objects of \emph{Bismon} (§~\ref{subsubsec:objects})
may have circular references, and circularity can only happen thru
objects (since other composite values such as nodes or sets are
immutable, §~\ref{subsubsec:immutvalues}). So the initial loader of
the persistent state proceeds in two passes. The first pass is
creating all the persisted objects as empty and loads the modules
needed by them, and the second pass is filling these objects.

\bigskip

\begin{figure}[h]
\fbox{\parbox{\textwidth}{
\begin{scriptsize}
  \verbatiminput{generated/001-extract-firsttestmodule-dump.tex}
\end{scriptsize}
}}
  \caption{dump example: \texttt{first\_test\_module} in file
    \texttt{store2.bmon}}
  \label{fig:objdumpfirsttestmodule}
\end{figure}

\bigskip

The figure \ref{fig:objdumpfirsttestmodule} shows an example of the
textual dump for some object (named \texttt{first\_test\_module}) of
objid {\small\texttt{\_9oXtCgAbkqv\_4y1xhhF5Nhz}} extracted from the
data file \texttt{store2.bmon}.

\medskip

\begin{quote}
\begin{small}
  
  The lines starting with \texttt{!(} and with \texttt{!)} are
  delimiting the object. Comments~\footnote{Once the persistence code
    - loading and dumping of the state - is mature enough, we will
    disable generation of comments in data files.} can start with a
  bar \texttt{|} till the following bar, or with two slashes
  \texttt{//} till the end of line.  {\verb+!~+} with matching
         {\verb+(~+} ... \verb+~)+ are for ``modifications'' (here, we
         set the \index{name}{\texttt{name}} of that object to
         \texttt{first\_test\_module}). Object payloads are also
         dumped in such ``modification'' form. \verb+!@+ puts the
         \index{mtime}{\emph{mtime}}. \verb+!$+ \textit{classobjid}
         sets the \index{class}{class} to the object of objid
         \textit{classobjid}. \verb+!:+ \textit{attrobj}
         \textit{valattr} put the attribute \textit{attrobj}
         associated with the value \textit{valattr}. \verb+!#+
         \textit{nbcomp} reserve the spaces for \textit{nbcomp}, and
         \verb/!&/~\textit{valcomp} appends the value \textit{valcomp}
         as a component.
  
\end{small}
\end{quote}

\begin{figure}[h]
  \begin{center}
  \begin{tabular}{l c l l}
    \emph{value} & $\leftarrow$ & \emph{int} & ; tagged integers \\
                 & $|$ & \texttt{\textbf{\_\_}} & ; nil \\
                 & $|$ & \texttt{\textbf{"}}\emph{string}\texttt{\textbf{"}} & ; string  {\scriptsize with JSON-like escapes} \\
                 & $|$ & \emph{objid} & ; object  {\scriptsize of given \emph{objid}}\\
                 & $|$ & \texttt{\textbf{\{}} \emph{objid}$_{elem}$ {...} \texttt{\textbf{\}}}  & ; set  {\scriptsize of elements of given \emph{objid}}\\
                 & $|$ & \texttt{\textbf{[}} \emph{objid}$_{comp}$ {...} \texttt{\textbf{]}}  & ; tuple  {\scriptsize of components of given \emph{objid}}\\
                 & $|$ & \texttt{\textbf{*}} \emph{objid}$_{conn}$ \texttt{\textbf{(}} \emph{value}$_{son}$ {...} \texttt{\textbf{)}}  & ; node {\scriptsize of given connective and son[s]}\\
                 & $|$ & \texttt{\textbf{\%}} \emph{objid}$_{conn}$ \texttt{\textbf{(}} \emph{value}$_{son}$ {...} \texttt{\textbf{)}}  & ; closure {\scriptsize of given connective and son[s]}\\
  \end{tabular}
  \end{center}
  \caption{syntax of values in dumped data files.}
  \label{fig:dumpvalsyntax}
\end{figure}

Data files start first with the \emph{objid} of modules used by
routines (in objects mentioned in that data file). These module-objids
are prefixed with {\bf{\verb+!^+}}. Then the collection of
objects (similar to figure \ref{fig:objdumpfirsttestmodule} each)
follows.

In data files, objects are represented by their \emph{objid}, perhaps
followed by a useless comment like
{\bf{\verb+|+}}\emph{this}{\bf{\verb+|+}}. And immutable values are in
the grammar given in figure \ref{fig:dumpvalsyntax} (where
\textbf{\texttt{\_\_}}, representing \emph{nil}, can also appear
inside tuples, nodes, closures but not within sets).  At dump time, a
\index{transient}{transient} object is replaced as \emph{nil}, so may
be dumped as \textbf{\texttt{\_\_}}. Within a set, it is skipped so
ignored. When the connective of a node or of a closure is a transient
object, that node or closure is not dumped, but entirely replaced by
\emph{nil} so dumped as \textbf{\texttt{\_\_}}.

The dump works, in a similar fashion of our naive GC, in two phases: a
scanning phase to build the hash-set of all dumped objects. A queue of
objects to be scanned is also used. Then an emission phase is dumping
them (one data file per object space). So dumping happens by sending
messages with selectors like \texttt{\textit{dump\_scan}},
\texttt{\textit{dump\_value}}, \texttt{\textit{dump\_data}}.




\newpage

% file staticanalys-bm.tex, which is \input from bismon-chariot-doc.tex
\section{Static analysis of source code in \emph{Bismon}}
\label{sec:staticanalys}

Static analysis involves a \emph{generated} GCC plugin (whose C++ code
is generated by the \emph{bismon} persistent monitor) which
communicates with the monitor and sends to it some digested form of
the analyzed C or C++ code. Some translation-unit specific processing
can happen in that GCC plugin, but the whole program aspects of the
static code analysis should obviously be done inside the monitor, and
requires -and justifies- its persistence. The complexity and
non-stability of \emph{GCC} internal representations justify some
semi-automatic approach in extracting them (see \S\ref{subsec:analygcc}
below).

The rest of this chapter will be written in the final D1.3\textsuperscript{v2} version.

A significant part of this chapter should be generated (like \emph{GCC
  MELT} generated its documentation, see
\cite{Starynkevitch-GCCMELTweb}) from the persistent state of
\emph{Bismon}. Perhaps this chapter should be put after the ``using
Bismon'' chapter (\S\ref{sec:using}).

\subsection{static analysis of \emph{GCC} code}
\label{subsec:analygcc}

The \emph{GCC} compiler has a complex (and ill-defined,
under-documented and evolving, so unstable) application programming
interface (API) which can be used by plugins. So \emph{Bismon} needs
to analyze the various \emph{GCC} plugin related \emph{header files}
to extract important information about that API, so to be later able
to generate \emph{GCC} plugin code. Such an extraction (inspired by
the approach inside \emph{Clasp}, which does similar things with the
help of \emph{Clang}, see \cite{Schafmeister:2015:CLASP} for details)
needs not to care about the \emph{Gimple} instructions, but only about
the abstract syntax tree in \emph{Tree} and \emph{Generic} forms (see
\cite{gcc-internals} \S11) to retrieve the full description of
\emph{GCC}.

This approach of extracting semi-automatically~\footnote{We are well
  aware that some work still needs to be done manually, in particular
  giving the really useful subpart of the \emph{GCC} API.} the GCC API
(of parsing GCC header files with some simple GCC plugin) is motivated
by past GCC MELT experience (where every feature of the GCC API had to
be \emph{explicitly} and manually described in MELT language; these
descriptions took a lot of time to be written and had to be manually
maintained; however, most of them could in theory be extracted
automatically from GCC headers).

A bootstrapping and incremental approach, in several ``steps'', is
worthwhile (and possible because of persistence): we will first
extract some very simple information from GCC header files, and use
them to improve the next extraction from the same GCC header
files. The \emph{slow} evolution~\footnote{GCC internals are
  \emph{slowly} evolving, because GCC itself is huge: its
  ``navigation'' is as slow as that of a supertanker which needs hours
  to turn and change directions. So for \emph{social} reasons the GCC
  community is changing the API slowly, but there is no promise of
  stability.}  of GCC API is practically relevant (most of the API of
\texttt{gcc-8.3} should stay in the next \texttt{gcc-9.0} version).

Descriptive data related to the API of a particular version of GCC
will thus stay persistently in the \emph{Bismon} monitor, but should
be updated at each release of \emph{GCC}. We care mostly about API
related to optimization passes, \emph{GENERIC}, \emph{Gimple},
\emph{SSA} and \emph{Optimized-Gimple}. We probably don't need to go
at the \emph{RTL} level. The version
\href{https://gcc.gnu.org/gcc-10/}{10 of \textsc{Gcc}} (released in
May 2020) incorporates several
\href{https://gcc.gnu.org/onlinedocs/gcc/Static-Analyzer-Options.html}{static
  analysis options}, that are activated with the \texttt{-fanalyzer}
option to \texttt{g++} or \texttt{gcc}. The version
\href{http://lists.llvm.org/pipermail/llvm-announce/2020-March/000087.html}{10
  of \textsc{Clang}} (released in March 2020) contains an improved
\href{https://clang-analyzer.llvm.org/}{Clang static analyzer} and
\href{https://clang.llvm.org/extra/clang-tidy/}{\texttt{clang-tidy}}
``linter'' like tool, check both coding styles and portability
related conventions. Both compilers are open source and available
\footnote{Both recent \textsc{Gcc} and \textsc{Clang} are buildable as
cross compiler for major 32 bits or 64 bits architectures such as
PowerPC, x86, x86-64, or ARM, provided one download their source
code.} on a \textsc{Linux} desktop and both should be of interest for
advanced \emph{IoT} software developers coding in C or in C++ and
capable of using the command line. \textsc{Gcc} analysis features be
configured thru appropriate \texttt{\#pragma}-s, and \textsc{Clang}
analysis features are changeable by conventional comments such as
\texttt{// NOLINT(google-explicit-constructor, google-runtime-int)}.

%Additional content of this \S\ref{subsec:analygcc} will be written
%for the final D1.3\textsuperscript{v2}.


\subsection{static analysis of IoT firmware or application code}
\label{subsec:analysiot}

Once the API of the current version of \emph{GCC} is known to the
persistent monitor, we can generate the C++ code of \emph{GCC} plugins
for cross-compilers used by IoT developers.

A first static analysis, useful to IoT developers, will be related to
whole-program detection of \emph{stack overflow} \index{stack
  overflow} (see also \cite{Payer:2018:MSV}). By the way, such an
analysis is currently not doable by \emph{Frama-C}, because it don't
know the size of each call frame. However, \emph{GCC} is already
computing that size (see the \texttt{-fstack-usage} option which dumps
the size of the call frame of each function, and the
\texttt{-Wframe-larger-than=\emph{bytesize}} option), and we simply
need to extract and keep it. We also need to get a good approximation
of the \emph{control flow graph\index{control flow graph}}. For that
we need to extract basic blocks and just \texttt{GIMPLE\_CALL}
\emph{Gimple} statements (ignoring other kinds of \emph{Gimple}
statements). Of course, indirect calls (thru function pointers, which
are infrequently used in most IoT code) are harder to handle (and
could require interaction with the IoT developer using our monitor, to
annotate them).

A proof-of-concept GCC plugin for GCC 8 (and 9) to take advantage of
existing internal GCC passes to compute some upper approximation of
the call stack size has been developped. That hand-written GCC plugin,
coded in file \texttt{gcc8plugin-demo-chariot-2019Q2.cc} of about a
thousand lines of C++, communicate with the \texttt{bismon} monitor
using some REST HTTP protocol with ad-hoc HTTP \texttt{POST} requests
having a JSON payload, in some \textsc{Chariot} specific JSON
format. The \texttt{bismon} monitor should display that diagnostic in
a Web browser tab. It could also use the \emph{language server
  protocol}~\footnote{See \bmurl{https://langserver.org/} for more.}
which is, in 2019, understood by most free software source code
editors running on Linux, including \texttt{emacs}, or \texttt{vim},
or \texttt{VSCode}. It might even later use the new
\emph{Sarif}\footnote{See
  \bmurl{http://docs.oasis-open.org/sarif/sarif/v2.0/csprd01/sarif-v2.0-csprd01.html}
  for more.}  protocol, designed for communication between static
source code analyser.

Notice that according to
\href{https://www.zdnet.com/article/chrome-70-of-all-security-bugs-are-memory-safety-issues/}{this
  webpage}, nearly 70\% of security bugs affecting the Chrome web
browser (by Google) are related to memory management issues in C++.
It is expected that junior European software developers of
non-critical IoT systems would experiment a similar bug distribution
in their IoT code. A long-term approach could be the costly training
of IoT software engineers to switch to programming languages with
better memory management, such as \href{https://golang.org/}{Go} or
\href{https://www.rust-lang.org/}{Rust}. However, rewriting an entire
IoT code base is too costly, and mixing several programming
languages\footnote{See
\href{https://softwareengineering.stackexchange.com/questions/370135/why-are-multiple-programming-languages-used-in-the-development-of-one-product-or}{this}
for a discussion of why is that interesting. Notice that recent
\href{http://gcc.gnu.org/}{GCC} compilers share some common internal
representations between several language front-ends.} in the same
software product can be worthwhile but requires some rare and
qualified labor. \href{https://gcc.gnu.org/gcc-10}{GCC 10} has a new
\href{https://gcc.gnu.org/onlinedocs/gcc/Static-Analyzer-Options.html}{static
  analysis} framework (with its \texttt{-fanalyzer} compiler option)
and powerful warning options\footnote{It is helpful to pass
\texttt{-Wall -Wextra} to the \texttt{gcc} or \texttt{g++} compiler,
usually thru some
\href{https://en.wikipedia.org/wiki/Build_automation}{build
  automation} tool such as
\href{http://ninja-build.org/}{\texttt{ninja}}}. The
\href{https://clang-analyzer.llvm.org/}{\textsc{Clang} static
  analyzer} could be useful. Some coding rules (such as
\cite{Holzmann:2006:power-of-10} or \index{Misra@\textsc{Misra C}}
\href{https://www.misra.org.uk/}{\textsc{Misra C}}) are available, but
the \href{https://en.wikipedia.org/wiki/Rice's_theorem}{Rice's
  theorem} \index{Rice's theorem} forbids the possibility of a sound
and complete static analyzer: false alarms cannot be avoided, and code \index{code review}
reviews by senior programmers is still necessary.

We probably would also take as an example the analysis of some
\href{http://mqtt.org/}{MQTT library}. The insight is to trust some
existing MQTT implementation~\footnote{Our purpose is not to prove the
  correctness of a given MQTT implementation, which would require a
  formal methods approach à la \textsc{Vessedia}, but to help the
  developer using and trusting it, by checking some specific coding
  rules.}, and to help \emph{junior} developers in using it, by
checking simple coding rules relevant to MQTT.

An interesting \textsc{Chariot}-compatible approach could be to use
\emph{topological data analysis} \index{topological data analysis}
\index{data analysis!topological} (cf
\cite{Chazal:2017:topodatanalys}) techniques, combined with some
machine learning (cf \cite{flach:2012:machine-learning}) and big data
\index{big data} / data mining \index{data mining} (cf
\cite{wu:2013:big-data-mining, clarke:2016:big-data-risks,
  zuboff:2015:big-other, helbing:2019:big-data-democracy}) approaches,
on some of the several directed graphs (notably the \emph{control flow
graph}, the \emph{call graph}, the \emph{dependency graph} for
example) of the whole analyzed program. Reputable free software
libraries\footnote{See \textsc{TensorFlow} on
\bmurl{https://www.tensorflow.org/}, and \textsc{Gudhi} on
\bmurl{http://gudhi.gforge.inria.fr/}, and many other similar
libraries.} are available on Linux. In principle, such an approach
might be used in \texttt{bismon} for a \emph{semi-automatic} detection
of \index{code smell} \index{smell!code} \emph{code smells}. Sadly,
the lack of allocated human resources, and the strong focus (see
\cite{Heder:2017:TRL}) \index{TRL} on high
TRL\footnote{\emph{Technical Readiness Level} and the
\href{https://en.wikipedia.org/wiki/Technology\_readiness_level}{TRL
  wikipage} for more.} results, forbids even trying such an
interesting approach in \textsc{Chariot}, taking into account that
industrial corporations are not even dreaming of it. However, these
approaches might be tried in some other projects, perhaps
\index{decoder@\textsc{Decoder}}
\href{https://www.decoder-project.eu/}{\textsc{Decoder}}.

\medskip
\subsection{static analysis related to pointers}
\label{subsec:analysptr}

Pointers \index{pointer} are an important part of the C11 and C++11
language specifications (see \cite{C11:std, CplusPlus11:std})}, but
  are difficult to understand. They need several chapters\footnote{A
    novice programmer should be explained that after \texttt{int
      tab[4]; int* p = \&tab+1;} both \texttt{tab[2]} and
    \index{alias!pointer} \texttt{p[1]} are pointer aliases, but
    \texttt{sizeof(tab)} is not \texttt{sizeof(p-1)} even if
    \texttt{tab == p-1}.}  in C or C++ textbooks such as
  \cite{gusted:t2019:modern, Stroustrup:2014:CplusPlus}. Practically
  \index{null@\texttt{NULL} in C}
  \index{nullptr@\texttt{\textbf{nullptr}} in C++} speaking, a pointer
  is an address, with \texttt{NULL} (in C) or
  \texttt{\textbf{nullptr}} (in C++) having a special and
  distinguished meaning: it is never the same as the result of the
  \index{address} \emph{address-of} operator (unary \texttt{\&} prefix
  operator).

 From the IoT or firmware developer's point of view, pointers -viewed
 as addresses- may behave strangely in practice, and differently from
 the language specifications: in theory, the \texttt{NULL} pointer
 might not sit at address 0. In practice, IoT or firmware developers
 do know that (on most implementations) it \emph{is} at address
 0. Dereferencing the \texttt{NULL} pointer is the prototypical
 example of \emph{undefined behavior}, yet some firmware
 code\footnote{A typical example would be some \textsc{Bios} or
   \textsc{Uefi} firmware or operating system kernel on most PC
   desktop motherboards, see
   \href{https://osdev.org/}{\texttt{osdev.org}} and
   \href{http://tinyvga.com/}{\texttt{tinyvga.com}} for more} may do
 that with good reasons.
 
\bigskip

\hrule


Additional content of this \S\ref{subsec:analysiot} will be written for the final D1.3\textsuperscript{v2}.


%%%%%%%%%%%%%%%%%%%%%%%%%%%%%%%%%%%%%%%%%%%%%%%%%%%%%%%%%%%%%%%%
%% Local Variables: ;;
%% compile-command: "cd ..; ./build-bismon-doc.sh" ;;
%% End: ;;
%%%%%%%%%%%%%%%%%%%%%%%%%%%%%%%%%%%%%%%%%%%%%%%%%%%%%%%%%%%%%%%%


\newpage
%% file using-bm.tex, which is \input from bismon-chariot-doc.tex
\section{Using \emph{Bismon}}
\label{sec:using}

This section §\ref{sec:using} should become somehow a user manual, and
will be written for the final D1.3\textsuperscript{v2}. It is both for
the ordinary IoT developer just using \emph{bismon} for static
analysis of IoT source code, and for the static analysis expert
configuring and programming it.

Most of that should be generated from data persisted inside \emph{bismon}. Perhaps
should be exchanged with the ``static analysis'' chapter
(§\ref{sec:staticanalys}).

%\bigskip

%Notice that most of \emph{GCC MELT} documentation was generated. See its
%{\small\bmurl{http://starynkevitch.net/Basile/gcc-melt/meltpluginapi.html/meltpluginapi.html}}. Likewise,
%most of this chapter will be machine generated from the persistent
%state of \emph{Bismon}


\subsection{How JSON is used by Bismon}
\label{subsec:json-usage}

The JSON\footnote{See \bmurl{http://json.org/} for more} \index{JSON}
textual format is a convenient, common and compact structured textual
format. It is used in \emph{Bismon}, in particular because of its web
interface, and supported as a payload \index{payload} (but not
directly\footnote{Adding immutable JSON values as a new kind of Bismon
  value could be considered in the future.} as an immutable value) for
objects of class \texttt{json\_object}.  \index{json
  object@\texttt{json\_object}}

Conceptually, the JSON model is close, but not identical to, the
Bismon persistent model: it provides structured and compositional
\index{object}%
\index{attribute}%
\index{component}%
\index{payload}%
constructs, but JSON objects have \emph{strings} as attributes, while
Bismon objects have arbitrary object references as attributes, and
also components and some optional payload.

\subsubsection{The canonical \emph{JSON} encoding of \emph{Bismon} values}
\label{subsubsec:canonical-json-encoding}

Therefore, there is some way to encode any \emph{Bismon} value into a
JSON value; this is the \emph{canonical JSON encoding of
  values}\index{canonical JSON encoding}, given in figure
\ref{fig:canon-json} below.

\begin{figure}[!htbp]
  \begin{relsize}{-1}
  \begin{center}
    \begin{tabular}{lclp{5cm}}
      %%%
      $\llbracket \textrm{nil} \rrbracket_{json}$ \index{nil} &
      $\rightarrow$ &
      \texttt{\textbf{null}}  &
      {\relsize{-1}{The Bismon nil is encoded as the JSON null}} \\
      %%%
      $\llbracket \textrm{unspecified} \rrbracket_{json}$ &
      $\rightarrow$ &
      \texttt{\textbf{false}}  &
      {\relsize{-1}{The Bismon \emph{unspecified} \index{unspecified} is encoded as the JSON false}} \\
      %%%
      $\llbracket \textrm{integer}~ i \rrbracket_{json}$ &
      $\rightarrow$ &
      $i$ {\relsize{-1}{(JSON integer)}} &
      {\relsize{-1}{tagged integers \index{integer} encoded as is}} \\
      %%%
      $\llbracket \textrm{boxed float}~ f \rrbracket_{json}$ &
      $\rightarrow$ &
      $f$ {\relsize{-1}{(JSON float)}} &
      {\relsize{-1}{boxed doubles \index{double} encoded as is, with decimal point}} \\
      %%%
      $\llbracket \textrm{string}~ s \rrbracket_{json}$ &
      $\rightarrow$ &
      $s$  {\relsize{-1}{(JSON string)}}  &
      {\relsize{-1}{Bismon strings \index{string} encoded as is}} \\
      \rule{0pt}{3ex} 
      %%%
      $\llbracket \textrm{object}~ \omega$ {\relsize{-1.5}{\textrm{of objid}~ \textit{oid}}}$ \rrbracket_{json}$ \index{object} \index{objid} &
      $\rightarrow$ &
      \texttt{\textbf{\{ "!oid" :}} $\mathit{oid}$ \texttt{\textbf{\}}} &
             {{\relsize{-1}{Bismon objects encoded with \texttt{"!oid"} JSON attribute giving the \index{objid}objid as a JSON string}}} \\
              \rule{0pt}{3ex}
      %%%
      $\llbracket \textrm{tuple}~ [ \omega_1 \ldots \omega_n ] ~ \rrbracket_{json}$ \index {tuple} &
      $\rightarrow$ & %
      \begin{minipage}[c]{4cm}
         \texttt{\textbf{\{ "!tup" :}} \\
         \hspace*{1.2em}  $\mathtt{\texttt{[}} ~ \mathit{oid}_1 \ldots \mathit{oid}_n ~ \mathtt{\texttt{]}}$ \texttt{\textbf{\}}}
      \end{minipage}
      & {\relsize{-1}{Bismon tuples encoded  with \texttt{"!tup"} JSON attribute giving the JSON array of corresponding objid JSON strings : ~  \mbox{$\mathit{oid}_i = \mathrm{objid} (\omega_i)$}
            }}
      \\
       \rule{0pt}{3ex}
      %%%
      $\llbracket \textrm{set}~ \{ \omega_1 \ldots \omega_n \} ~ \rrbracket_{json}$ \index{set} &
      $\rightarrow$ & %
      \begin{minipage}[c]{4cm}
         \texttt{\textbf{\{ "!set" :}} \\
         \hspace*{1.2em}  $\mathtt{\texttt{[}} ~ \mathit{oid}_1 \ldots \mathit{oid}_n ~ \mathtt{\texttt{]}}$ \texttt{\textbf{\}}}
      \end{minipage}
      & {\relsize{-1}{Bismon sets encoded  with \texttt{"!set"} JSON attribute giving the JSON array of corresponding objid JSON strings : ~  \mbox{$\mathit{oid}_i = \mathrm{objid} (\omega_i)$}
            }}
      \\
       \rule{0pt}{3ex}
      %%%
      $\llbracket \textrm{node}~ \mathtt{\textbf{*}} ~ \omega_{conn} \mathtt{(} \sigma_1 \ldots \sigma_n \mathtt{)} ~ \rrbracket_{json}$ \index{node} &
      $\rightarrow$ & %
      \begin{minipage}[c]{4.5cm}
        \texttt{\textbf{\{ "!node" :}} $\mathit{oid}_{conn}$ \texttt{\textbf{,}} \\
         \hspace*{0.8em}  \texttt{\textbf{"!sons" :}} \\
         \hspace*{1.2em} %
         $\mathtt{\texttt{[}} ~ \llbracket \sigma_1 \rrbracket_{json} ~ \ldots ~  \llbracket \sigma_n \rrbracket_{json} ~ \mathtt{\texttt{]}}$
         \texttt{\textbf{\}}}
      \end{minipage}
      & {\relsize{-1}{Bismon nodes encoded  with \texttt{"!node"} JSON attribute giving the objid $\mathit{oid}_{conn} = \mathrm{objid} (\omega_{conn})$ of the connective $\mathit{oid}_{conn}$, and with  \texttt{"!sons"} JSON attribute associated to the array of encodings of that node's sons $\sigma_i$}
            }
      \\
       \rule{0pt}{3ex}
      %%%
      $\llbracket \textrm{closure}~ \mathtt{\textbf{\%}} ~ \omega_{rout} \mathtt{(} \kappa_1 \ldots \kappa_n \mathtt{)} ~ \rrbracket_{json}$ \index{closure} &
      $\rightarrow$ & %
      \begin{minipage}[c]{4.5cm}
        \texttt{\textbf{\{ "!clos" :}} $\mathit{oid}_{rout}$ \texttt{\textbf{,}} \\
         \hspace*{0.8em}  \texttt{\textbf{"!cval" :}} \\
         \hspace*{1.2em} %
         $\mathtt{\texttt{[}} ~ \llbracket \kappa_1 \rrbracket_{json} ~ \ldots ~  \llbracket \kappa_n \rrbracket_{json} ~ \mathtt{\texttt{]}}$
         \texttt{\textbf{\}}}
      \end{minipage}
      & {\relsize{-1}{Bismon closures encoded  with \texttt{"!clos"} JSON attribute giving the objid  $\mathit{oid}_{rout} = \mathrm{objid} (\omega_{conn})$ of the closure's routine, and with  \texttt{"!cval"} JSON attribute associated to the array of encodings of that closure's closed values $\kappa_i$}
            } \\
      \hline \\
      \rule{0pt}{6ex} \\
      \begin{minipage}[c]{3.5cm}
        $ \nu = $ \\ $\mathrm{apply}  ~ \mathtt{\textbf{\%}} ~ \omega_{rout} \mathtt{(} \kappa_1 \ldots \kappa_n \mathtt{)} $ \\
        to $\mathit{ctxt}, \mathit{depth}$ 
      \end{minipage} &
      $\leftarrow$ & %
      \begin{minipage}[c]{4.5cm}
        \texttt{\textbf{\{ "!apply" :}} $\mathit{oid}_{rout}$ \texttt{\textbf{,}} \\
         \hspace*{0.8em}  \texttt{\textbf{"!cval" :}} \\
         \hspace*{1.2em} %
         $\mathtt{\texttt{[}} ~ \mathit{json}_1 \ldots \mathit{json}_n ~ \mathtt{\texttt{]}}$
         \texttt{\textbf{\}}}
      \end{minipage}
      & {\relsize{-1}{application of object routine $\omega_{rout}$ whose objid is $\mathit{oid}_{rout}$ with closed values $\kappa_1 = \mathit{encode}_{json} ( \mathit{json}_1 )$ \ldots  $\kappa_n = \mathit{encode}_{json} ( \mathit{json}_n )$ to the context, the depth  (during decoding) \textcolor{red}{@@NOT SURE}}}
      \\
       \rule{0pt}{3ex}
    \end{tabular}
    \smallskip
  \end{center}
  \end{relsize}
  \caption{canonical JSON encoding $\llbracket v \rrbracket_{json}$ of a Bismon value $v$.}
  \label{fig:canon-json}
\end{figure}

The canonical JSON encoding is implemented\footnote{Coded in C, in file
\href{https://github.com/bstarynk/bismon/blob/master/jsonjansson\_BM.c}{\texttt{jsonjansson\_BM.c}}}
as the \texttt{canonjsonifyvalue\_BM} \index{canonjsonifyvalue!\texttt{canonjsonifyvalue\_BM}} function.

\medskip

\subsubsection{The nodal \emph{JSON} decoding into \emph{Bismon} values}
\label{subsubsec:nodal-json-decoding}

Since JSON is a structured and compositional, \index{tree!JSON}
tree-like, representation, and because \index{node} nodes are the only
kind of structured immutable \emph{Bismon} values,
\index{decoding!JSON nodal} \index{nodal!JSON decoding} any JSON value
can obviously be decoded into a \emph{Bismon} values, using mostly
nodes for structuring data, following the rules listed in
\ref{fig:nodal-json} below. Actually, there are several variants of
nodal decodings, depending on how JSON strings looking like
\index{objid} full objids (e.g. JSON
\texttt{"\_756o00yB7Zs\_1USbaS25hxl"}), or abbreviated objids
(e.g. JSON \texttt{"€\_9Z2BgJbf4"}), or named objects (e.g.  JSON
\texttt{"arguments"}, related to Bismon object \texttt{arguments},
i.e. \texttt{€\_0jFqaPPHg}) are really decoded.

\begin{figure}[!htbp]
  \begin{relsize}{-1}
  \begin{center}
    \begin{tabular}{lclp{5.3cm}}
      %%%
      $\langle \mathtt{\textit{\texttt{null}}} \rangle^{nod}$ \index{null} &
      $\rightarrow$ &
      \texttt{\textbf{json\_null}}  &
             {\relsize{-1}{The JSON \texttt{null}\index{null!JSON} is nodal-decoded as the \texttt{json\_null} Bismon object \texttt{€\_6WOSg1mpN}}}
      \\
      %%%
      $\langle \mathtt{\textit{\texttt{false}}} \rangle^{nod}$ \index{null} &
      $\rightarrow$ &
      \texttt{\textbf{json\_false}}  &
             {\relsize{-1}{The JSON \texttt{false} is nodal-decoded as the \texttt{json\_false} Bismon object \texttt{€\_1h1MMlmQi}}}
      \\
      %%%
      $\langle \mathtt{\textit{\texttt{true}}} \rangle^{nod}$ \index{null} &
      $\rightarrow$ &
      \texttt{\textbf{json\_true}}  &
             {\relsize{-1}{The JSON \texttt{true} is nodal-decoded as the \texttt{json\_true} Bismon object \texttt{€\_0ekuRPtKaI}}}
      \\
      %%%
      $\langle \mathrm{integer} ~ \iota \rangle^{nod}$  &
      $\rightarrow$ &
      tagged integer $\iota$  &
             {\relsize{-1}{The JSON integers are nodal-decoded as the corresponding \emph{Bismon} integer}}
      \\
      %%%
      $\langle \mathrm{real} ~ \delta \rangle^{nod}$  &
      $\rightarrow$ &
      boxed double $\delta$  &
             {\relsize{-1}{The JSON reals are nodal-decoded as the corresponding \emph{Bismon} boxed double}}
      \\
      \rule{0pt}{2ex} \\
      %%%
      $\langle \mathrm{\textrm{objid-looking} ~ string ~} \sigma \rangle^{nod}$ \index{objid} &
      $\rightarrow$ &
      \begin{minipage}{4.7cm}
        object $\omega$, or else a \\
        node $\textbf{*}\; \mathtt{\textbf{\texttt{object}}} (\omega) $ with an \\
        object $\omega \in \{ \mathtt{json\_null},$ \\
        \hspace*{0.4em} $ \mathtt{json\_false},$ $  \mathtt{json\_true},$ \\
        \hspace*{0.4em} $\mathtt{json\_array}, \mathtt{json\_object} \}$
      \end{minipage}
      &
      {\relsize{-1}{an objid-looking string $\sigma$, starting with an underscore \texttt{\_} and matching the objid of an existing object, \emph{may} be nodal-decoded as
          \emph{the existing} object $\omega$ such as
          $\mathrm{objid}(\omega) = \sigma$} or else as a node of connective \texttt{object} if $\omega$ is a special object mentioned here.}
      \\
      %%%
      $\langle \mathrm{\textrm{name-looking} ~ string ~} \sigma \rangle^{nod}$ \index{objid} &
      $\rightarrow$ &
      \begin{minipage}{4.7cm}
        object $\omega$, or else a \\
        node $\textbf{*}\; \mathtt{\textbf{\texttt{object}}} (\omega) $ with an \\
        object $\omega \in \{ \mathtt{json\_null},$ \\
        \hspace*{0.4em} $ \mathtt{json\_false},$ $  \mathtt{json\_true},$ \\
        \hspace*{0.4em} $\mathtt{json\_array}, \mathtt{json\_object} \}$
      \end{minipage} &
      {\relsize{-1}{a name-looking string $\sigma$, starting with a letter and naming an existing object, is nodal-decoded as the \emph{existing named} object $\omega$ such as $\mathrm{name}(\omega) = \sigma$, or else, for names used here, is decoded as
          \emph{a node}  \mbox{$\textbf{*}\; \mathtt{\textbf{\texttt{object}}} ( \omega) $} such as
          $\mathrm{name}(\omega) = \sigma$}}
      \\
      \rule{0pt}{2ex} \\
      %%%
      $\langle \mathrm{\textrm{any} ~ string} ~ \sigma \rangle^{nod}$ \index{objid} &
      $\rightarrow$ &
      \emph{Bismon} string $\sigma$  &
      {\relsize{-1}{a string $\sigma$ \emph{would} otherwise be nodal-decoded as
          is into the same \emph{Bismon string} $\sigma$}}
      \\
      %%%
      $\langle \mathtt{\textbf{[}} \mathit{\,js}_1 \mathtt{\textbf{,}} \ldots \mathit{\,js}_n \mathtt{\textbf{]}}  \rangle^{nod}$ &
      $\rightarrow$ &
      \begin{minipage}{4.7cm}
      $\mathtt{\texttt{\textbf{*}}} \mathtt{\texttt{\textbf{json\_array}}}$ \\
       \hspace*{1.5em} $ ( \langle \mathit{\,js}_1 \rangle^{nod} \ldots  \langle \mathit{\,js}_n \rangle^{nod} ) $
      \end{minipage}
      &
       {\relsize{-1}{A JSON array is compositionally nodal-decoded into a node of connective \texttt{json\_array} and sons given by nodal-decoding the components of that array}} \\
      \rule{0pt}{6ex} \\
      %%%
      $\langle \mathtt{\textbf{\{}} \alpha_1 \mathtt{\textbf{:}} \mathit{\,js}_1 \mathtt{\textbf{,}} \ldots \alpha_n \mathtt{\textbf{:}} \mathit{\,js}_n \mathtt{\textbf{\}}}  \rangle^{nod}$ &
      $\rightarrow$ &
      \begin{minipage}{4.7cm}
      $\mathtt{\texttt{\textbf{*}}} \mathtt{\texttt{\textbf{json\_object}}} ($ \\
        \hspace*{0.5em} $ \mathtt{\texttt{\textbf{*}}} \mathtt{\texttt{\textbf{json\_entry}}} (  \langle \alpha_1 \rangle^{nod} \mathtt{\texttt{\textbf{,}}} $ \\
         \hspace*{4.5em} $ \langle \mathit{\,js}_1 \rangle^{nod} )$ \\
       \hspace*{2em} $\vdots$ \\
        \hspace*{0.5em} $ \mathtt{\texttt{\textbf{*}}} \mathtt{\texttt{\textbf{json\_entry}}} (  \langle \alpha_n \rangle^{nod} \mathtt{\texttt{\textbf{,}}} $ \\
         \hspace*{4.5em} $ \langle \mathit{\,js}_n \rangle^{nod} )$ \\
        \hspace*{1em} $)$
      \end{minipage}
      &
       {\relsize{-1}{A JSON object is compositionally nodal-decoded into a node of connective \texttt{json\_objects} and sons given by \texttt{json\_entries} subnodes whose first son is a string $\alpha_i$ or object $\omega_i$ or node $\mathtt{\textbf{\texttt{* id}}} (\omega_i)$ where $\alpha_i = \mathrm{objid}(\omega_i)$ or node  $\mathtt{\textbf{\texttt{* name}}} (\omega_i)$ where $\alpha_i = \mathrm{objname}(\omega_i)$ }} \\
      \rule{0pt}{6ex} \\
    \end{tabular}
  \end{center}
  \end{relsize}
  \caption{nodal JSON decoding $\langle \mathit{js} \rangle^{nod}$ of a JSON value $\mathit{js}$.}
  \label{fig:nodal-json}
\end{figure}

{\color{red}{@@ to be completed a lot,
    explaining conversions of Bismon values to and from JSON.}}

\subsection{Web interface internal design}
\label{subsec:webinterf}

The Web interface \index{web interface} of \emph{bismon} is supposed
to be used without malice (see
§\ref{subsubsec:leveraging-static-analysis} and
§\ref{subsubsec:about-bismon}), with a recent graphical web
browser~\footnote{Such as Firefox 60.5 or later, or Google Chrome 72.0
  or later, both exist in 2019Q1.}  using HTTP/1.1. In particular,
\emph{bismon} does not take any measure against
\index{denial-of-service} denial-of-service attacks, since it is
supposed to be used on a trusted and friendly corporate intranet or
local area network, not directly on the wild Internet. The network
administrator running \emph{bismon} could deploy usual relevant
techniques (firewalls, \texttt{iptables}, HTTP proxying, DMZ, etc ...)
to avoid such attacks. In practice, there are few web browsers - so
few HTTP clients - interacting with \emph{bismon} simultaneously :
only a dozen of people in some IoT development team, and each uses
his/her graphical browser \index{browser} - a \emph{recent} Firefox or
Chrome~\footnote{So no particular effort is even taken to support a
  variety of old browsers: we don't have any code to e.g. support
  Internet Explorer pecularities or deficiencies. Likewise,
  scalability to thousands of simultaneous HTTP connections is out of
  scope in \emph{bismon}, but it is essential in most web
  applications.}. Each \emph{Bismon} user is expected to have one, or
only a few, browser tab[s] interacting with the \texttt{bismon}
server, and these tabs, if there are more than one, are handled as
different web browsers so have different web sessions.  They are
physically and geographically located on the same local area network
as the machine running the \emph{bismon} monitor. So, from web
technologies perspective, \emph{bismon} is making different
trade-offs~\footnote{For example, we could accept making some HTTP
  exchange - e.g. with AJAX - on \emph{every} keystroke on the
  keyboard, but such practice won't be acceptable in usual web
  services. Also, we don't care much about minimizing the HTTP
  exchanges - no ``minification'' needed in practice!} than
``traditional'' web servers or web applications : the web browser
$\leftrightarrow$ \emph{bismon} web server round-trip transmission
time is supposed to be very small so frequent AJAX requests are
possible, the bandwidth is expected to be quite large so voluminous
HTTP responses are acceptable, the number of simultaneous web
connections or of web sessions is tiny. Therefore most web
optimizations are practically unneeded.

With its initial (and current, in mid-2019) naive stop-the-world
garbage collector, the interactive performance and user experience
(i.e. user look-and-feel) of \emph{Bismon} is expected to be
unsatisfactory (since that GC could ``block'' the \texttt{bismon}
monitor and web service for more than half a second - during which the
web interface stays unresponsive, if running the GC on a large enough
heap; but see footnote \ref{fn:initial-gc} suggesting an
improvement). With significant work, that could be improved.

Each HTTP request \index{request} \index{web request} either
corresponds to a ``static'' file path under \texttt{webroot/} (for a
\texttt{GET} or \texttt{HEAD} HTTP request) or else it is handled
dynamically. For a static file path, that file is served directly by
\texttt{onion\_handler\_export\_local\_new} with a
\texttt{Content-Type} corresponding to its suffix; for example an HTTP
\texttt{GET} request of \texttt{favicon.ico} is answered with the
content of \texttt{webroot/favicon.ico} file, and an HTTP \texttt{GET}
request of \texttt{jscript/jquery.js} is served by the content of
\texttt{webroot/jscript/jscript.js}. Care is taken~\footnote{In
  particular, any HTTP request containing \texttt{..} is rejected.} to
avoid serving any static file outside of \texttt{webroot/}. So the
\texttt{webroot/} directory contains static content such as images,
external JavaScript libraries, CSS stylesheets, etc... Static content
requests are always handled the same, so they work even without any
cookies. \index{request!static} \index{static request}

Any HTTP request which cannot be handled as a static resource like
above, because it has no corresponding file under \texttt{webroot/},
is considered as a request for dynamic content and is called a
\index{request!dynamic} \index{dynamic request} \emph{dynamic
  request}. Dynamic content requires a web session \index{web session}
\index{session!web session} cookie \index{web cookie} \index{cookie}
named \texttt{BISMONCOOKIE} which contains~\footnote{A practical
  example of \texttt{BISMONCOOKIE} value might be
  \texttt{n000041R970099188t330716425o\_6IHYL1fOROi\_58xJPnBLCTe}: 41
  is the serial number counting web sessions in the running
  \emph{bismon} process, 970099188 and 330716425 are two random
  numbers, \texttt{\_6IHYL1fOROi\_58xJPnBLCTe} is the
  randomly-generated objid of the web session object.} a cryptographic
quality hash (in practice unforgeable) and mentions the objid
\index{objid} (cf §\ref{subsubsec:objects}) of some web session
object. If there is no cookie, or if the \index{cookie} cookie is
invalid or wrong (e.g. forged), a login \index{login} form is
returned. So any HTTP request for a dynamic content (that is which is
not handled as a static resource like above) is rejected (with HTTP
status \texttt{403 Forbidden}) if the user (a \index{contributor}
contributor in \emph{bismon} parlance,
cf. §\ref{subsubsec:bismon-evolving}) is not logged in.

Dynamic requests are reified as very temporary \emph{bismon} objects
of class \texttt{webexchange\_object}. Their \emph{web exchange}
payload~\footnote{There is no programmatic way to create such a web
  exchange payload. It can only be created by processing such dynamic
  HTTP requests.} contains not only a string buffer~\footnote{Since a
  string buffer should contain valid UTF-8 string content without nul
  bytes, this restriction forbids binary contents in HTTP replies to
  dynamic requests. Hence, dynamically computed image contents are not
  possible, unless they use a textual format like SVG.}, to be filled
with the HTTP response content, but also mentions the web request (as
processed by \texttt{libonion}) and the web session object computed
from the \texttt{BISMONCOOKIE} and may contain some arbitrary data
value. The web exchange object is supposed to be filled -like string
buffers are- and at last given some integer HTTP status and
immediately sent back to the browser.  Their web session object is
created at web login time and is of class
\texttt{websession\_object}~\footnote{So the only way to create a web
  session payload is thru the login form. There is no programmatic way
  to create it.}. It knows the contributor who is logged in, the
expiration time of the session, some session data (an arbitrary
\emph{bismon} value; of course more data can sit in attributes or in
components of that web session object), and the web socket \index{web
  socket} connection (if any) to the browser using that session. The
session storage~\footnote{See
  \url{https://developer.mozilla.org/en-US/docs/Web/API/Window/sessionStorage}
  for more.} associated to key \texttt{"bismontab"} identifies and
gives the tab number in the browser. An inactive web session expires
in about an hour~\footnote{See the
  \texttt{USER\_WEBSESSION\_EXPIRATION\_DELAY} constant in
  \texttt{web\_ONIONBM.c}}.

Of course, web request objects or web session objects are
\index{transient} transient and are not and should not be persisted at
dump \index{dump} time (cf. §\ref{subsec:persistence}). So after each
restart of the \emph{bismon} monitor, its web users
(i.e. contributors) should login \index{login} again.

A dynamic request is handled by some closure \index{closure} and
should be answered in a couple~\footnote{See the
  \texttt{WEBEXCHANGE\_DELAY\_BM} constant in file
  \texttt{web\_ONIONBM.c} ...} of seconds; otherwise a \index{web
  timeout} web timeout \index{timeout!web} occurs. That web handler
\index{web handler} closure is applied to the remaining URL path
string and to the web exchange object created in the
\texttt{libonion}-specific thread dealing with the HTTP request, so
outside of the agenda machinery (cf §\ref{subsec:multithreadist}), and
usually would add some \index{tasklet} tasklet into the \index{agenda}
agenda. Most of the time, a fraction of a second later, some other
tasklet would complete the filling the web request object and give
some HTTP status code such as \texttt{200 OK}, then an HTTP reply is
sent back to the browser. If a timeout occurs because the web request
object has not been taken care of quickly enough, an HTTP \texttt{500
  Internal Server Error} is given back to the browser and that web
request object is cleared.

The mapping between URL paths (or prefixes) and web handler closures
handling dynamic requests for them is given by the
\texttt{webdict\_root}~\footnote{For example: an URL like
  \texttt{http://localhost:8086/show/status} is handled by some
  \emph{bismon} monitor listening HTTP requests on port 8086 and with
  \texttt{webdict\_root} associating the string \texttt{"show"} to
  some closure $\kappa$, that web handler closure $\kappa$ would be
  applied to the suffix string \texttt{"status"} and to the web
  exchange object $\omega$ created for that HTTP request. The result
  of that application is ignored, only side effects -often adding some
  tasklets into the agenda, and/or filling the web exchange object
  with some XHTML5, etc...- are useful. If the string in such a web
  handling dictionnary is associated to some other object
  $\omega_{wh}$ of class \texttt{webhandler\_dict\_object}, that
  dictionnary object $\omega_{wh}$ is recursively explored with the
  rest of that URL path (e.g. \texttt{"status"} in our example).}
dictionnary predefined object; for an empty path in the URL (such as
{\relsize{-1}{\texttt{http://localhost:8086/}}} for example), its
\texttt{web\_empty\_handler} attribute is~\footnote{Since dictionnary
  objects map \emph{non-empty} strings to non-nil values
  (cf. §\ref{subsubsec:objects}).} used.  If finally no web handler
closure is found, an \texttt{404 Not found} status is returned. The
\texttt{the\_web\_sessions} predefined object stores the dictionnary
of transient web session objects and associates a cookie string to its
web session object. That dictionnary is forcibly cleared at start of
the web server inside \emph{bismon}, but it should be loaded empty,
since web session objects are created and should remain transient.

In practice, dynamic requests are usually generating the HTML5
\index{HTML5} content very dynamically. For generated HTML, it is much
easier to produce \index{XHTML5} XHTML5, the XML variant \index{XML}
of HTML5, because its textual syntax is~\footnote{For instance, within
  a \texttt{<script} HTML5 element containing JavaScript, it is not
  even allowed in HTML5 to have \texttt{if (x \&lt; 5)} even if
  ordinary HTML rules suggest to use \texttt{\&lt;} instead of
  \texttt{<} in textual content... That makes compositional generation
  of mixture of HTML and JavaScript emitting HTML much harder.} much
more regular and easier to generate than with plain HTML5.

The \texttt{webxhtml\_module} \index{module!\texttt{webxhtml\_module}}
in \emph{bismon} has code to ease the emission of XHTML5. And XHTML5
fragments are emitted by the \texttt{emit\_xhtml} routine
\index{emit xhtml@\texttt{emit\_xhtml}}
object~\footnote{So that \texttt{emit\_xhtml} is, like PHP, a
  machinery to emit arbitrary XHTML. However, we want to avoid
  thinking -like PHP was originally designed- in terms of emitting a
  stream of characters, and \texttt{emit\_xhtml} is supposed to emit
  structured XHTML from some structured, tree-like, internal
  representation. That internal representation is a DAG (directed
  acyclic graph).}. That \texttt{emit\_xhtml}, which get as arguments:
the value \texttt{v\_html} to emit; an arbitrary web context object
\texttt{o\_emitctx} which might be in simple cases just some web
session object; a string buffer object \texttt{o\_strbuf} which is
often the web exchange object; the tagged integer recursion depth
\texttt{v\_depth}, which is in general \emph{not}~\footnote{In very
  simple cases, without closures or sequences in the DAG of emitted
  values, the depth could be the depth of XHTML elements, so could be
  the indentation. In general, it is not.} the emitted
  indentation. When the string buffer is too full or the recursion
  depth is too deep, that \texttt{emit\_xhtml} fails. When the emitted
  HTML-reifying value \texttt{v\_html} is nil, nothing happens. When
  it is a scalar, it is emitted trivially: a string is emitted
  HTML-encoded (so \texttt{\&lt;} for \texttt{<}, etc...); a tagged
  integer is emitted in decimal notation; When \texttt{v\_html} is an
  object $\omega_{html}$, it is emitted per the following rules:

  \begin{itemize}
    
\item the \texttt{newline} object emits an indented newline.
  \index{newline@\texttt{newline}}
  
\item the \texttt{nlsp} object emits a newline when the current line
  is long enough, or else a space.
  \index{nlsp@\texttt{nlsp}}
  
\item the \texttt{space} object emits a space character.
  \index{space@\texttt{space}}
  
\item instances of \texttt{html\_void\_element\_object} emit some void
\index{html void element object@\texttt{html\_void\_element\_object}}
  element like e.g. \texttt{<hr class='foo'/>} using the
  \texttt{€\_0FRLxSGQlZ} routine. The \texttt{emit\_xhtml\_open}
\index{emit xhtml open@\texttt{emit\_xhtml\_open}}
  selector should emit -as a side effect- the opening tag \texttt{<hr
    class='foo'} without the ending \texttt{/>}, and returns the string naming the tag, e.g. \texttt{"hr"}.
  
\item instances of \texttt{html\_element\_object} emit recursively
\index{html element object@\texttt{html\_element\_object}}
  using \texttt{€\_5NH940lCHYJ} some nested XHTML element starting
  with a start tag like \texttt{<div} but ending with an end tag like
  \texttt{</div>}; the components of that objects are emitted
  recursively (with an incremented recursion depth). The spacing style
\index{html spacing@\texttt{html\_spacing}}
  is first determined by sending \texttt{html\_spacing} with the
  \texttt{o\_emitctx} and the depth. It can be \texttt{newline} for
  indented, newline separated, content; or \texttt{nlsp} for space or
  newline separated, unindented content; or \texttt{space} for space
  separated content; any other spacing style -notably nil- don't emit
  any separators in the content. The start tag is emitted with
  \texttt{emit\_xhtml\_open} returning the tag string like before
  (e.g. \texttt{"span"} for a \texttt{<span class='somecl'} emission,
  then \texttt{>} is output to end the opening tag, then the
  components, then the end tag is emitted, using the returned tag
  string from \texttt{emit\_xhtml\_open}.
  
\item instances of \texttt{html\_sequence\_object} emit
      recursively their components but without surrounding tags.
      \index{html sequence object@\texttt{html\_sequence\_object}}
      
\item instances of \texttt{html\_active\_object} emit recursively HTML
      \index{html active object@\texttt{html\_active\_object}}
      stuff thru a message of selector \texttt{emit\_xhtml} sent to them.
      
\item any other object is emitted by its name, if it has some, or by
  its objid. This is mostly intended to represent common repeated names or
  words by a single and shared object.
\end{itemize}

When  \texttt{v\_html} is a node of connective $\omega_{conn}$, it is emitted per the following rules:

\begin{itemize}

  \index{int@\texttt{int}}
\item if $\omega_{conn}$ is one of \texttt{int}, \texttt{hexa},
  \index{int@\texttt{int}}
  \index{hexa@\texttt{hexa}}
  \index{octa@\texttt{octa}}
  \texttt{octa} and \texttt{v\_html} is an unary node with a integer
  son $n$, that integer $n$ is emitted in decimal, hexadecimal, octal respectively.

\item if $\omega_{conn}$ is \texttt{id}~\footnote{So nodes of
  connective \texttt{id}, \texttt{object} or \texttt{name} can be used
  to emit objects of class \texttt{html\_void\_element\_object},
  \texttt{html\_element\_object}, \texttt{html\_sequence\_object},
  \texttt{html\_active\_object} which would be handled specially
  otherwise.}  and \texttt{v\_html} is an unary node with an object
  \index{id@\texttt{id}}
  \index{object@\texttt{object}}
  \index{name@\texttt{name}}
  son $\omega_{son}$ its objid is emitted.
  
\item if $\omega_{conn}$ is \texttt{buffer} and \texttt{v\_html} is an
  \index{buffer@\texttt{buffer}}
  unary node with an object son $\omega_{son}$ which has a string
  buffer payload, that string is emitted HTML-encoded.
  
\item if $\omega_{conn}$ is \texttt{object} and \texttt{v\_html} is an unary node with an object son $\omega_{son}$ its name or objid is emitted.
  \index{object@\texttt{object}}
  
\item if $\omega_{conn}$ is \texttt{name} and \texttt{v\_html} is an
  unary node with an object son $\omega_{son}$ its name is emitted,
  \index{name@\texttt{name}}
  and when $\omega_{son}$ is not a named object, we have a failure. If $\omega_{conn}$ is \texttt{name} and \texttt{v\_html} is an
  binary node with an object first son $\omega_{son}$, and some arbirary non-nil second son $\epsilon$ its name is emitted,
  and when $\omega_{son}$ is not a named object, the $\epsilon$ is recursively emitted
  
\item if $\omega_{conn}$ is \texttt{sequence}, every son of
  \texttt{v\_html} is emitted in sequence, with an incremented
  $depth$. Nothing is additionally emitted between
  them. \index{sequence@\texttt{sequence}}

\item if $\omega_{conn}$ is \texttt{space}, or \texttt{newline}, or
  \texttt{nlsp}, every son of \texttt{v\_html} is emitted in sequence,
  with an incremented $depth$. Between each son, a space
  (respectively, a newline, or a smart space of newline) is emitted.
  \index{space@\texttt{space}} \index{newline@\texttt{newline}}
  \index{nlsp@\texttt{nlsp}}


\item for any other object $\omega_{conn}$ as connective, we extract
  its \texttt{emit\_xhtml\_node} attribute $v_{emit~node}$ and its
  \texttt{emit\_xhtml\_connective\_open} attribute
  $v_{emit~open}$. Only one of them should be present (non-nil value)
  and it should be a closure. If $v_{emit~node}$ is given, it is
  applied to $\omega_{conn}$ \texttt{o\_emitctx o\_strbuf $depth + 1$
    v\_html}, else if $v_{emit~open}$ is present, we apply it (like
  the \texttt{emit\_xhtml\_open} selector above) to $\omega_{conn}$
  \texttt{o\_emitctx o\_strbuf $depth + 1$ v\_html} to obtain an XHTML
  element tag. We also extract and use its \texttt{html\_spacing}
  attribute. Then proceed like for \texttt{html\_element\_object}
  using the sons as components....
  
\item {\color{red}{@@ to be completed a lot.}}
  
\end{itemize}


When \texttt{v\_html} is a closure, it is applied {\color{red}{@@ to
    be completed}} and the result of that application is recursively
emitted. When \texttt{v\_html} is a sequence (set or tuple), its
components are emitted recursively.

If no rule is applicable, \texttt{emit\_xhtml} fails.

\medskip

The web session objects are also used for \index{WebSocket}
\emph{WebSockets} with the following additional conventions. The
\texttt{bismon} server uses WebSockets only for asynchronous
communication from that \texttt{bismon} server to Web
browsers~\footnote{So web browsers don't communicate
  \emph{asynchronously} with the \texttt{bismon} server. For such
  communications from browser to \texttt{bismon}, Web browser always
  use synchronous HTTP requests, e.g. using AJAX techniques.}. The
WebSocket messages from \texttt{bismon} to web browsers are arbitrary
\emph{JSON} values.

\subsection{Using \texttt{bismon} for \textsc{Chariot}}
\label{subsec:bismon-for-chariot}

To run the \texttt{bismon} monitor for \textsc{Chariot} related
activities, that monitor should initialize its state for these
activities. So you need to pass \texttt{-i init\_chariotdemo} as a
program argument when running \texttt{bismon} in that case.

%%%%%%%%%%%%%%%%%%%%%%%%%%%%%%%%%%%%%%%%%%%%%%%%%%%%%%%%%%%%%%%%
%% Local Variables: ;;
%% compile-command: "cd ..; ./build-bismon-doc.sh" ;;
%% End: ;;
%%%%%%%%%%%%%%%%%%%%%%%%%%%%%%%%%%%%%%%%%%%%%%%%%%%%%%%%%%%%%%%%


\newpage
% file miscwork-bm.tex, which is \input from bismon-chariot-doc.tex
\section{Miscellanous work}
\label{sec:miscwork}

\subsection{Contributions to other free software projects}

\label{subsec:contribfree}
This is related to subtask ST1.3.2 of \textsc{Chariot} GA.

\subsubsection{Aborted contribution to \texttt{libonion}}
\label{subsubsec:contriblibonion}
The \texttt{libonion} library is a free software HTTP server library
(LGPLv3 licensed) that is used in \texttt{bismon} for its web service
feature. See its web site \bmurl{https://www.coralbits.com/libonion/}
for a description, and its source repository
\bmurl{https://github.com/davidmoreno/onion} for more.

The handling of \texttt{SIGTERM} signal (and others) is deemed
unsatifactory. See the opened issue 229 {\relsize{-1}{(on
    \bmurl{https://github.com/davidmoreno/onion/issues/229})}} in
\texttt{libonion}.  We discussed that issue on google group with the
\emph{libonion} community, and came to a disagreement (our design was
considered too complex, but we believe that such a complexity is
needed to avoid bugs in the rare cases of a multi ``\texttt{onion}''
application, which \texttt{bismon} is not).

Independently of that issue, we improved our \texttt{bismon} to avoid
needing or depending on that \texttt{SIGTERM} feature in
\texttt{libonion} (by using \texttt{signalfd} Linux specific
facilities in \texttt{bismon} itself and passing the
\texttt{O\_NO\_SIGTERM} flag to \texttt{onion\_new}...).

So the effort on improving \texttt{SIGTERM} handling in
\texttt{libonion} was concluded.

\subsubsection{Contribution to \emph{GCC}}
\label{subsubsec:contribgcc}
There is no contribution yet to \emph{GCC}, because it is not yet
needed in october 2018. We reserve some effort for future such
contributions, when our \emph{GCC} plugin generator would require
them. In the lucky case where no adaptation of \emph{GCC} plugin
infrastructure is necessary, the effort could be moved to other work
in T1.3 (notably ST1.3.3).

\subsection{Design and implementation of the compiler and linker extension}

\label{subsec:compilinkext}
This is related to subtask ST1.3.4 (and also ST1.3.1) of \textsc{Chariot} GA 

The compiler extensions will be \emph{generated} GCC plugins.

The linker extension will compute some ``cryptographic quality'' hash
code of the C or C++ translation units of the IoT software. Then it
will interact with the blockchain, according to the \emph{§6 API for
  Private key related transactions} of the \emph{D1.2 Method for
  coupling preprogrammed private keys on IoT devices with a Blockchain
  system}. That API is a Web API and a C or C++ compatible plain API
or library should be developed, following the tutorial code example of
D1.2.

This chapter will be updated and completed in the upcoming and final
version (in D1.3~\textsuperscript{v2}).


%%%%%%%%%%%%%%%%%%%%%%%%%%%%%%%%%%%%%%%%%%%%%%%%%%%%%%%%%%%%%%%%
%% Local Variables: ;;
%% compile-command: "cd ..; ./build-bismon-doc.sh" ;;
%% End: ;;
%%%%%%%%%%%%%%%%%%%%%%%%%%%%%%%%%%%%%%%%%%%%%%%%%%%%%%%%%%%%%%%%


\newpage
% file conclus-bm.tex, which is \input from bismon-chariot-doc.tex
\section{Conclusion}
\label{sec:conclusion}

The \texttt{bismon} free software is developed in an agile and
incremental manner~\footnote{So there are no released stable versions
  of this software, but snapshots.} (required by its bootstrapping
approach), with continuous updates to
\bmurl{https://github.com/bstarynk/bismon/}.

In october 2018, the persistence machinery is working and daily used
to enhance \texttt{bismon}. The agenda mechanism is working. A naive
stop-the-world mark-and-sweep precise garbage collector is
implemented. The generation of internal C code is done (by
hand-written routines, still coded in C), this enables the
meta-programming approach. The web interface is worked upon: a
\texttt{libonion} based infrastructure is already handling HTTP
requests, and a GDPR-compliant login form is presented on web
browsers. Our \texttt{jsweb\_module} contains the functions related to
Javascript (nearly complete) and HTML generation (work in progress). The
syntactical editor (replacing the crude GTK interface) and then the
GCC plugin generation should be worked on.

In august 2019, the web machinery is mostly working. More generated C
code is available. The JSON handling is incomplete. Bismon
continuations \index{contination!reification} \index{reification!of
  continuations} are almost\footnote{Thanks to generated invocations of
  the \texttt{LOCALFRAME\_BM}
  \index{LOCALFRAME\_BM@\texttt{LOCALFRAME\_BM}} \emph{C} macro, which
  provides 90\% of the development work: full transient reification of
  partial continuations, that is of \index{call stack} \index{call
    frame} call stack segments, is just a matter of clustering emitted
  stack-local \index{stackframe\_stBM@\texttt{stackframe\_stBM}}
  \texttt{struct stackframe\_stBM}-based linked-lists of Bismon call
  frames.}  reifiable into transient \index{transient!object for
  continuations} objects, having as payload a linked-list sequence of
\index{call frame} call frames.

The final D1.3~\textsuperscript{v2} version (scheduled for M30) of
this deliverable will explain the Web interfaces (both for the
ordinary user, i.e. the IoT developer; and for the static analysis
expert) and the generation of C++ code for GCC plugins, with some
examples of simple, IoT focused, whole-program static source code
analysis performed by \emph{bismon}. So the final
D1.3~\textsuperscript{v2} document will contain a longer conclusion.

Within the timeframe allocated for \textsc{Chariot} it was not
realistically possible in May 2020 to partly or fully generate
\href{https://gcc.gnu.org/onlinedocs/gccint/Plugins.html}{GCC plugins}
C++ code (like past
\href{http://starynkevitch.net/Basile/gcc-melt/}{GCC MELT} did), in
particular because \textsc{Bismon} has no usable documentation, and
understanding its persistent heap of 3485 objects is not realistic
without such a documentation. A garbage collection
\cite{Jones:2016:GC-handbook} design bug (and its subtle interaction
with the powerful but complex \href{https://www.gtk.org/}{GTK}
graphical toolkit) makes the current \textsc{Bismon} (of git
\href{https://github.com/bstarynk/bismon/commit/cb1c4ccfe3802fa330d48fc97c2913943736ba2f}{id
  \texttt{cb1c4ccfe3802fa33}}....) extremely brittle, to the point of
being barely usable. The original insight was to generate most parts
of such \emph{Bismon} documentation, per the
\href{https://en.wikipedia.org/wiki/Unix_philosophy}{Unix philosophy}
and decade of related practice (from the original
\href{https://www.troff.org/}{\texttt{troff}} to prior practice in
\href{http://starynkevitch.net/Basile/gcc-melt/}{GCC MELT}, or to
\href{https://www.doxygen.nl/}{\textsc{Doxygen}} or
\href{http://ocaml.org/}{\textsc{Ocaml}} ...) but such a
documentation, even if it is quite reasonably easy to generate from an
orthogonally persistent
\href{https://en.wikipedia.org/wiki/Semantic_network}{semantic
  network} such as \textsc{Bismon}'s heap, would largely overflow the
70 pages hard limit (\textsc{Chariot} consortium defined) of this
report: notice that the generated
\href{http://starynkevitch.net/Basile/gcc-melt/}{GCC MELT} \index{GCC
  MELT@\textsc{Gcc Melt}} past documentation had hundreds of A4
pages....

Machine learning techniques inspired by \cite{zhang:2019:learned}
could be relevant in \textsc{Bismon}. See also the
\href{http://refpersys.org/}{\textsc{RefPerSys}} \index{RefPerSys
  project@\textsc{RefPerSys} project} research\footnote{Also related:
  \href{https://afia.asso.fr/journee-hommage-j-pitrat/}{talks in the
    memory of J.Pitrat, AFIA, March 6\textsuperscript{th}, 2020,
    Paris.}} project, inspired by \cite{Pitrat:1996:FGCS,
  Pitrat:2009:AST, Pitrat:2009:ArtifBeings,
  Starynkevitch-1990-EUM}. The \index{machine learning}
\index{topological data analysis}
\index{deeplearning4j@\textsc{DeepLearning4J} framework}
\href{https://deeplearning4j.org/}{\texttt{deeplearning4j.org}}
infrastructure (used as a web service), or opensource C++ machine
learning libraries such as \index{mlpack@\textsc{MlPack} machine
  learning framework} \index{gudhi@\textsc{Gudhi} topological data
  analysis framework} \href{https://mlpack.org}{\texttt{mlpack.org}}
or \href{https://www.tensorflow.org/}{\textsc{TensorFlow} machine
  learning}, or \index{ttk@\textsc{Ttk} topological data analysis
  toolkit} topological data analysis libraries such as
\href{https://gudhi.inria.fr/}{\textsc{Gudhi}} or
\href{https://topology-tool-kit.github.io/}{\textsc{Ttk}} (see
\cite{Masood:2019:ttk}) be coupled to \textsc{Bismon}. Such future
work would however require further funding for at least a year of
qualified developer work (see also
\cite{Maglogiannis:2007:emerging-ai-app} and
\index{ai4eu@\textsc{Ai4eu} project} the
\href{https://www.ai4eu.eu/}{\texttt{ai4eu.eu}} project, but don't
forget the empirical
\href{https://en.wikipedia.org/wiki/Hofstadter\%27s\_law}{Hofstadter's
  law}).

\medskip



\newpage
% file appendix-bm.tex, which is \input from bismon-doc.tex
% see https://github.com/bstarynk/bismon for more about Bismon

% https://tex.stackexchange.com/a/226497/42406

\begin{appendices}


\section{Building \texttt{bismon} from its source code}
\label{sec:building-bismon}

We focus here on how to build \texttt{bismon} from its source code on
\href{http://debian.org/}{Debian}-like distributions running on x86-64
computers, such as
\href{https://www.debian.org/releases/stable/}{Debian Buster 10.6}, or
\href{https://ubuntu.com/}{Ubuntu 20.04}, or
\href{https://linuxmint.com/}{Linux Mint 20}. Familiarity with the
command line is required\footnote{For example, the reader is expected
  of being able to build \href{https://www.gnu.org/software/make/}{GNU
    make} or \href{https://www.gnu.org/software/bash/}{GNU bash} from
  their source code.}, with \texttt{root} access (e.g. using
\texttt{sudo}). Fluency with \href{https://git-scm.com/}{\texttt{git}}
is expected, and it is strongly advised to \texttt{git commit} every
few hours (including your persistent store, when \texttt{bismon} is
\emph{not running}).

The reader is expected to be authorized (by his/her management, if
that build is done professionally) to build \texttt{bismon} from its
source code and probably also some recent
\href{https://gcc.gnu.org/}{GCC} cross-compiler on his/her Linux workstation
and should budget several days of work for that.


We use violet Courier font like {\textcolor{violet}{\texttt{echo this}}}
to typeset literal or sample commands to be typed in a terminal
emulator.
  
{\large Be aware that \textbf{\texttt{bismon} requires specifically some
  \href{https://gcc.gnu.org/gcc-10/}{GCC 10} compiler} and won't work
  with e.g. a GCC 9 compiler.}


If you have to build GCC 10 from its source code, take into account the following advices:

\begin{itemize}

\item if your distribution has some slightly older version of
  \texttt{gcc}, get all the build dependencies of that package[s]. On
  Debian or Ubuntu, that means running as root some command similar to
  {\textcolor{violet}{\texttt{aptitude build-dep gcc-9 g++-9}}... Of
    course replace the \texttt{9} by your particular version of
    \textit{GCC}.

\item run {\textcolor{violet}{\texttt{gcc -v}}} and write on paper the
  output. It will inspire the configuration command you will have to
  run later. If the output does not mention \texttt{--enable-plugin}
  you need to recompile GCC. If the version is not GCC 10, you also
  need to recompile GCC. Otherwise, you could use your system's GCC.

\item the GCC compiler needs to be built \textit{outside of its source tree}. Practically speaking, you should:
  \begin{enumerate}
  \item download some \texttt{gcc-10.2.0.tar.xz} file (into
    \texttt{/tmp/} ...) from a mirror site close to your geographical
    location (probably check its integrity using \texttt{sha512sum}).
  \item extract that tarball: {\textcolor{violet}{\texttt{tar xvf /tmp/gcc-10.2.0.tar.xz}}}
    in some existing writable directory (you should not be root), for example your \texttt{\$HOME/gnu/} directory. You'll need
    about a gigabyte of disk space for that source code with more than
    a hundred thousands inodes. You'll get a fresh \texttt{gcc-10.2*}
    source directory.
  \item create a fresh build directory, e.g. \texttt{MyGCC-10-Build},
    using some \texttt{mkdir MyGCC-10-Build} command from the
    \textit{parent} directory (same example: from your
    \texttt{\$HOME/gnu/} directory, but \textbf{not} from
    \texttt{\$HOME/gnu/gcc-10.2}...).
  \item Read carefully the \href{https://gcc.gnu.org/install/}{Installing GCC} chapter.
    Read it several times (the URL is \href{https://gcc.gnu.org/install/}{gcc.gnu.org/install/}).
  \item Take some rest if needed, and whatever nutriment is the best
    to raise your attention (a cup of tea, coffee, etc...)
  \item Configure with great care your GCC compiler from inside
    {MyGCC-10-Build}... You'll certainly want the
    \texttt{--enable-plugin --enable-languages=c,c++,jit
      --enable-host-shared} options to
    \texttt{\$HOME/gnu/gcc-10.2/configure} and you probably want more
    of them. That configure step lasts a few seconds, but is lazy
    (most of the configuration would happen later).
  \end{enumerate}
  
\end{itemize}

\medskip

The build procedure happens in two phases:

\begin{itemize}

\item a \textbf{configuration step}, to be run only once in a while,
  or when your Linux distribution has changed or upgraded, or when you
  have added extra useful libraries, or have upgraded your GCC
  compiler.

\item a \textbf{compilation step}, to be run more frequently
  (e.g. every night using
  \href{https://man7.org/linux/man-pages/man1/crontab.1.html}{\texttt{crontab(1)}}....)

\end{itemize}

\medskip

\subsection{Prerequisites for building \texttt{bismon}}
\label{subsec:prereq-bismon}

The \texttt{bismon} source code is on
\href{https://github.com/bstarynk/bismon/}{\texttt{\textbf{github.com/bstarynk/bismon/}}}
and the reader is expected to be capable of getting that source code
on his/her Linux workstation. A possible command to retrieve that code
might be \texttt{git clone https://github.com/bstarynk/bismon.git} ;
you'll then obtain a \emph{fresh} \texttt{bismon/} subdirectory
containing the source code. About 100Mbytes of disk space (for less
than 2000 inodes) is required.

A recent \texttt{libonion} library\footnote{This is an open source
  library for web HTTP and HTTPS service. It is LGPL licensed.}
(version 0.8 at least) is required. Fetch \texttt{libonion}'s source
code from
\href{https://github.com/davidmoreno/onion}{\texttt{\textbf{github.com/davidmoreno/onion/}}}
and follow its build instructions: probably \texttt{mkdir \_build} then
\texttt{cd \_build} then \texttt{cmake ..} then \texttt{make} and at
last \texttt{sudo make install}. That \texttt{libonion} library needs
less than 25Mbytes of disk space,
\href{https://cmake.org}{\texttt{cmake}} and several libraries (in
particular support for \texttt{openssl}, \texttt{gcrypt},
\texttt{systemd}, \texttt{sqlite3}, \texttt{lzma}, \texttt{libicu},
\texttt{libpam}) to be built. Check and inspect your
\texttt{onion/version.h} header file\footnote{You might use
  \href{https://man7.org/linux/man-pages/man1/locate.1.html}{\texttt{locate(1)}}
  or
  \href{https://man7.org/linux/man-pages/man1/find.1.html}{\texttt{find(1)}}
  to find files on your Linux box. On \emph{my} Linux machine, that
  header file is in \texttt{/usr/local/include/onion/version.h} and
  comes from \emph{libonion}
  \href{https://github.com/davidmoreno/onion/commit/43128b03199518d4878074c311ff71ff0018aea8}{git
    commit \texttt{43128b031995}}....}, it should have some
\texttt{ONION\_VERSION} close to \texttt{0.8.150} at least.

The \href{https://www.gnu.org/software/readline/}{GNU readline}
(GPLv3+ licensed) library is required, at version 8. It is useful for
autocompletion abilities in interactive situations.

Ian Lance Taylor's
\href{https://github.com/ianlancetaylor/libbacktrace}{libbacktrace}
library is needed for backtraces on error and in warnings, and
possibly for future (generated) introspective code. This library takes
advantage of \href{https://en.wikipedia.org/wiki/DWARF}{DWARF}
debugging metadata in object files and executable, so it is advised to
compile every \emph{Bismon} source file (either handwritten or
generated) with \texttt{-g} (and possibly also \texttt{-O2} for
optimization) flag to \texttt{gcc} or \texttt{g++}.


The \emph{Bismon} project \index{build!automation} uses internally \href{https://www.gnu.org/software/make/}{GNU \texttt{make}},
  version 4.2 at least. Our hand-written \texttt{GNUmakefile} is
  driving it.
  
\subsection{File naming conventions in \texttt{bismon}}

By our conventions, files \index{file} \index{naming!conventions}
whose base name\footnote{In the sense of the
\href{https://man7.org/linux/man-pages/man1/basename.1.html}{\texttt{basename(1)}}
command applied to the file path.} start with a single underscore
(that is, a \textbf{\texttt{\_}} character) are generated: for example
\texttt{\_bismon-config.mk} and \texttt{\_bm\_config.h},
etc... However, some of them need to be kept, backed-up and version
controlled but would be regenerated by running
\index{make@\texttt{make}} \index{redump@\texttt{redump}}
\index{bismon-config@\texttt{\_bismon-config.mk}}
\index{bm-config@\texttt{\_bm\_config.h}} \texttt{make redump}.

File names whose base name start with two underscores, such as
\texttt{\_\_timestamp.c}, are temporary and can be removed. They would
be removed by running \texttt{make clean} or \texttt{make
  distclean}.
\index{clean@\texttt{clean}} 
\index{distclean@\texttt{distclean}} Of course,
\href{https://en.wikipedia.org/wiki/Object_file}{object file}s
\index{object!file}
(suffixed \texttt{.o}) and
\href{https://en.wikipedia.org/wiki/Library_(computing)#Shared_libraries}{shared
  libraries} (suffixed \texttt{.so}, see
\index{library!shared}
\cite{Drepper:2011:sharedlib}) are also temporary, and could be
removed then regenerated. Some of these (in particular under
\texttt{modubin/} directory) are
\index{modubin@\texttt{modubin/}} 
\href{https://man7.org/linux/man-pages/man3/dlopen.3.html}{\texttt{dlopen(3)}}-ed.

The main executable is named \texttt{bismon}. But
\texttt{BM\_makeconst} and \texttt{BISMON-config} are auxiliary
metaprograms (generating C or C++ code). All three are
\href{https://en.wikipedia.org/wiki/Executable_and_Linkable_Format}{ELF}
executables.

\medskip

\subsection{Naming conventions and source files organization for \texttt{bismon}}

\medskip

{\large \textbf{naming and coding conventions in hand-written \emph{C} code}}

\begin{itemize}

  \item \textbf{All} public
    \href{https://en.wikipedia.org/wiki/Executable_and_Linkable_Format}{ELF}
    \index{ELF}
    \textbf{names of hand-written functions or global variables} (as
    known to
    \href{https://man7.org/linux/man-pages/man1/nm.1.html}{\texttt{nm(1)}},
    \href{https://man7.org/linux/man-pages/man1/objdump.1.html}{\texttt{objdump(1)}}
    or to
    \href{https://man7.org/linux/man-pages/man3/dlsym.3.html}{\texttt{dlsym(3)}}
    \textbf{are conventionally suffixed by}
           {\texttt{\textbf{\_BM}}}\,. For example, we have some
           \texttt{prime\_above\_BM} function giving some prime number
           above a given reasonable positive integer.

         \item \textbf{We have conventional suffixes:} Our public
           \index{suffix!file} \texttt{struct}-s are generally tagged
           with a name ending with {\texttt{\textbf{\_stBM}}}\,; Our
           \texttt{typedef}-ed types are suffixed with
                  {\texttt{\textbf{\_tyBM}}}\,; usually their field
                  names is globally unique and share a common prefix
                  (e.g. in \texttt{struct parstoken\_stBM} field names
                  all start with \texttt{tok}). Public signatures
                  (useful for C function pointers) are suffixed with
                  {\texttt{\textbf{\_sigBM}}}\, (for example, the
                  initialization of generated modules is a C function
                  of signature \texttt{moduleinit\_sigBM}\,). Most
                  public \texttt{enum}-s have their name ending with
                  {\texttt{\textbf{\_enBM}}}\,
                  e.g. \texttt{space\_enBM} for space numbers.

              \item Preprocessor symbols or macros are in all capital
                ending with {\texttt{\textbf{\_BM}}}\,, notably the
                important \texttt{LOCALFRAME\_BM} variadic macro for
                \index{localframe-bm@\texttt{LOCALFRAME\_BM} macro}
                local roots known to our garbage collector.
\end{itemize}

{\large \textbf{Hand-written \emph{C} code files}}

\begin{itemize}
  \item The header file \texttt{bismon.h} is our
    \index{bismon-h@\texttt{bismon.h} header} only public header file,
    and is \texttt{\#include}d everywhere. It includes system headers
    (e.g. \texttt{<unistd.h>} or \texttt{<pthread.h>}, and the
    following ``internal'' headers:
    \begin{enumerate}
      \item \texttt{cmacros\_BM.h} is \texttt{\#define}-ing important
        \index{cmacro-h@\texttt{cmacros\_BM.h} header} global
        preprocessor macros, including \texttt{FATAL\_BM} for fatal
        errors, \texttt{LOCALFRAME\_BM} variadic macro for local
        roots, \texttt{DBGPRINTF\_BM} for debugging output,
        \texttt{WARNPRINTF\_BM} for warning messages,
        \texttt{INFOPRINTF\_BM} for informational messages, etc... The
        \texttt{ROUTINEOBJNAME\_BM} macro
        \index{fatal-bm@\texttt{FATAL\_BM} macro}
        \index{dbgprintf-bm@\texttt{DBGPRINTF\_BM} macro}
        \index{warnprintf-bm@\texttt{WARNPRINTF\_BM} macro}
        \index{infoprintf-bm@\texttt{INFOPRINTF\_BM} macro}
        \index{routineobjectname-bm@\texttt{ROUTINEOBJNAME\_BM} macro}
        is giving the routine name of \index{objid} a given
        \textit{objid}.

      \item \texttt{id\_BM.h} is implementing
        \index{id-bm@\texttt{id\_BM.h} header} our object ids.

      \item \texttt{types\_BM.h} is defining our global types,
        \index{type-bm@\texttt{types\_BM.h} header} 
        \index{value-tybm@\texttt{value\_tyBM} opaque types} 
        \texttt{struct}-s, etc... Notice the \texttt{value\_tyBM}
        opaque type (a \texttt{void*} pointer) for Bismon values.

        \item \texttt{global\_BM.h} is declaring our \texttt{extern}al
          \index{global-bm@\texttt{global\_BM.h} header} global data,
          some of which is generated. \index{data!global}
        \item \texttt{fundecl\_BM.h} is declaring our global
          \index{fundecl-bm@\texttt{fundecl\_BM.h} header}
          \index{function} hand-written functions. Some of them are
          \texttt{static inline} for efficiency reasons (for example
          \index{elapsertime-bm@\texttt{elapsedtime\_BM.h} function}
          \index{valhash-bm@\texttt{valhash\_BM} function}
          \index{time!elapsed} \index{hash!of values}
          \texttt{elapsedtime\_BM} returning the elapsed clock time as
          a \texttt{double} number in seconds, or \texttt{valhash\_BM}
          to compute the hash code of a Bismon value.

        \item \texttt{inline\_BM.h} is implementing our global \texttt{static
            inline} functions.
          \index{inline-bm@\texttt{inline\_BM.h} header}
          \index{function!inline}
    \end{enumerate}

  \item \texttt{agenda\_BM.c} is implementing our agenda
    \index{agenda} with tasklets \index{tasklet} (see \S
    \ref{subsec:multithreadist} above).
          \index{agenda-bm@\texttt{agenda\_BM.c} file}

  \item \texttt{allocgc\_BM.c} is implementing low-level memory
          \index{allocgc-bm@\texttt{allocgc\_BM.c} file}
    allocation and garbage collector \index{garbage collector} (see \S
    \ref{subsec:gcvalobj} above).
  
  \item \texttt{assoc\_BM.c} is implementing associative lists and
          \index{assoc-bm@\texttt{assoc\_BM.c} file} \index{list!associative}
    tables, \index{object} \index{attribute} \index{association} in
    particular for object attributes.

  \item The \texttt{code\_BM.c} file contains many Bismon routines for
          \index{code-bm@\texttt{code\_BM.c} file} \index{routine}
    \index{closure} closures.

  \item The \texttt{dump\_BM.c} file is implementing the \index{dump} dump of the
    persistent store. \index{persistence} \index{store} See \S
    \index{dump-bm@\texttt{dump\_BM.c} file}
      \ref{subsec:persistence}.

  \item The \texttt{emitcode\_BM.c} file contains many Bismon routines
    for \index{emission} emission or \index{generation} of C code in
    \index{emitcode-bm@\texttt{emitcode\_BM.c} file}
    \index{module} modules.

  \item The \texttt{engine\_BM.c} file is related to \index{tasklet} tasklets 
    \index{engine-bm@\texttt{engine\_BM.c} file} in the agenda (see \S
    \ref{subsec:multithreadist} above).

  \item \texttt{gencode\_BM.c} is related
    \index{gencode-bm@\texttt{gencode\_BM.c} file} to C code
    generation. \index{code!generation} \index{generation!of code}

  \item \texttt{id\_BM.c} implements objid \index{objid}
    \index{id-bm@\texttt{id\_BM.c} file} support.

  \item \texttt{jsonjansson\_BM.c} is for JSON \index{JSON}
    \index{jsonjansson-bm@\texttt{jsonjansson\_BM.c} file} support.
    values.

  \item \texttt{list\_BM.c} is for list \index{value!list}
    \index{list-bm@\texttt{list\_BM.c} file}  values.

  \item \texttt{load\_BM.c} is for loading the persistent store
    \index{load-bm@\texttt{load\_BM.c} file} 
  \index{load} \index{persistence}.

  \item \texttt{main\_BM.c} has the \texttt{main} function 
    \index{main-bm@\texttt{main\_BM.c} file} and support functions (fatal errors, etc...).

  \item \texttt{node\_BM.c} implements node values. \index{node-bm@\texttt{node\_BM.c} file}
    \index{value!node}

  \item \texttt{object\_BM.c} implements
    objects. \index{object-bm@\texttt{object\_BM.c} file}
    \index{object} \index{value!object}

  \item \texttt{parser\_BM.c} implements the parser
    \index{parser-bm@\texttt{parser\_BM.c} file} with callbacks
    \index{callback!parser}

  \item \texttt{primes\_BM.c} contains an array of prime numbers,
    \index{primes-bm@\texttt{primes\_BM.c} file} and related
    utilities. They could be useful in hash tables \index{hash table}
    and in some hash functions. In several cases,
    \href{https://en.wikipedia.org/wiki/Flexible_array_member}{flexible
      array members} inside \textsc{Bismon} are allocated with a prime
    number size. \index{prime} \index{member!flexible array}
    \index{flexible-array@flexible array member}

  \item \texttt{scalar\_BM.c} implements scalar values
    \index{value!string} \index{string} \index{double!boxed}
    \index{value!double}
    numbers. \index{scalar-bm@\texttt{scalar\_BM.c} file} (strings,
    boxed doubles).

  \item \texttt{sequence\_BM.c} implements sets and
    tuples.\index{sequence-bm@\texttt{sequence\_BM.c} file}

  \item \texttt{user\_BM.c} relates
    \index{user-bm@\texttt{user\_BM.c} file} to reifications of
    contributors and \index{contributor} users.

  \item \texttt{misc\_BM.cc} is a \textbf{C++} file,
    \index{misc-bm@\texttt{misc\_BM.cc} file} to take advantage of
    some standard C++ \index{C++} \index{container!C++} containers.
\end{itemize}

{\large \textbf{The persistent store}}

The persistent data (see \S \ref{sec:datapersist} above) sits in files
\index{persistent!store} \index{store!persistent}
\index{store-bm@\texttt{store*BISMON.bmon} files} \texttt{store*BISMON.bmon}
(using
\href{https://man7.org/linux/man-pages/man7/glob.7.html}{\texttt{glob(7)}}
notation); more precisely

\begin{itemize}

\item \texttt{store{1}-BISMON.bmon} is for predefined objects. The header file
  \index{genbm-predef@\texttt{genbm-predef} file}
  \texttt{genbm\_predef.h} is generated from them at dump time.

\item file \texttt{store{2}-BISMON.bmon} contains the global object space. Several
  global objects describe modules whose C code is generated (e.g. at
  dump time) \index{dump} \index{module} under sub-directory
  \index{modules-dir@\texttt{modules/} directory}
  \textbf{\texttt{modules/}}\,.

\item other \texttt{store\textcolor{blue}{\textbf{\textit{i}}}-BISMON.bmon}
  textual files contain\footnote{So there cannot be any
    \texttt{store{0}-BISMON.bmon} file, since space 0 is for transient objects
    which are never dumped.}  objects in space ranked $i$ \ldots. Notice
  that all these files are both loaded and dumped, and should be
  backed-up (like the \textbf{\texttt{modules/}}\, directory)
  regularily.

\end{itemize}

These \emph{generated} textual
  \index{storei-bm@\texttt{store\textit{i}-BISMON.bmon} files}
\texttt{store\textcolor{blue}{\textit{\textbf{i}}}-BISMON.bmon} \emph{files}
should be \emph{version controlled} by the
\index{git@\texttt{git}}
\href{https://git-scm.com/}{\texttt{git}} tool. You might use the \texttt{make
  redump} command to regenerate the persistent store and the modules,
and it is advised\footnote{Once \texttt{make redump} fails, the
  persistent store is inconsistent and corrupted. This should not
  happen, but when it does, use
  \href{https://git-scm.com/docs/git-bisect}{\texttt{git bisect}} to
  find the latest consistent state of your \textit{Bismon} repository.}
to run it daily.
  
{\large \textbf{Users and contributors related files}}

The \textsc{Bismon} system does need some minimal data about
users. The reader of this report is expected to verify (perhaps with
the help of lawyers) that such data is compliant and compatible with
regulations like the European \index{GDPR} GDPR.

\begin{itemize}
\item the textual file \texttt{contributors\_BM} describes the
  known\index{contributors-bm@\texttt{contributors\_BM} file}
  contributors to the \textsc{Bismon} software. That file has comments
  (or ignored lines) starting with the \texttt{\#}
  character. Non-comment lines contain three or four fields, separated
  by semi-colons (i.e. \texttt{;}):

  \begin{enumerate}
  \item the user or contributor name, as known to the system. It could be some pseudo.
  \item the unique objid of the \textsc{Bismon} object describing that user or contributor
  \item the email of that user.
  \item an optional email alias of the same user
  \end{enumerate}


\item the textual file \texttt{passwords\_BM} associates an objid with
  some encrypted password.
  known\index{passwords-bm@\texttt{passwords\_BM} file} This is used
  for the login form of the web interface, and should not be readable
  by group or others. Lines inside this files are either comments (or
  ignored lines) starting with the \texttt{\#} character, or entry
  lines with a user name, then a semicolon, then the objid, then a
  semi-colon, then the encrypted password.

\end{itemize}

Since the main author of this draft report is known to \textsc{Bismon}
and reified in the object of objid \texttt{\_6UYrSn7piPM\_3eYhLtoXl},
the file \texttt{contributors\_BM} should at least contain a line like
perhaps\\ {\parbox{8cm}{{\relsize{-1.3}{ \texttt{Basile
      Starynkevitch;\_6UYrSn7piPM\_3eYhLtoXlmL;b.star@email.invalid;bstarynk@localhost}}}}}

It is preferable to run the \texttt{./bismon} software with specific
command lines argument to update the contributors file and the
passwords file.

\subsection{Generators and meta-programs in \texttt{bismon}}

\textcolor{brown}{\textbf{Generating code is one of the core ideas of \textsc{Bismon}}}. Such code
generation happens both at build time and at run time. The generated
code is usually some C file\footnote{With additional funding and more
time, we could have used
\href{https://gcc.gnu.org/onlinedocs/jit/}{\texttt{libgccjit}} to
generate directly some \texttt{*.so} shared object.}.

At build time, two meta-programs and some shell or GNU awk scripts are involved; each of these two metaprograms has a
single handwritten C++ source code file:

\begin{itemize}
  \item \texttt{BISMON-config}
    \index{bismon-config@\texttt{BISMON-config} metaprogram} is
    querying some parameters from the user (that is the Linux sysadmin
    installing \texttt{bismon}) and generates some C++ files.

  \item \texttt{BM\_makeconst}
    \index{bm-makeconst@\texttt{BM\_makeconst} metaprogram} is usually
    scanning some handwritten C file (for example, our
    \texttt{engine\_BM.c} file, etc...)  and producing some headers or
    utility files.

  \item \texttt{timestamp-emit.sh} is a shell script
    \index{timestamp-emit.sh@\texttt{timestamp-emit.sh} shell script}
    (using internally the \texttt{emit-git-sources.gawk} GNU gawk
    script)
    \index{emit-git-sources.gawk@\texttt{emit-git-sources.gawk}
      internal \textsc{gawk} script} which emits a simple C file
    (containing only data) with the timestamp and some metadata
    information about the build.
    
\end{itemize}

But once the \texttt{bismon}
\href{https://en.wikipedia.org/wiki/Executable_and_Linkable_Format}{ELF}
executable exists, the above metaprograms are not useful
anymore. However, they are needed to recompile \texttt{bismon} (which
you might want to do periodically, i.e. every evening).

\medskip

At run time, the \texttt{bismon} executable is routinely generating C
or C++ code. Some C code (under the \textbf{\texttt{modules/}}\,
directory) is generated to extend the behevior of \texttt{bismon}
itself : the generated C code is compiled, and the resulting shared
object is
\href{https://man7.org/linux/man-pages/man3/dlopen.3.html}{\texttt{dlopen(3)}}-ed
but never\footnote{Not \texttt{dlclose}-ing is of course some kind of
\href{https://en.wikipedia.org/wiki/Memory_leak}{memory leak}, since
the \href{https://en.wikipedia.org/wiki/Virtual_address_space}{virtual
  address space} of the process running \texttt{bismon} is never
shrinking.. This explains why the \texttt{bismon} process should be
restarted at least daily.  Our
\href{https://github.com/bstarynk/misc-basile/blob/master/manydl.c}{\texttt{manydl.c}}
program demonstrates that
\href{https://man7.org/linux/man-pages/man3/dlopen.3.html}{\texttt{dlopen(3)}}-ing
many thousands of times is practically possible on modern
\textsc{Linux} workstations.}
\href{https://man7.org/linux/man-pages/man3/dlclose.3.html}{\texttt{dlclose(3)}}-ed.

\index{file!generated} \index{generation!of files} Conventionally, we
want the generated persistent files to contain the
\texttt{\textsection{GENERATED\_PERSISTENT}\textsection} string, and
\index{generated-persistent@\texttt{\textsection{GENERATED\_PERSISTENT}\textsection}}
if possible to have a \texttt{\textsection} inside the file path.  But
generated temporary (or transient) files should contain the
\texttt{\textcurrency{GENERATED}\textcurrency} string, and if possible
\index{generated@\texttt{\textcurrency{GENERATED}\textcurrency}} have
a starting underscore (that is, a
\textcolor{blue}{\large\textbf{\texttt{\_}}} character) in their file
name.

{\textcolor{red}{\large @@TO BE COMPLETED}}

\section{Configuring \texttt{bismon} from its source code}
\label{sec:configure-bismon}

\textbf{\large Warning:} This configuration step has to be done again as soon
as your \href{http://gcc.gnu.org/}{GCC} compiler or cross-compiler has
changed versions, or when you have added new important libraries on
the \textsc{Linux} workstation running \emph{Bismon}.

First, \textbf{inspect, and \emph{improve if needed}, the \texttt{Configure} shell
script} for \texttt{/bin/bash}. \textbf{Then run that script} using the
\texttt{./Configure} command.
\index{configure@\texttt{Configure} script}
\index{configuration}

\section{Building \texttt{bismon} from its source code}
\label{sec:building-bismon}

\textbf{\large Warning:} This building step should probably be run
every evening, e.g. using a
\href{https://man7.org/linux/man-pages/man1/crontab.1.html}{\texttt{crontab}(1)}
job. It should be done only after the \texttt{./Configure} script has
been successfully run.

First, \textbf{inspect, and \emph{improve if needed}, the \texttt{Build} shell
script} for \texttt{/bin/bash}. \textbf{Then run that script} using the
\textcolor{\texttt{./Build}} command. 
\index{build@\texttt{Build} script}
\index{building!\texttt{bismon}}


\subsection{Checking the version of \texttt{bismon}}

Once the \texttt{./Build} script did work correctly, there should be
some \texttt{bismon} executable file. Check first using the
\texttt{file ./bismon} command, it should give you something similar
to:

\begin{verbatim}
./bismon: ELF 64-bit LSB shared object, x86-64, version 1 (SYSV), dynamically linked,
 interpreter /lib64/ld-linux-x86-64.so.2,
BuildID[sha1]=b4e74f225dff6115a01764abfe361a1b7757a208, for GNU/Linux 3.2.0,
with debug_info, not stripped
\end{verbatim}

Then use \texttt{ldd ./bismon} to verify that all dependencies are
present. On some computer, I am possibly getting the following output:

\begin{relsize}{-1}
\begin{verbatim}
 % ldd ./bismon
	linux-vdso.so.1 (0x00007ffee532a000)
	libonion.so.0 => /usr/local/lib/libonion.so.0 (0x00007f5ee9ec2000)
	libglib-2.0.so.0 => /lib/x86_64-linux-gnu/libglib-2.0.so.0 (0x00007f5ee9d99000)
	libjansson.so.4 => /usr/local/lib/libjansson.so.4 (0x00007f5ee9d8a000)
	libcrypt.so.1 => /lib/x86_64-linux-gnu/libcrypt.so.1 (0x00007f5ee9d4f000)
	libpthread.so.0 => /lib/x86_64-linux-gnu/libpthread.so.0 (0x00007f5ee9d2c000)
	libdl.so.2 => /lib/x86_64-linux-gnu/libdl.so.2 (0x00007f5ee9d24000)
	libstdc++.so.6 => /lib/x86_64-linux-gnu/libstdc++.so.6 (0x00007f5ee9b43000)
	libm.so.6 => /lib/x86_64-linux-gnu/libm.so.6 (0x00007f5ee99f4000)
	libgcc_s.so.1 => /lib/x86_64-linux-gnu/libgcc_s.so.1 (0x00007f5ee99d9000)
	libc.so.6 => /lib/x86_64-linux-gnu/libc.so.6 (0x00007f5ee97e7000)
	libxml2.so.2 => /lib/x86_64-linux-gnu/libxml2.so.2 (0x00007f5ee962d000)
	libpam.so.0 => /lib/x86_64-linux-gnu/libpam.so.0 (0x00007f5ee961b000)
	libgcrypt.so.20 => /lib/x86_64-linux-gnu/libgcrypt.so.20 (0x00007f5ee94fb000)
	libgnutls.so.30 => /lib/x86_64-linux-gnu/libgnutls.so.30 (0x00007f5ee9325000)
	libsqlite3.so.0 => /lib/x86_64-linux-gnu/libsqlite3.so.0 (0x00007f5ee91fc000)
	libhiredis.so.0.14 => /lib/x86_64-linux-gnu/libhiredis.so.0.14 (0x00007f5ee91e9000)
	libsystemd.so.0 => /lib/x86_64-linux-gnu/libsystemd.so.0 (0x00007f5ee913a000)
	libpcre.so.3 => /lib/x86_64-linux-gnu/libpcre.so.3 (0x00007f5ee90c7000)
	/lib64/ld-linux-x86-64.so.2 (0x00007f5eea0b9000)
	libicuuc.so.66 => /lib/x86_64-linux-gnu/libicuuc.so.66 (0x00007f5ee8edf000)
	libz.so.1 => /lib/x86_64-linux-gnu/libz.so.1 (0x00007f5ee8ec3000)
	liblzma.so.5 => /lib/x86_64-linux-gnu/liblzma.so.5 (0x00007f5ee8e9a000)
	libaudit.so.1 => /lib/x86_64-linux-gnu/libaudit.so.1 (0x00007f5ee8e6e000)
	libgpg-error.so.0 => /lib/x86_64-linux-gnu/libgpg-error.so.0 (0x00007f5ee8e4b000)
	libp11-kit.so.0 => /lib/x86_64-linux-gnu/libp11-kit.so.0 (0x00007f5ee8d13000)
	libidn2.so.0 => /lib/x86_64-linux-gnu/libidn2.so.0 (0x00007f5ee8cf2000)
	libunistring.so.2 => /lib/x86_64-linux-gnu/libunistring.so.2 (0x00007f5ee8b70000)
	libtasn1.so.6 => /lib/x86_64-linux-gnu/libtasn1.so.6 (0x00007f5ee8b5a000)
	libnettle.so.7 => /lib/x86_64-linux-gnu/libnettle.so.7 (0x00007f5ee8b20000)
	libhogweed.so.5 => /lib/x86_64-linux-gnu/libhogweed.so.5 (0x00007f5ee8ae8000)
	libgmp.so.10 => /lib/x86_64-linux-gnu/libgmp.so.10 (0x00007f5ee8a62000)
	librt.so.1 => /lib/x86_64-linux-gnu/librt.so.1 (0x00007f5ee8a57000)
	liblz4.so.1 => /lib/x86_64-linux-gnu/liblz4.so.1 (0x00007f5ee8a36000)
	libicudata.so.66 => /lib/x86_64-linux-gnu/libicudata.so.66 (0x00007f5ee6f75000)
	libcap-ng.so.0 => /lib/x86_64-linux-gnu/libcap-ng.so.0 (0x00007f5ee6f6d000)
	libffi.so.7 => /lib/x86_64-linux-gnu/libffi.so.7 (0x00007f5ee6f5f000)
\end{verbatim}
\end{relsize}

At last, you can ask \texttt{bismon} to give its version information
using \texttt{./bismon --version} which might output something like:

\begin{relsize}{-1}
\begin{verbatim}
% ./bismon --version
./bismon: version information
	 timestamp: jeu. 28 janv. 2021 13:32:53
	 git id: d62d320623ad2a7e73019e736870039508c36d02
	 last git commit: d62d320623ad tell about the Build script in end of BISMON-config.cc
	 last git tag: heads/master
	 source checksum: d929d9edebcd6a76618901c7ebbaed71
	 source dir: /home/basile/bismon-master
	 GNUmakefile: /home/basile/bismon-master/GNUmakefile
########
run
	   ./bismon --help
to get help.
\end{verbatim}
\end{relsize}

Please report the output of \texttt{./bismon --version} before any question on \textsc{Bismon}.

The output of \texttt{./bismon --help} is giving up to date
information about invoking \texttt{./bismon} (e.g. in some
\texttt{cron} script). For example:

\begin{relsize}{-1.5}
\verbatiminput{generated/005-bismon-help.tex}
\end{relsize}

\section{Dumping and restoring the \texttt{bismon} persistent heap}
\label{sec:dumping-restoring-heap}

It is absolutely essential that the \texttt{make redump} command works
well., and you need to run that command regularily.
\index{redump@\texttt{redump} target for \texttt{make}} See also \S
\ref{subsec:persistence}

It is practically important to use the periodically
\href{http://git-scm.com/}{\texttt{git} version control} system on the
repository, and this includes the persistent heap dump files
\texttt{store*BISMON.bmon} - when \texttt{bismon} is \textbf{not
  running}. It is recommended to \texttt{git commit} these files twice
a day at least, when the \texttt{bismon} executable is not running.

\end{appendices}


%%%%%%%%%%%%%%%%%%%%%%%%%%%%%%%%%%%%%%%%%%%%%%%%%%%%%%%%%%%%%%%%

\clearpage

%% https://en.wikibooks.org/wiki/LaTeX/Indexing#Adding_index_to_table_of_contents
\addcontentsline{toc}{section}{Index}

\medskip

%\section*{sec Index}

\printindex

\medskip

\clearpage
\addcontentsline{toc}{section}{References}

%\section*{sec References}

\medskip

\bibliography{bismon-biblio}

\begin{flushright}
  \begin{relsize}{-2}
    For books in French, I have provided a \emph{tentative} translation into
    English of their title in brackets.
  \end{relsize}
\end{flushright}

\medskip

\bigskip

\hrule

\bigskip

\subsubsection*{About this document}

\begin{center}
\Ovalbox{
\begin{minipage}{0.9\textwidth}
  \medskip
  \begin{quote}
    % TODO: remove the grey when HeVeA can process this
    %%%%%%%%%%%%%%%%%%%%%%%%%%
    To produce this document, both in PDF {\textcolor{Grey}{and HTML}}
    forms : build \emph{bismon}\footnote{See the
      \href{https://github.com/bstarynk/bismon/blob/master/README.md}{\texttt{README.md}}
      file on \bmurl{https://github.com/bstarynk/bismon/} for building
      instructions.} on your Linux computer, then run \texttt{make
      doc}\footnote{That uses {\LaTeX} and
      \href{http://hevea.inria.fr/}{\emph{HeVeA}}. {\textcolor{Grey}{HTML generation might not work in summer 2019.}} }  or just
    \texttt{make latexdoc} to get only its PDF form.

Feedback and improvements on this document can be suggested by email
(to \bmemail{basile@starynkevitch.net} or
\bmemail{basile.starynkevitch@cea.fr}) or by submitting patches to
\textit{Bismon} thru its \bmurl{https://github.com/bstarynk/bismon}
repository (or directly by email, with your permission to include
it). Notice that this document may contain generated documentation,
and will contain more and more generated parts in the future.
  \end{quote}
\bigskip
\end{minipage}
}
\end{center}

\bigskip

\hrule

\bigskip

\section*{Acknowledgements}

Thanks to my colleague Franck Védrine, to several members of the \textsc{Chariot} consortium who have
proofread this report, and to Niklas Rosencratz (from Sweden, outside
of the consortium) who voluntarily found mistakes in it and proposed,
in the repository on \bmurl{https://github.com/montao/bismon-docker/},
a \texttt{Dockerfile} for \emph{bismon}.

Thanks also to Jérémie Salvucci (France) and Abhishek Chakravarti
(India) for many valuable questions, suggestions, and discussions -in
numerous audioconference (or face to face) exchanges, or in private
emails-, about reflexive persistent systems in general, and more
specifically about \emph{Bismon}. Both did suggest several
improvements to this report and to the \emph{Bismon} software.

\medskip

\hrule 

\medskip

\vspace{2cm}

\bigskip

The CC-BY-SA license below for this (\textsc{Chariot}
D1.3\textsuperscript{v2}) deliverable is required to enable major parts
of this report to be later incorporated into a proper
\emph{bismon} free software documentation. For example, the Debian
Linux distribution has a policy~\footnote{See Debian documentation
  project on \bmurl{https://www.debian.org/doc/ddp} etc...} strongly
recommending a specific set of licenses (notably CC-BY-SA) for
documentation. Using other (deemed proprietary) licenses in free
software documentation is decreasing the future chances of such
documentation being later incorporated in Linux distributions.

\begin{quote}
   \bmincludewidthgraphics{75pt}{CC-BY-SA-4-img}{png}{png} \\
    This entire document
is licensed under the Creative Commons Attribution-ShareAlike 4.0
International License. To view a copy of this license, visit
\bmurl{http://creativecommons.org/licenses/by-sa/4.0/} or send a letter to
Creative Commons, PO Box 1866, Mountain View, CA 94042, USA.
\end{quote}

\bigskip

\vspace{2cm}

\bigskip

\hrule 

\medskip

\begin{flushright}
  \begin{relsize}{-1}
    (git commit \texttt{\bmgitcommit}; generated on
    \textit{\bmdoctimestamp})\\
    Some variant of this draft report is
    downloadable from \bmurl{http://starynkevitch.net/Basile/bismon-doc.pdf} and
    elsewhere.
  \end{relsize}
\end{flushright}

\end{document}
%%%%%%%%%%%%%%%%%%%%%%%%%%%%%%%%%%%%%%%%%%%%%%%%%%%%%%%%%%%%%%%%
%% Local Variables: ;;
%% compile-command: "cd ..; ./build-bismon-doc.sh LaTeX" ;;
%% End: ;;
%%%%%%%%%%%%%%%%%%%%%%%%%%%%%%%%%%%%%%%%%%%%%%%%%%%%%%%%%%%%%%%%
