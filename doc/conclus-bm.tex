% file conclus-bm.tex, which is \input from bismon-chariot-doc.tex
\section{Conclusion}
\label{sec:conclusion}

The \texttt{bismon} free software is developed in an agile and
incremental manner~\footnote{So there are no released stable versions
  of this software, but snapshots.} (required by its bootstrapping
approach), with continuous updates to
\bmurl{https://github.com/bstarynk/bismon/}.

In october 2018, the persistence machinery is working and daily used
to enhance \texttt{bismon}. The agenda mechanism is working. A naive
stop-the-world mark-and-sweep precise garbage collector is
implemented. The generation of internal C code is done (by
hand-written routines, still coded in C), this enables the
meta-programming approach. The web interface is worked upon: a
\texttt{libonion} based infrastructure is already handling HTTP
requests, and a GDPR-compliant login form is presented on web
browsers. Our \texttt{jsweb\_module} contains the functions related to
Javascript (nearly complete) and HTML generation (work in progress). The
syntactical editor (replacing the crude GTK interface) and then the
GCC plugin generation should be worked on.

In august 2019, the web machinery is mostly working. More generated C
code is available. The JSON handling is incomplete. Bismon
continuations \index{contination!reification} \index{reification!of
  continuations} are almost\footnote{Thanks to generated invocations of
  the \texttt{LOCALFRAME\_BM}
  \index{LOCALFRAME\_BM@\texttt{LOCALFRAME\_BM}} \emph{C} macro, which
  provides 90\% of the development work: full transient reification of
  partial continuations, that is of \index{call stack} \index{call
    frame} call stack segments, is just a matter of clustering emitted
  stack-local \index{stackframe\_stBM@\texttt{stackframe\_stBM}}
  \texttt{struct stackframe\_stBM}-based linked-lists of Bismon call
  frames.}  reifiable into transient \index{transient!object for
  continuations} objects, having as payload a linked-list sequence of
\index{call frame} call frames.

The final D1.3~\textsuperscript{v2} version (scheduled for M30) of
this deliverable will explain the Web interfaces (both for the
ordinary user, i.e. the IoT developer; and for the static analysis
expert) and the generation of C++ code for GCC plugins, with some
examples of simple, IoT focused, whole-program static source code
analysis performed by \emph{bismon}. So the final
D1.3~\textsuperscript{v2} document will contain a longer conclusion.

Machine learning techniques inspired by \cite{zhang:2019:learned}
could be relevant in \textsc{Bismon}. See also the
\href{http://refpersys.org/}{\textsc{RefPerSys}}
research\footnote{Also related:
\href{https://afia.asso.fr/journee-hommage-j-pitrat/}{talks in the
  memory of J.Pitrat, AFIA, March 6\textsuperscript{th}, 2020,
  Paris.}} project.

\medskip

