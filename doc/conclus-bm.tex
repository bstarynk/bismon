% file conclus-bm.tex, which is \input from bismon-doc.tex
\section{Conclusion}

The \texttt{bismon} free software is developed in an agile and
incremental manner~\footnote{So there are no released stable versions
  of this software, but snapshots.} (required by its bootstrapping
approach), with continuous updates to
\bmurl{https://github.com/bstarynk/bismon/}.

In october 2018, the persistence machinery is working and daily used
to enhance \texttt{bismon}. The agenda mechanism is working. A naive
stop-the-world mark-and-sweep precise garbage collector is
implemented. The generation of internal C code is done (by
hand-written routines, still coded in C), this enables the
meta-programming approach. The web interface is worked upon: a
\texttt{libonion} based infrastructure is already handling HTTP
requests, and a GDPR-compliant login form is presented on web
browsers. Our \texttt{jsweb\_module} contains the functions related to
Javascript (nearly complete) and HTML generation (incomplete). The
syntactical editor (replacing the crude GTK interface) and the GCC
plugin generation should be worked on.

@@To be completed

\medskip


\begin{boxedminipage}{0.6\textwidth}

To build this document (both in PDF and HTML forms) : build
\emph{bismon} \footnote{See the \emph{Readme} on
  \bmurl{https://github.com/bstarynk/bismon/} for building
  instructions} on your Linux workstatation, then run \texttt{make
  doc} (that uses {\LaTeX} and
\href{http://hevea.inria.fr/}{\emph{HeVeA}}).

Feedback and improvements on this document can be suggested by email
(to \bmemail{basile@starynkevitch.net} or
\bmemail{basile.starynkevitch@cea.fr}) or by submitting patches to
\textit{Bismon} thru its \bmurl{https://github.com/bstarynk/bismon} repository
(or by email). Notice that this document may contain generated
documentation, and will contain more and more generated parts in the
future.
\end{boxedminipage}
