% file intro-bm.tex
\section{Introduction}

\subsection{About \textit{Bismon}}

\textit{Bismon} (a \emph{temporary name}) \footnote{\texttt{bismon} is a \textbf{temporary name}
  and could be changed, once we find a better name for it} is a free
software (GPLv3+ licensed)\footnote{The source code is unreleased but available, and continuously evolving, on \bmurl{github.com/bstarynk/bismon}} static source code whole-program analysis framework whose
initial domain will be \emph{Internet of Things} (or
\index{IoT}{IoT})\footnote{IoT is viewed as the first application
  domain of \textit{Bismon}, but it is hoped that most of
  \textit{Bismon} could be reused and later extended for other
  domains}. It is designed to work with the \textit{Gcc} compiler (see
\bmurl{gcc.gnu.org}) on a Linux workstation\footnote{Linux specific features are needed by \textit{Bismon}, which is unlikely to be buildable or run under other operating systems. My Linux distribution is \emph{Debian/Unstable}}. \textit{Bismon} is the
successor of \textit{GCC MELT} \footnote{The \textit{GCC MELT} web
  pages used to be on \texttt{gcc-melt.org} -a DNS domain relinquished
  in april 2018- and are archived on
  \bmurl{starynkevitch.net/Basile/gcc-melt}} (see \cite{Starynkevitch2007Multistage, starynkevitch-DSL2011}).

{\large \textit{Bismon} is \textbf{work in progress}}, and many things
described here are not (fully, or even partly!) implemented in 2018 or
could drastically change later. In particular, \textit{Bismon} is not
yet usable as a static source code analyser in mid-2018 (since work on
the infrastructure is not complete).

\bigskip

\subsection{Lessons learned from \textit{GCC MELT}}

@@ To be written

\subsection{Driving principles for  \textit{Bismon}}

\textit{Bismon} is (like \textit{GCC MELT} was) a \textbf{domain
  specific language} implementation, targetted to ease static source
code analysis (above the \textit{GCC} compiler), with the following
features:

\begin{itemize}
  \item \textbf{persistency}, somehow \textit{orthogonal
    persistency}. So most of the data handled by \textit{Bismon} can
    be persisted on disk, and reloaded at the next run. Some data is
    temporary by nature and should not be persisted. Such data is
    called temporary or \textbf{transient}. But the usual approach is
    to run the \textit{Bismon} program from some initial loaded state
    and have it dump its memory state on disk before exiting (and
    reload that augmented state at the next run).

  \item \textbf{dynamic typing}, like many scripting languages (such
    as Guile, Python, Lua, etc)

  \item \textbf{homoiconicity} and \textbf{reflection} with
    \textbf{introspection}: all the DSL code is explicitly represented
    as data which can be processed by \textit{Bismon}, and the current
    state is accessible by the DSL.

    \item \textbf{translated} to \emph{C} code (and generated
      \emph{JavaScript} + \emph{HTML} in the browser, and generated
      \emph{C++} code of \emph{GCC} plugins)

    \item \textbf{bootstrapped implementation}: ideally, all of
      \textit{Bismon} code (including C code dealing with data
      representations, persistent store, etc...) should be
      generated. However, this ideal has not yet be attained, and
      there is still some hand-written C code. It is hoped that most
      of the hand-written C code will become replaced by generated C
      code.
    \item ability to \textbf{generate GCC plugins}: the C++ code of
      GCC plugins performing static analysis of a single translation
      unit should be generated.
\end{itemize}


