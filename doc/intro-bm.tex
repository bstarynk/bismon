% file intro-bm.tex
\section{Introduction}


\subsection{Expected audience}

The reader of this document is expected to:
\begin{itemize}

  \item be fluent in C (cf. \cite{Kernighan:1988:CPL}) and C++
    (\cite{Stroustrup:2014:CplusPlus}) programming (notably on Linux
    and/or for embedded products),

  \item knowing well the \index{C11}{C11} standard
    (cf. \cite{C11:std,Memarian:2016:PLDI}) and/or the \index{C++11}{C++11} one
    (\cite{CplusPlus11:std}) and understanding well the notion of
    \index{undefined behavior}{\emph{undefined behavior}} \footnote{See
      \bmurl{http://blog.llvm.org/2011/05/what-every-c-programmer-should-know.html}
      and \bmurl{https://blog.regehr.org/archives/1520}} in C or C++
    programs,

  \item be a daily advanced user of \index{Linux}{Linux} for software development
    activities using GCC and related developer tools
    (e.g. \textit{binutils}, \texttt{git}, \texttt{make} or
    \texttt{ninja}, source code editor like \texttt{emacs} or
    \texttt{vim}, the {\LaTeX} text formatter\footnote{See
      \bmurl{https://www.latex-project.org/}}) on the \emph{command
      line}.


\item to be capable of building large free software projects (such as
  the GCC compiler (cf \cite{gcc-internals} \footnote{See
    \bmurl{http://gcc.gnu.org}}), the Linux kernel, the Qt toolkit and
  other open source projects of perhaps millions of source code lines)
  and smaller ones (e.g. \texttt{libonion} \footnote{see
    \bmurl{https://coralbits.com/libonion/}}) from their \emph{source}
  form.

\item to have contributed or participated to some free software or
  open source projects (and understanding their social and economical
  (cf \cite{Weber:2004:SuccessOpenSource}) implications, the practical work
  flow, the importance of developer communities)
  
\item should be interested in static source code analysis (so have
  already tried some such tools like \emph{Frama-C} \footnote{See
    \bmurl{http://frama-c.com/}} \cite{Cuoq:2012:Frama-C}, \emph{Clang
    analyzer} \footnote{See \bmurl{https://clang-analyzer.llvm.org/}},
  ...), and be aware of compiler concepts and technologies \cite{Aho:2006:DragonBook}.

\item be familiar with operating systems principles
  \cite{Tanenbaum:92:OS,ArpaciDusseau14-Book} and well
  versed in Linux programming
  \cite{Mitchell:2001:ALP,Kerrisk:2010:LinuxProgramming} \footnote{look
    into \texttt{man} pages on
    \bmurl{http://man7.org/linux/man-pages/}}.

  \item be interested in various programming languages
    \cite{Abelson1996:SICP,Scott:2007:PLP,Queinnec:1996:LSP},
    including domain specific ones.

\end{itemize}


To build this document (both in PDF and HTML forms) : build
\emph{bismon} \footnote{See the \emph{Readme} on
  \bmurl{https://github.com/bstarynk/bismon/} for building
  instructions}, then run \texttt{make doc} (that uses {\LaTeX} and
\emph{HeVeA}).

Feedback and improvements on this document can be suggested by email
(to \bmemail{basile@starynkevitch.net} or
\bmemail{basile.starynkevitch@cea.fr}) or by submitting patches to
\textit{Bismon} thru its \bmurl{github.com/bstarynk/bismon} repository
(or by email). Notice that this document may contain generated
documentation, and will contain more and more generated parts in the
future.

\subsection{About \textit{Bismon}}

\textit{Bismon} (a \emph{temporary name}) \footnote{\texttt{bismon} is a \textbf{temporary name}
  and could be changed, once we find a better name for it} is a free
software (GPLv3+ licensed)\footnote{The source code is unreleased but available, and continuously evolving, on \bmurl{github.com/bstarynk/bismon}} static source code whole-program analysis framework whose
initial domain will be \emph{Internet of Things} (or
\index{IoT}{IoT})\footnote{IoT is viewed as the first application
  domain of \textit{Bismon}, but it is hoped that most of
  \textit{Bismon} could be reused and later extended for other
  domains}. It is designed to work with the \textit{Gcc} compiler (see
\bmurl{gcc.gnu.org}) on a Linux workstation\footnote{Linux specific features are needed by \textit{Bismon}, which is unlikely to be buildable or run under other operating systems. My Linux distribution is \emph{Debian/Unstable}}. \textit{Bismon} is the
successor of \textit{GCC MELT} \footnote{The \textit{GCC MELT} web
  pages used to be on \texttt{gcc-melt.org} -a DNS domain relinquished
  in april 2018- and are archived on
  \bmurl{starynkevitch.net/Basile/gcc-melt}} (see \cite{Starynkevitch2007Multistage, starynkevitch-DSL2011}).

{\large \textit{Bismon} is \textbf{work in progress}}, and many things
described here are not (fully, or even partly!) implemented in 2018 or
could drastically change later. In particular, \textit{Bismon} is not
yet usable as a static source code analyser in mid-2018 (since work on
the infrastructure is not complete).

\bigskip

\subsection{Lessons learned from \textit{GCC MELT}}

@@ To be written

\subsection{Driving principles for  \textit{Bismon}}

\textit{Bismon} is (like \textit{GCC MELT} was) a \textbf{domain
  specific language} implementation, targetted to ease static source
code analysis (above the \textit{GCC} compiler), with the following
features:

\begin{itemize}

  \item \textbf{persistency}, somehow \textit{orthogonal
    persistency}. It is needed in particular because compiling some
    software project (analyzed by \textit{Bismon}) is a long-lasting
    procedure involving \textit{several} compiling and linking
    processes and commands. So most of the data handled by
    \textit{Bismon} can be persisted on disk, and reloaded at the next
    run. Some data is temporary by nature and should not be
    persisted. Such data is called temporary or
    \textbf{transient}. But the usual approach is to run the
    \textit{Bismon} program from some initial loaded state and have it
    dump its memory state on disk before exiting (and reload that
    augmented state at the next run).

  \item \textbf{dynamic typing}, like many scripting languages (such
    as Guile, Python, Lua, etc). Of course the dynamically typed data
    inside the \textit{Bismon monitor} is \index{garbage collection}{\textbf{garbage collected}} (cf. \cite{Jones:2011:GC-handbook})

  \item \textbf{homoiconicity} and \textbf{reflection} with
    \textbf{introspection}: all the DSL code is explicitly represented
    as data which can be processed by \textit{Bismon}, and the current
    state is accessible by the DSL.

    \item \textbf{translated} to \emph{C} code; and \textbf{generated}
      \emph{JavaScript} + \emph{HTML} in the browser, and generated
      \emph{C++} code of \emph{GCC} plugins

    \item \textbf{bootstrapped implementation}: ideally, all of
      \textit{Bismon} code (including C code dealing with data
      representations, persistent store, etc...) should be
      generated. However, this ideal has not yet be attained, and
      there is still some hand-written C code. It is hoped that most
      of the hand-written C code will become replaced by generated C
      code.
    \item ability to \textbf{generate GCC plugins}: the C++ code of
      GCC plugins performing static analysis of a single translation
      unit should be generated.

    \item with \textbf{collaborative web interface}, used by a
      \emph{small} team of \emph{trusted and well-behaving}
      developers\footnote{The initial \texttt{bismon} implementation
        had a GTK interface, nearly unusable. That interface is
        temporarily used to fill the persistent store till the web
        interface is usable. The GTK interface is already obsolete and
        should disappear at end of 2018}. The users of \emph{bismon}
      are expected to trust each other, and to use the \texttt{bismon}
      tool responsibly\footnote{For example, each \emph{bismon} user
        has the technical ability to erase most of its data, but is
        expected to not do so} (likewise, developers accessing a
      \texttt{git} version control repository are supposed to act
      responsibly). So protection against malicious behavior of
      \texttt{bismon} users is out of scope.

      Since \textit{Bismon} should be usable by a small team of
      developers (perhaps two or a dozen of them)\footnote{So
        \textit{Bismon}, considered as a Web application, would have
        at most a dozen of browsers -and associated users- accessing
        it. Hence, scalability to many HTTP connections is not at all
        a concern (in contrast with most usual web applications).}, it
      is handling some \index{personal data}{personal data} (relevant
      to \index{GDPR}{GDPR}), such as the first and last names (or
      pseudos) of users and their email and maintain a password file
      (used in the Web \index{login}{login} form). Compliance to
      regulations (e.g. European GDPR) is out of scope and should be
      in charge of the entities and/or persons using and/or deploying
      \textit{Bismon}. The login form template \footnote{on
        \bmurl{https://github.com/bstarynk/bismon/blob/master/login\_ONIONBM.thtml}}
      could be adapted on each deployment site.
\end{itemize}

Notice that \textit{Bismon} is \textbf{provided as free software}
(unreleased in 2018, but available on
\bmurl{github.com/bstarynk/bismon} ...) in \emph{source form only} and
should be \textbf{usable} (only) \textbf{on a Linux/x86-64
  workstation}... (typically, at least 32 gigabytes of RAM and
preferably more, at least 8 cores, several hundreds gigabytes of disk
or SSD).

The \textit{Bismon} system contains \textbf{persistent data} (which is
part of the system itself and should not be considered as ``external''
data; each team using \textit{Bismon} would run its own customized
version of it.), and should be \textbf{regularily backed up}, and
preferably version controlled at the user site. It is strongly
recommended to use \index{git}{\texttt{git}} \footnote{See
  \bmurl{http://git-scm.com/}} or perhaps some other distributed
\index{version control}{version control} system, to \texttt{git
  commit} its files several times a day (probably hourly or even more
frequently, as often as a developer is committing his C++ code), and
to backup the files on some external media or server at least
daily. How that is done is outside of the scope of this document. The
\emph{dump facilities} inside \textit{Bismon} are expected to be used
quite often (as often as you would save your report in a word
processor, or your source file in a source code editor), probably
several times per hour. So a developer team using \textit{Bismon}
would probably \texttt{git clone} either
\texttt{git@github.com:bstarynk/bismon.git} thru SSH or
\bmurl{https://github.com/bstarynk/bismon.git}, build it (after
downloading and building required dependencies), and work on that
\texttt{git} repository (and of course back-up it quite often).


%%%%%%%%%%%%%%%%%%%%%%%%%%%%%%%%%%%%%%%%%%%%%%%%%%%%%%%%%%%%%%%%
\subsection{Multi-threaded and distributed aspects of \textit{Bismon}}

The \textit{Bismon monitor} is by itself a multi-threaded
process\footnote{In contrast of most scripting language
  implementations, we try hard to avoid any ``global interpreter
  lock'' and strive to develop a genuinely multi-threaded monitor.}.
It uses a \emph{fixed thread pool} of worker threads (often active)
\footnote{The number of worker threads is given by the \texttt{--job}
  program argument to \texttt{bismon}. For an 8-cores workstation, it
  is suggested to set it to 5 or 6. It should be at least 2, and at
  most 15. This number also limits the set of simultaneously running
  external processes.}, and additional (generally idle) threads for
web support and other facilities. The \textit{Bismon monitor} is
occasionally starting some external processes (e.g. for the
compilation of \emph{C} code dynamically generated by \textit{Bismon},
before \texttt{dlopen}-ing such modules).

%In practice the Bismon
%monitor is likely to be started every working day and would run the
%whole day (in particular because the set of \texttt{dlopen}-ed modules
%is continuously growing and never shrinking; in other words, code
%generated by \text{Bismon} for its internal purposes is \emph{never
%  \index{garbage collection}{garbage collected}}).
%

