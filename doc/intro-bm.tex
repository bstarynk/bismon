% file intro-bm.tex
\section{Introduction}

\subsection{About \textit{Bismon}}

\textit{Bismon} (a \emph{temporary name}) \footnote{\texttt{bismon} is a \textbf{temporary name}
  and could be changed, once we find a better name for it} is a free
software (GPLv3+ licensed)\footnote{The source code is unreleased but available, and continuously evolving, on \bmurl{github.com/bstarynk/bismon}} static source code whole-program analysis framework whose
initial domain will be \emph{Internet of Things} (or
\index{IoT}{IoT})\footnote{IoT is viewed as the first application
  domain of \textit{Bismon}, but it is hoped that most of
  \textit{Bismon} could be reused and later extended for other
  domains}. It is designed to work with the \textit{Gcc} compiler (see
\bmurl{gcc.gnu.org}) on a Linux workstation\footnote{Linux specific features are needed by \textit{Bismon}, which is unlikely to be buildable or run under other operating systems. My Linux distribution is \emph{Debian/Unstable}}. \textit{Bismon} is the
successor of \textit{GCC MELT} \footnote{The \textit{GCC MELT} web
  pages used to be on \texttt{gcc-melt.org} -a DNS domain relinquished
  in april 2018- and are archived on
  \bmurl{starynkevitch.net/Basile/gcc-melt}} (see \cite{Starynkevitch2007Multistage, starynkevitch-DSL2011}).

{\large \textit{Bismon} is \textbf{work in progress}}, and many things
described here are not (fully, or even partly!) implemented in 2018 or
could drastically change later. In particular, \textit{Bismon} is not
yet usable as a static source code analyser in mid-2018 (since work on
the infrastructure is not complete).

\bigskip



