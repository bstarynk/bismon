% file intro-bm.tex, which is \input from bismon-doc.tex
\section{Introduction}

This D1.3v1 \textsc{Chariot} deliverable is a first \textbf{draft}
-and \emph{incomplete}- version of D1.3v2 scheduled at M30 on
``Specialized Static Analysis tools for more secure and safer IoT
software development''. It targets software engineering (and
indirectly also software architects) experts working on IoT software
coded in C or C++.

\subsection{Mapping \textsc{Chariot} output}
\label{subsec:mapchariot}

We refer to \textsc{Chariot} \emph{Grant Agreement} (GA). See table \ref{tab:mapchariot}


\begin{table}
  \caption{\label{tab:mapchariot} adherence to CHARIOT's GA deliverable and tasks descriptions}
  \medskip
  \begin{relsize}{-1.3}
    \begin{tabular}{|p{0.19\textwidth}|p{0.36\textwidth}|p{0.13\textwidth}|p{0.21\textwidth}|}
      \hline \textbf{\textsc{Chariot} GA component title} &
      \textbf{\textsc{Chariot} GA component outline} &
      \textbf{respective document chapter[s]} & \textbf{justification}
      \\ \hline \multicolumn{4}{|c|}{{\large \textbf{deliverable}}}
      \\
      \hline
%
      \begin{minipage}[t]{0.18\textwidth}
        \smallskip
        
        \textbf{{\relsize{+1.5}{D1.3}}} \\
        Specialized Static Analysis tools for
          more secure and safer IoT software development.
      \end{minipage}
              &
      % 
      {The source code (top level documentation) of the prototype static
      analysis tools developed in task T1.3, including the definition
      of data formats and protocols, updates and adapation of existing
      libraries and software components, the persistent monitor
      outline and documentation, anf features description and
      documentation of the compiler and linker extensions. An initial
      version set (V1) will be compiled by M12 followed by a revised
      version set (v2) in M30.}
      & {this whole document} 
      & {a single deliverable (with two versions of it) describes the work.} \\
      \hline
      \multicolumn{4}{|c|}{{\large \textbf{tasks}}} \\
      \hline
      \multirow{4}{*}{\begin{minipage}{0.19\textwidth}\smallskip
          \textbf{\relsize{+1.5}{T1.3}}          
       Specialized Static Analysis tools for
       more secure and safer IoT software development.
        \end{minipage}
      }
      & \textbf{ST1.3.1} definition of data formats and protocols
      & §\ref{sec:datapersist}; §\ref{subsec:compilinext} \& §\ref{}
      & The persistent monitor data and format are described in §\ref{sec:datapersist}.
      The protocol with \textsc{Chariot}'s blockchain is related to §\ref{subsec:compilinkext} and chapter 6 of D1.2 \\
      \cline{2-4}
      & \textbf{ST1.3.2} significant patches to existing free software components
      & §\ref{subsec:contribfree}
      & Section §\ref{subsec:contribfree} describe past work, and why future contributions to \emph{GCC} could be needed. \\
      \cline{2-4}
      & \textbf{ST1.3.3} design and implementation of the persistent monitor & HH & II \\
      \cline{2-4}
      & \textbf{ST1.3.4} design and implementation of the compiler and linker extension & §\ref{subsec:compilinext} & LL \\
      \hline
%-      \multirow{4}{*}{%
%-     \begin{minipage}[t]{0.18\textwidth}
%-       \smallskip
%-       
%-       \textbf{{\relsize{+1.5}{T1.3}}} \\
%-       Specialized Static Analysis tools for
%-       more secure and safer IoT software development.
%-
%-       \smallskip
%-      \end{minipage} &
%-      \emph{ST1.3.1} definition of the data format and protocols & §XXX & tobewrittenA \\
%-      \hline
%-      \emph{ST1.3.2} significant patches to existing free software components & §YYYY & tobewrittenB \\
%-      \hline
%-      \emph{ST1.3.3} design and implementation of the persistent monitor & §ZZZZ & hasbeenwritten \\
%-      \hline
%-      \emph{ST1.3.4} design and implementation of the compiler and linker extension & §TTTT & tobewrittenB \\
%-      \hline
       
      
    \end{tabular}
 \end{relsize}
\end{table}

\subsection{Expected audience}
\label{subsec:audience}

The reader of this document (within \textsc{Chariot}, a software
engineering expert working on IoT software or firmware coded in C or
C++) is expected to:

\begin{itemize}

  \item be fluent in C (cf. \cite{Kernighan:1988:CPL}) and/or C++
    (\cite{Stroustrup:2014:CplusPlus}) programming (notably on Linux
    and/or for embedded products, perhaps for IoT),

  \item be knowing a bit the \index{C11}{C11} standard
    (cf. \cite{C11:std,Memarian:2016:PLDI}) and/or the
    \index{C++11}{C++11} one (\cite{CplusPlus11:std}) and
    understanding well the notion of \index{undefined
      behavior}{\emph{undefined behavior}} \footnote{See
      \bmurl{http://blog.llvm.org/2011/05/what-every-c-programmer-should-know.html}
      and \bmurl{https://blog.regehr.org/archives/1520}} in C or C++
    programs,

  \item be a daily advanced user of \index{Linux}{Linux} for software
    development activities using GCC and related developer tools
    (e.g. \textit{binutils}, version control like \texttt{git}, build
    automation like \texttt{make} or \texttt{ninja}, source code
    editor like \texttt{emacs} or \texttt{vim}, the {\LaTeX} text
    formatter~\footnote{See
      \bmurl{https://www.latex-project.org/}. Some knowledge of
            {\LaTeX} is useful to improve or contribute to this
            document.}) on the \emph{command line}.
    
    \item be \emph{easily} able, in principle, to
      \textbf{compile}~\footnote{When compiling IoT software such as
        firmware, it usually is of course some
        \emph{cross}-compilation.}  his/her or other software coded in
      C (or in C++) \textbf{on the command line} (\emph{without} any
      \index{IDE!integrated development environment}IDE
      -\index{integrated development environment}~integrated software
      environment- or \index{SDK!software development kit} SDK
      -\index{software development kit}~software development kit-)
      with a \textbf{\emph{sequence} of \texttt{gcc} (or \texttt{g++})
        commands}~\footnote{In practice, we all use some \emph{build
          automation} tool, such as \texttt{make}, \texttt{ninja} or
        generators for them such as \texttt{cmake}, \emph{autoconf},
        \texttt{meson}, etc... But the reader is expected to be able
        to configure that, e.g. to add more options to \texttt{gcc} or
        to \texttt{g++} (perhaps in his/her \texttt{Makefile}) and is
        able to think in terms of a sequence of elementary
        \texttt{gcc} or \texttt{g++} compilation commands (or, when
        using Clang, \texttt{clang} or \texttt{clang++} commands).}
      \textbf{on Linux}.



\item to be capable of building large free software projects (such as
  the GCC compiler (cf \cite{gcc-internals} \footnote{See
    \bmurl{http://gcc.gnu.org}}), the Linux kernel, the Qt toolkit and
  other open source projects of perhaps millions of source code lines)
  and smaller ones (e.g. \texttt{libonion} \footnote{see
    \bmurl{https://coralbits.com/libonion/}}) from their \emph{source}
  form.

\item have successfully downloaded and built the \emph{Bismon
  monitor} \index{Bismon} from its source code on
  \bmurl{https://github.com/bstarynk/bismon}, on his Linux workstation.
  
\item have contributed or participated to some free software or open
  source projects and understanding their social (cf
  \cite{Raymond:2001:CathBaz}) and economical (cf
  \cite{Weber:2004:SuccessOpenSource, Tirole:2016:EcoBienCommun,
    Nagle:2018:Contributing}) implications, the practical work flow,
  the importance of developer communities.
  
\item be interested in static source code analysis, so have already
  tried some such tools like \emph{Frama-C} \footnote{See
    \bmurl{http://frama-c.com/}} (cf. \cite{Cuoq:2012:Frama-C}),
  \emph{Clang analyzer} \footnote{See
    \bmurl{https://clang-analyzer.llvm.org/}}, ..., and be aware of
  \index{compiler} compiler concepts and technologies (read
  \cite{Aho:2006:DragonBook}).

\item be familiar with operating systems principles
  (see \cite{Tanenbaum:92:OS,ArpaciDusseau14-Book}) and well
  versed in Linux programming
  (cf. \cite{Mitchell:2001:ALP,Kerrisk:2010:LinuxProgramming} \footnote{look
    into \texttt{man} pages on
    \bmurl{http://man7.org/linux/man-pages/}}).

  \item be interested in various programming languages
    (cf. \cite{Abelson1996:SICP,Scott:2007:PLP,Queinnec:1996:LSP}),
    including domain specific ones.

\end{itemize}



The numerous footnotes in this report are for a second reading (and
may be used for forward references). To understand this report
describing a circular and reflexive system, you should read it twice
(skipping footnotes at the first read).

\bigskip

\subsection{The \textsc{Chariot} vision on specialized static source code analysis for more secure and safer IoT software development}

\subsubsection{About static source code analysis and IoT}

There are many existing documents related to improving safety and
security in IoT software (e.g. \cite{Chen:2011:DAS, Medwed:2016:ISC}),
and even more on static source code analysis in general
(cf. \cite{Gomes2009AnOO, GosevaPopstojanova2015OnTC,
  Binkley:2007:SCA} and many others). %

Several conferences are dedicated to static analysis~\footnote{The
  25\textsuperscript{th} Static Analysis Symposium will happen in
  august 2018, see
  \bmurl{http://staticanalysis.org/sas2018/sas2018.html}; most ACM
  SIGPLAN conferences such as POPL, PLDI, ICFP, OOPSLA, LCTES, SPLASH,
  DSL, CGO, SLE... have papers related to static source code
  analysis.}.  All dominant C compilers (notably GCC and Clang, but
also MicroSoft's \emph{Visual C}\texttrademark) are using complex
static source code analysis techniques for optimizations and warnings
purposes (and that is why C compilers are monsters~\footnote{see
  \bmurl{https://softwareengineering.stackexchange.com/a/273711/40065}
  for more.}). It is wisely noticed (in
\cite{GosevaPopstojanova2015OnTC}) that \textbf{state-of-the-art
  static source code analysis tools are \emph{not} very effective in
  detecting \index{vulnerability!security} security
  vulnerabilities}~\footnote{Se we can only hope an \emph{incremental}
  progress in that area. Static source code analysis in
  \textsc{Chariot} won't make miracles.}, so they are not a ``silver
bullet'' \index{silver bullet} (\cite{Brooks:1987:NSB}). Many
taxonomies of software defects \index{defect!software} already exist
(e.g. \cite{Silva:2016:SES, Wagner:2008:DCD, Levine:2009:DDE}
etc....), notably for IoT\index{IoT} (see \cite{Carpent:2018:RRA,
  Ahmad:2018:ModelBasedIoT, Laszlo:2018:Vessedia}).

Static source code analysis tools can -grossly speaking-
be~\footnote{This is a gross simplification! In practice, there is a
  continuous spectrum of source code analyzers, much like there is a
  spectrum between compilers and interpreters (with e.g. bytecode or
  JIT implementations sitting in between compilation and naive
  interpretation).} viewed as being of either one of two kinds:

\begin{itemize}
  \item strongly formal methods based, semantically oriented,
    ``sound'' tools (e.g. built above abstract interpretation -cf.
    \cite{Cousot:2014:AIP,CousotCousot77-1}-, model checking -cf.
    \cite{Schlich:2010:MCS} and \cite{Jhala:2009:SMC}-, and other
    formal techniques on the whole program... See also
    \cite{Andreasen:2017:SAI}) which can give excellent results but
    require a lot of expertise to be used, and may take a long time to
    run~\footnote{There are cases where those static analyzers need
      weeks of computer time to give interesting results.}. For
    examples, \emph{Frama-C} (cf \cite{Cuoq:2012:Frama-C}),
    \emph{Astrée} (cf \cite{Mine:2015:TIU}), etc... The expert user
    needs either to explicitly annotate the analyzed source code
    (e.g. in ACSL for \emph{Frama-C}, see \cite{Baudin:2018:ACSL,
      Delahaye:2013:CSL, Amin:2017:LAW}), and/or cleverly tune the
    many configuration knobs of the static analyzer, and often
    both. Often, the static analyzer itself has to be extended to be
    able to give interesting results on one particular analyzed source
    code~\footnote{The \emph{Astrée} project can be seen as the
      development of a powerful analyzer tool \emph{specifically}
      suited for the needs of Airbus control command software; it
      implements many complex abstract interpretation lattices wisely
      chosen to fit the relevant traits of the analyzed code. Neither
      \emph{Astrée} nor \emph{Frama-C} can easily -without any
      additional tuning or annotations- and successfully be used on
      most Linux command line utilities (e.g. \texttt{bash},
      \texttt{\emph{coreutils}}, \texttt{\emph{binutils}},
      \texttt{gcc}, ...)  or servers (e.g. \texttt{systemd},
      \texttt{lighttpd}, \texttt{Wayland} or \texttt{Xorg}, or IoT
      frameworks such as MQTT...). But \emph{Frama-C} can be extended
      by additional plugins so is a \emph{framework} for sound static
      analysis.}. Many formal static analyzers
    (e.g. \cite{Greenaway:2014:DSS, Vedala:2012:ADP}) focus on
    checking just \emph{one} aspect or property of security or
    safety. Usually, formal and sound static analyzers can practically
    cope only with small sized analyzed programs of at most one or a
    few hundred thousands lines of C code (following some
    \emph{particular} coding style or using some definable
    \emph{subset} of the C language~\footnote{For instance, both
      \emph{Frama-C} and \emph{Astrée} have issues in dealing with
      standard dynamic C memory allocation above \texttt{malloc};
      since they target above all the safety critical real-time
      embedded software market where such allocations are
      \emph{forbidden}.}).  In practice, the formal sound static
    analyzers are able to prove \emph{automatically} some
    \emph{simple} properties of small, highly critical, software
    components (e.g. avoiding the need of \emph{unit testing} at the
    expense of very \emph{costly software development efforts}).

  \item lightweight ``syntax'' oriented ``unsound'' tools, such as
    Coverity Scan~\footnote{See \bmurl{https://scan.coverity.com/}} or
    Clang-Analyzer, or even recent compilers (GCC or Clang) with all
    warnings and link-time optimization~\footnote{Link-time
      optimization (e.g. compiling and \emph{linking} with \texttt{gcc
        -O2 -flto -Wall -Wextra} using GCC) slows down the build time by more than a
      factor of two since the intermediate internal representation (IR) of the
      compiler (e.g. Gimple \index{Gimple} for \emph{GCC}, see \cite{gcc-internals} §12) is kept in object files and
      reload at ``link-time'' which is done by the \emph{compiler}
      working on the whole program's IR, so is rarely used.}  enabled. Of
    course, these simpler approaches give many false positive warnings
    (cf \cite{Nadeem:2012:HFP}), but machine learning techniques (cf
    \cite{Perl:2015:VFP}) using bug databases could help.
    
\end{itemize}
Most static analyzers require some kind of interaction with their user
(cf \cite{Lipford:2014:ICA}), in particular to present partial
analysis results and explanations about them, or complex information
like control flow graphs, derived properties at each statements.

The \href{http://vessedia.eu/}{\textsc{Vessedia}} \index{Vessedia} project is an H2020
IoT-related project~\footnote{The \textsc{Vessedia} project (Verification
  Engineering of Safety and Security Critical Industrial Applications)
  has received funding from the European Union's Horizon 2020 research
  and innovation programme under grant agreement No. 731453, within
  the call H2020-DS-LEIT-2016, started in january 2017, ending in
  december 2019.} which is focusing on a strong formal methods
approach for IoT \emph{software} and insists on a ``single system
formal verification'' approach (so it makes quite weak hypothesis on
the ``systems of systems'' view); it ``aims at enhancing safety and
security of information and communication technology (ICT) and
especially the Internet of Things (IoT). More precisely the aim of
[the VESSEDIA] project consists in making formal methods more
accessible for application domains that want to improve the security
and reliability of their software applications by means of Formal
Methods''~\footnote{Taken from
  \bmurl{https://vessedia.eu/about}}. Most \textsc{Vessedia} partners
(even the industrial ones) are versed in formal static analysis
techniques (many of them being already trained to use \emph{Frama-C}
several years before, and several of them contributing actively to
that tool.). Some of the major achievements of \textsc{Vessedia}
includes formal (but \emph{fully automatic}) proofs of often
\emph{simple} (and sometimes very complex and very specific)
properties of some basic software components (e.g. lack of undefined
behavior in the memory allocator, or the linked list implementation,
of Contiki). Some automatically proven properties can be very complex,
and the very hard work~\footnote{In terms of human efforts, the
  formalization of the problem, and the annotation of the code and
  guidance of the prover, takes much more time and money (often more
  than a factor of 30x) than the coding of the C code. Within
  \textsc{Chariot} no large effort is explicitly allocated for such
  concrete but difficult tasks.}  is in formalizing these properties
(in terms of C code!)  and then in assisting the formal tool to prove
them.


In contrast, \textsc{Chariot} focuses mainly on a \emph{systems of
  systems} (e.g. networks of systems and systems of networks)
approach, so ``aims to address is how safety-critical-systems should
be securely and appropriately managed and integrated with a fog
network made up of heterogeneous IoT devices and gateways.''. Within
\textsc{Chariot}, static analysis methods have to be ``simple'' and
support its \emph{Open IoT Cloud Platform} thru its \emph{IoT Privacy,
  Security and Safety Supervision Engine} \index{IPSE}~\footnote{Taken
  from \bmurl{https://www.chariotproject.eu/About/}}, and some
industrial \textsc{Chariot} partners, while being IoT network and
hardware experts, are noticing that their favorite IDE (provided by
their main IoT hardware vendor) is running some GCC under the hoods
during the build of their firmware, but are not used to static source
code analysis tools. The \textsc{Chariot} approach to static source
code analysis do \emph{not} require the same level of expertise as needed for
the \emph{Verified in Europe} label pushed by the \textsc{Vessedia}
project.

The \textbf{\textsc{Chariot} approach} to static source analysis
\textbf{leverages on} an \textbf{\emph{existing}} \emph{recent}
\textbf{GCC cross-compiler}~\footnote{The actual version of \emph{GCC}
  is important; we want to stick -when reasonably possible- to the
  latest GCC releases, e.g. to \href{https://gcc.gnu.org/gcc-8/}{GCC
    8} in autumn 2018. In the usual case, that \emph{GCC} is a
  cross-compiler. In the rare case where the IoT system runs on an
  x86-64 device under Linux, that \emph{GCC} is not a cross-, but a
  straight compiler.}. So the IoT developer following the
\textsc{Chariot} methodology would just add some additional flags to
\emph{existing} \texttt{gcc} or \texttt{g++} cross-compilation
commands, and needs simply to change slightly his/her build automation
scripts (e.g. add a few lines to his \texttt{Makefile}). Such a gentle
approach (see figure \ref{fig:chariotcompil}; of course the \emph
{details} of compilation commands would change, the commands shown in
the figure are grossly simplified!) has the advantage of not
disturbing much the usual developer workflow and habit, and addresses
also the \emph{junior} IoT software developer. The compilation and
linking processes are communicating -via some additional \emph{GCC}
plugins (cf. \cite{gcc-internals} §24) doing inter-process
communication- with our \index{persistent monitor}
\index{persistent!monitor} \emph{persistent monitor}, tentatively
called \texttt{bismon} \index{Bismon}. It is preferable (see
\cite{gcc-runtime-library-exception}) to use free software GCC plugins
(or free software generators for them) when compiling proprietary
firmware with the help of these plugins; otherwise, there might
be~\footnote{Of course, I -Basile Starynkevitch- am not a lawyer, and
  you should check any potential licensing issues with your own
  lawyer.} some licensing issues on the obtained proprietary binary
firmware blob, if it was compiled with the help of some hypothetical
\emph{proprietary} GCC plugin.

\begin{figure}[h]
  \begin{center}
    \bmincludewidthgraphics{0.85\textwidth}{chariot-compil-fig}{eps}{svg}
  \end{center}
  \caption{the \textsc{Chariot} compilation of some IoT firmware or
    application {\textbf{(simplified)}}}
  \label{fig:chariotcompil}
\end{figure}

%No \textsc{Chariot} industrial
%partner have any prior extensive experience of static formal-methods
%based source code analysis techniques and tools such as
%\emph{Frama-C}. Most of them don't even build their firmware on a
%Linux workstation (but depend on some proprietary IDE containing an
%obsolete version of \emph{GCC} and have to run that on Windows.).

\subsubsection{The power of an existing compiler: GCC}

\textsc{Chariot} static analysis tools will leverage on the mainstream
\textsc{Gcc}~\footnote{``Gnu Compiler Collection''. See \bmurl{http://gcc.gnu.org/} for more. In
  practice, it is useful to build a recent \textsc{Gcc}
  cross-compiler, fitted for your IoT system, from its published
  source code - see \bmurl{https://gcc.gnu.org/mirrors.html} for a
  list of mirrors.} compiler (generally used as a
\emph{\index{compiler}\index{cross-compiler}cross-compiler} for IoT
firmware development) Current versions of \emph{GCC} (that is, GCC 8.2
in september 2018) are capable of quite surprising optimizations
(internally based upon some sophisticated static analysis techniques
and advanced heuristics).


We show below several examples of optimizations done by recent
\emph{GCC} compilers. Usually, these optimizations happen in the
middle-end and work on internal intermediate representations which are
mostly not~\footnote{Most of the internal \emph{GCC} representations
  -e.g. \emph{Gimple} or \emph{SSA}- are common to all target systems;
  however, some constants, like the size and alignment of primitive
  data types such as \texttt{long} or pointers, are known at
  preprocessing phase or at early \emph{Gimplification} phase.} target
specific.

\bigskip

{{\raisebox{3pt}{\textcolor{brown}{\rule{0.2\textwidth}{2.0pt}}}} ~ \large{recursive inlining with constant folding}}


\begin{table}[t]
\caption{\label{tab:factinlinecsrc} recursive inlining with constant folding in \emph{GCC} (C source)}
   \medskip
  \begin{center}
    \begin{relsize}{-1.2}
     \begin{tabular}{c}
      \lstinputlisting[language=C]{examples/factinline12.c} \\ 
       \textbf{\emph{C source code}} \\ 
       \hline
     \end{tabular}
    \end{relsize}
  \end{center}
\end{table}

The table \ref{tab:factinlinecsrc} shows a simple example of C
code. After preprocessing and parsing, it becomes quickly expanded in
some \emph{Gimple}\index{gimple} representation, whose elementary
instructions are arithmetic with two operands, or simple tests with
jumps, or simple calls (in so called \emph{A-normal form}
\index{A-normal form}, where nested calls like \verb+a=f(g(x),y);+ get
transformed into a sequence introducing a temporary $\tau$ such as
$\tau$\verb+=g(x)+ then \verb+a=f(+$\tau$\verb+,y)+, etc...), shown in table
\ref{tab:factinlinegimple}. This is a textual (and incomplete)
representation of some in-memory \emph{internal intermediate
  representation} during compilation.

\input{generated/002-gimple-factinline.tex}

After gimplification, many other optimizations happen. \textbf{The
  \emph{GCC} compiler runs more than two hundred optimization passes
  !}\index{optimization}\index{pass!optimization}. The table
\ref{tab:factinlinessa} shows the ``static single assigment'' form
(SSA, see \cite{pop:ssa} and \cite{gcc-internals} §13), where variables are duplicated so that each SSA
variable gets assigned only \emph{once}. The control flow is reduced
to \index{basic block} \emph{basic blocks} (with a single entry point
at start, and perhaps several conditional exit edges). Then special
$\phi$ nodes introduce places where such a variable may come from two
other ones (after branches).

\input{generated/003-ssaopt-factinline.tex}


At last, many other optimizations happen. And the optimized form in
table \ref{tab:factinlineoptim} shows that \texttt{fact12} just
returns the constant 479001600 (which happens to be the result of
\texttt{fact(12)} computed \emph{at compile-time}). Finally, the
generated assembler code has no trace of \texttt{fact} function, and
contains just what is shown in table \ref{tab:factinlineasm} (where
many useless comment lines, giving the detailed configuration of the
cross compiler, have been removed).


\bigskip

\newpage

{{\raisebox{3pt}{\textcolor{brown}{\rule{0.2\textwidth}{2.0pt}}}} ~ \large{heap allocation optimization}}

\bigskip

Our second example shows that \emph{GCC} is capable of clever
optimizations around dynamic heap allocation and de-allocation. Its
source code in file \texttt{mallfree.c} is shown in table
\ref{tab:mallfreecsrc}. The straight \emph{GCC}
compiler~\footnote{Similar optimizations also happen with a GCC MIPS
  targetted cross-compiler.} (on Linux/x86-64) is optimizing and able
to remove the calls to \texttt{malloc} and to \texttt{free}, following
the \emph{as-if rule} \index{as-if rule}.

\begin{table}[t]
\caption{\label{tab:mallfreecsrc} optimization around heap allocation by \emph{GCC} (C source)}
   \medskip
  \begin{center}
    \begin{relsize}{-1.2}
     \begin{tabular}{c}
      \lstinputlisting[language=C]{examples/mallfree.c} \\ 
       \textbf{\emph{C source code}} \\ 
       \hline
     \end{tabular}
    \end{relsize}
  \end{center}
\end{table}


\include{generated/004-gimplessa-mallfree}

The \emph{Gimple} form shown in table
\ref{tab:mallfreegimple}. Pointer arithmetic has been expanded to
target-specific \emph{address} arithmetic in byte units, knowing that
\texttt{sizeof(int)} is 4.

\begin{table}[t]
\caption{\label{tab:mallfreegimple} optimization around heap allocation by \emph{GCC} (Gimple)}
   \medskip
  \begin{center}
    \begin{relsize}{-1.5}
     \begin{tabular}{c}
      \lstinputlisting[language=C]{generated/mallfree-gimple.c} \\ 
       \textbf{\emph{Gimple code}} \\ 
       \hline
     \end{tabular}
    \end{relsize}
  \end{center}
\end{table}

\medskip

The \emph{SSA/optimized} form appears in table
\ref{tab:mallfreeoptim}. It shows that the call to \texttt{malloc} and
to \texttt{free} have been optimized away, so the \texttt{weirdsum}
function don't use heap allocation anymore.

\begin{table}[t]
\caption{\label{tab:mallfreeoptim} optimization around heap allocation by \emph{GCC} (SSA/optimized)}
   \medskip
  \begin{center}
    \begin{relsize}{-1.5}
     \begin{tabular}{c}
      \lstinputlisting[language=C]{generated/mallfree-optimized.c} \\ 
       \textbf{\emph{SSA/optimized code}} \\ 
       \hline
     \end{tabular}
    \end{relsize}
  \end{center}
\end{table}

So the generated x86-64 assembler code in table \ref{tab:mallfreeasm}
contain no calls to \texttt{malloc} or \texttt{free}, hence contains the same code that would be generated from just \texttt{int weirdsum(int x, int y) \{return x+y;\}}.

%\verbfilebox{mallfree-tail.s}

\smallskip

\begin{table}[t]
\caption{\label{tab:mallfreeasm} optimization around heap allocation by \emph{GCC} (x86-64 assembly)}
   \medskip
  \begin{center}
    \begin{relsize}{-1.5}
     \begin{tabular}{c}
       %\lstinputlisting{language=[x64]Assembler}{generated/mallfree-tail.s} \\
       \begin{minipage}{0.8\textwidth}
         %\listinginput[4]{1}{generated/mallfree-tail.s}
         \VerbatimInput{generated/mallfree-tail.s}
       \end{minipage} \\
       \textbf{\emph{x86-64 assembly}} \\ 
       \hline
     \end{tabular}
    \end{relsize}
  \end{center}
\end{table}

\medskip

This \texttt{mallfree.c} example could look artificial (because human
developers won't code this way). However, a similar example might
happen in real life after preprocessor expansion and inlining in large
header-mostly libraries. In addition, most genuine C++11 \index{C++11} code
(e.g. using standard container \index{container!C++} templates from \texttt{<map>} or
\texttt{<vector>} standard headers) would produce conceptually similar
code (since many standard constructors and destructors would call
internally the standard \texttt{::operator new} and \texttt{::operator
  delete} operations, which get inlined into calling the system
\texttt{malloc} and \texttt{free} functions).

\newpage
%%%%%%%%%%%%%%%%%%%%%%%%%%%%%%%%%%%%%%%%%%%%%%%%%%%%%%%%%%%%%%%%


\subsubsection{Leveraging simple static source analysis on \emph{GCC}}

By hooking thru \textsc{Chariot} specific GCC
\emph{plugins}~\footnote{Notice that GCC plugins work mostly on Linux
  -but not really on Windows-, so this explains the \textsc{Chariot}
  requirement of [cross-]compiling IoT firmware on a Linux
  workstation.}  into usual [cross-] \emph{compilation} processes
(some \texttt{gcc} or \texttt{g++}, etc... such as a
\texttt{mips-linux-gnu-gcc-8} or a \texttt{arm-linux-gnueabi-g++-8},
etc...), IoT software developers will be able to take advantage of all
the numerous optimizations and processing done by \emph{GCC}. However,
a typical firmware build would take many dozens of such compilation
processes, since every translation unit (practically \texttt{*.c} C
source files and \texttt{*.cc} C++ source files of the IoT firmware)
requires its compilation process~\footnote{Technically, a C
  compilation process would be running some \texttt{cc1} internal
  executable started by some \texttt{*gcc*} [cross-] compilation
  command.}. In practice, IoT software developers would use some
\emph{existing} \textbf{build automation} tool (such as \texttt{make},
\texttt{ninja}, \texttt{meson}, \texttt{cmake} etc...)  \index{build
  automation} which is running suitable compilation commands. They
would need to adapt~\footnote{How to adapt cleverly their
  \texttt{Makefile} to take advantage of \textsc{Chariot} provided
  static analysis is of course the responsability of the IoT
  developer. Surely several extra lines are needed, and the
  \texttt{CFLAGS=} line of their \texttt{Makefile} should be
  changed. For other builders such as \texttt{ninja}, etc..., some
  similar configuration changes would be needed.}  and configure their
build process (e.g. by editing their \texttt{Makefile}-s, etc...),
notably to fetch their \emph{GCC} plugin C++ code and compile it into
some \texttt{*.so} shared object to be later \texttt{dlopen}-ed by
some cross-compiler \texttt{cc1} process, and to use these plugins in
their cross-compilation of their IoT firmware. By working in
cooperation with existing \emph{GCC} compilation tools, the IoT
developer don't have to change much his/her current development
practices. However, these various compilation processes are producing
partial new (or updated) internal representations, and need to know
about other translation units. So some \textbf{persistence is needed}
to keep some data related to past~\footnote{Notice that recent
  \emph{GCC} provide a link-time optimization (LTO) ability, if the
  developer compiles \emph{and links} with e.g. \texttt{gcc -O2
    -flto}. But LTO is not widely used since it slows down a lot the
  building time, and the plugin infrastructure of GCC is not very LTO
  friendly and would be brittle to use. At last, there won't be any
  practical user interface with such an approach. So persistence is
  practically needed, both without LTO and if using LTO.}  compilation
processes during perhaps the entire project development work.

IoT developers need to interact with their static source code analysis
(\emph{GCC} based) tool. In particular, they might need to understand
more the optimizations done by their (\textsc{Chariot} augmented, thru
\emph{GCC} plugins) compiler, and they also need to be able to query
some of the numerous intermediate data related to that static analysis
and compilation. Practically, a small team of IoT developers working
on the same IoT firmware project would interact with the static
analysis infrastructure, that is with a single \emph{persistent
  monitor} process. That persistent monitor (\emph{Bismon}) would be
some ``server-like'' program, started probably every working day in
the morning and loading its previous state, and gently stopped every
evening and dumping its current state on disk. It needs to keep its
persistent state in some files. For convenience, a textual format is
preferable~\footnote{This is conceptually similar the the SQL dump
  files of current RDBMS. But of course \emph{Bismon} don't use an SQL
  format, but its own textual format.}. These persistent store files
could (and probably should) be regularily backed up and kept in some
version control system.

Since a single \emph{Bismon} process is used by a small team of IoT
developers, it provides some Web interface: each IoT developer will
interact with the persistent monitor thru his/her web
browser~\footnote{We don't aim compatibility with all web browsers
  -including old ones- but just with the latest \emph{Chrome} and
  \emph{Firefox} browsers, using HTML5, JavaScript, WebSockets, AJAX
  technologies.}. In addition, a static analysis expert (which could
perhaps be the very senior IoT developer of the team) will configure
the static analysis (also thru a web interface).

The figure \ref{fig:bismonit} gives an overall picture: on the top,
both Alice and Bill are working on the same IoT project source code
(and each have a slightly different version of that code, since Alice
might work on routine \texttt{foo\_start} in file \texttt{foo.cc},
while Bob is coding the routine \texttt{bar\_event\_loop} in file
\texttt{bar.c}). Both Alice and Bob (IoT developers in the same team,
working on the same IoT firmware) are compiling with the same
\emph{GCC} cross-compiler (the GCC egg) enhanced by a plugin (the
small green puzzle piece, at right of GCC eggs). They use their
favorite editor or IDE to work on the IoT source code, and run from
time to time a builder (e.g. \texttt{make}). They use a browser (with
a rainbow in the figure) to interact with the monitor and query static
analysis data. The purple dashed arrows represent HTTP exchanges
between their browser and the monitor. The compilation processes,
extended by the GCC plugin, communicate with the monitor (thru the
gray dashed arrows). A static analysis expert (the ``geek'', at left
of the Bismon monitor) is configuring the monitor thru his own web
interface. The monitor is \emph{generating} (orange arrow) the C++
code for GCC plugins (small blue hexagon at right), and the IoT
developer needs to change his/her build procedures to compile and use
that generated GCC plugin. \emph{Bismon} also uses meta-programming
techniques to emit (curved blue arrow to left) internal code (bubble)
- notably C code dynamically loaded by the monitor and JavaScript/HTML
used in browsers. The monitor persists its entire state on disk, so
can restart in the state that it has dumped in its previous run.

\begin{figure}
  \begin{center}
    \bmincludewidthgraphics{0.85\textwidth}{bismon-monitor-fig}{eps}{svg}
  \end{center}
  \caption{The \emph{Bismon} monitor used by some IoT developer team following the \textsc{Chariot} approach.}
  \label{fig:bismonit}
\end{figure}

\newpage 
%%%%%%%%%%%%%%%%%%%%%%%%%%%%%%%%%%%%%%%%%%%%%%%%%%%%%%%%%%%%%%%%
\subsection{About \textit{Bismon}}

\textit{Bismon} (a \emph{temporary name})\index{Bismon} \footnote{\texttt{bismon} is
  a \textbf{temporary name} and could be changed, once we find a
  better name for it} is a free software (GPLv3+
licensed)\footnote{The source code is unreleased but available, and
  continuously evolving, on
  \bmurl{https://github.com/bstarynk/bismon}} static source code
whole-program analysis framework whose initial domain will be
\emph{Internet of Things} (or \index{IoT}{IoT})\footnote{IoT is viewed
  as the first application domain of \textit{Bismon}, but it is hoped
  that most of \textit{Bismon} could be reused and later extended for
  other domains}. It is designed to work with the \textit{Gcc}
compiler (see \bmurl{gcc.gnu.org}) on a Linux
workstation\footnote{Linux specific features are needed by
  \textit{Bismon}, which is unlikely to be buildable or run under
  other operating systems. My Linux distribution is
  \emph{Debian/Unstable}}. \textit{Bismon} is conceptually the
successor of \textit{GCC MELT} \footnote{The \textit{GCC MELT} web
  pages used to be on \texttt{gcc-melt.org} -a DNS domain relinquished
  in april 2018- and are archived on
  \bmurl{https://starynkevitch.net/Basile/gcc-melt}} (see
\cite{Starynkevitch2007Multistage, starynkevitch-DSL2011}), but don't
share any code with it.

{\large \textit{Bismon} is \textbf{work in progress}}, and many things
described here are not (fully) implemented in 2018 or
could drastically change later. In particular, \textit{Bismon} is not
yet usable as a static source code analyser in mid-2018 (since work on
the infrastructure is not complete).

\bigskip

\subsection{Lessons learned from \textit{GCC MELT}}

@@ To be written

\medskip

\subsection{Driving principles for  \textit{Bismon}}

\textit{Bismon} is (like \textit{GCC MELT} was) a \textbf{domain
  specific language} implementation, targetted to ease static source
code analysis (above the \textit{GCC} compiler), with the following
features:

\begin{itemize}

  \item \index{persistence}{\textbf{persistence}}, somehow
    \textit{orthogonal persistence}. It is needed in particular
    because compiling some software project (analyzed by
    \textit{Bismon}) is a long-lasting procedure involving
    \textit{several} compiling and linking processes and commands. So
    most of the data handled by \textit{Bismon} can be persisted on
    disk, and reloaded at the next run. However, some data is
    temporary by nature\footnote{E.g. data related to a web session,
      or to a web HTTP exchange, or to an ongoing \texttt{gcc}
      compilation process, etc... needs not to be persisted, and would
      be useless when restarting the \textit{Bismon monitor}.} and
    should not be persisted. Such data is called temporary or
    \index{transient}{\textbf{transient}}. But the usual approach is
    to run the \textit{Bismon} program from some initial loaded state
    and have it \index{dump}{\textbf{dump}} its memory state on
    disk~\footnote{Look also into Liam Proven FOSDEM 2018 talk about
      \emph{The circuit less traveled} on
      \bmurl{https://archive.fosdem.org/2018/schedule/event/alternative\_histories/}
      for an interesting point of view regarding persistent systems.}.
    before exiting (and reload that augmented state at the next run),
    and probably more often than that (e.g. twice an hour, or even
    every five minutes).

  \item \textbf{dynamic typing}, like many scripting languages (such
    as Guile, Python, Lua, etc). Of course the dynamically typed data
    inside the \textit{Bismon monitor} is \index{garbage
      collection}{\textbf{garbage collected}}
    (cf. \cite{Jones:2011:GC-handbook})
    \footnote{\label{fn:initial-gc}The initial GC of the monitor is a crude precise garbage
      collector, but multi-thread compatible; it uses a naive
      stop-the-world mark\&sweep algorithm. Improving that GC is a
      difficult work, and past experience on \emph{GCC MELT} taught us
      that developing and debugging a GC is hard, and is a good
      illustration of Hofstadter's law (See
      \cite{Hofstadter:1979:GEB}). We might consider later using MPS
      from \bmurl{https://www.ravenbrook.com/project/mps/} but doing
      that could require recoding or regenerating \emph{a lot} of code, since MPS has specific calling conventions and invariants.}

  \item \textbf{homoiconicity} and \textbf{reflection} with
    \textbf{introspection} (cf \cite{Pitrat:1996:FGCS,
      Pitrat:1990:Metaconnaissances, Pitrat:2009:AST,
      Pitrat:2009:ArtifBeings}): all the DSL code is explicitly
    represented as data which can be processed by \textit{Bismon}, and
    the current state is accessible by the DSL.

    \item \textbf{translated} to \emph{C} code; and \textbf{generated}
      \emph{JavaScript} + \emph{HTML} in the \index{browser}{browser}, and generated
      \emph{C++} code of \emph{GCC} plugins

    \item \textbf{bootstrapped implementation}: ideally, all of
      \textit{Bismon} code (including C code dealing with data
      representations, persistent store, etc...) should be
      generated. However, this ideal has not yet be attained, and
      there is still some hand-written C code. It is hoped that most
      of the hand-written C code will become replaced by generated C
      code.
    \item ability to \textbf{generate GCC plugins}: the C++ code of
      GCC plugins performing static analysis of a single translation
      unit should be generated.

    \item with \textbf{collaborative web interface}, used by a
      \emph{small} team of \emph{trusted and well-behaving}
      developers~\footnote{The initial \texttt{bismon} implementation
        had a hand-coded crude GTK interface, nearly unusable. That
        interface is temporarily used to fill the persistent store
        till the web interface (generated by \emph{Bismon}) is
        usable. The GTK interface is already obsolete and should
        disappear at end of 2018. The Web interface (work in
        progress!) is mostly generated - all the HTML and JavaScript
        code is generated (or taken from outside existing projects
        e.g. \href{http://jquery.com/}{\emph{JQuery}} or
        \href{http://codemirror.net/}{\emph{CodeMirror}}), and their
        HTML and JavaScript generators are made of generated C
        code.}. The users of \emph{bismon} are expected to trust each
      other, and to use the \texttt{bismon} tool
      responsibly\footnote{For example, each \emph{bismon} user has
        the technical ability to erase most of the data inside
        \textit{Bismon monitor}, but is expected to not do so. There
        is no mechanism to forbid or warn him doing such bad things.}
      (likewise, developers accessing a \texttt{git} version control
      repository are supposed to act responsibly even if they are
      technically capable of removing most of the source code and its
      history stored in that repository). So protection against
      malicious behavior of \texttt{bismon} users is out of scope.

      Since \textit{Bismon} should be usable by a small team of
      developers (perhaps two or a dozen of them)\footnote{So
        \textit{Bismon}, considered as a Web application, would have
        at most a dozen of \index{browser}{browsers} -and associated users- accessing
        it. Hence, scalability to many HTTP connections is not at all
        a concern (in contrast with most usual web applications).}, it
      is handling some \index{personal data}{personal data} (relevant
      to \index{GDPR}{GDPR}), such as the first and last names (or
      pseudos) of users and their email and maintain a password file
      (used in the Web \index{login}{login} form). Compliance to
      regulations (e.g. European GDPR) is out of scope and should be
      in charge of the entities and/or persons using and/or deploying
      \textit{Bismon}. The login form template \footnote{on
        \bmurl{https://github.com/bstarynk/bismon/blob/master/login\_ONIONBM.thtml}}
      could be adapted on each deployment site.
\end{itemize}


\subsubsection{About \emph{Bismon} as a domain-specific language}

Notice that \textit{Bismon} is not even thought as a \emph{textual}
domain specific language~\footnote{In contrast, \emph{GCC MELT} was
  textual and had \texttt{*.melt} source \emph{files} edited with
  \texttt{emacs}. This made refactoring difficult, since automatic
  move of textual fragments was not realistically possible.} (and this
is possible because it is persistent). There is not (and there won't
be) any canonical \emph{textual} syntax for ``source code'' in
\textit{Bismon} ~\footnote{This idea is not new: neither Smalltalk
  (cf. \cite{Goldberg:1983:Smalltalk}), nor Common Lisp
  (cf. \cite{Steele:1990:CommonLisp}), are defined as having a textual
  syntax with an associated operational semantics based on it. Even
  the syntax of C is defined \emph{after} preprocessing. What is
  -perhaps informally- defined for Smalltalk and Common Lisp is some
  abstract internal representation of ``source code'' in ``memory''
  and its ``behavior''. In contrast, Scheme has both its textual
  syntax and its semantics well defined in R5RS, see
  \cite{Adams:1998:R5RS}.}. \index{source code}{\emph{Source code}} is
defined (socially) as the preferred form on which developers are
working on programs. For C or C++ or Scheme programs, source code
indeed sits in textual files in practice (even if the C standard don't
speak of files, only of ``translation units'', see \cite{C11:std}),
and the developer can use any source code editor (perhaps an editor
unaware of the syntax of C, of C++, of Scheme) to change these source
files. In contrast, a developer contributing to \textit{Bismon} is
browsing and changing some internal representations thru the
\textit{Bismon} user interface (a Web interface~\footnote{in mid-2018,
  that Web interface was incomplete, and I still had to temporarily
  use some obsolete GTK-based interface that even I find so disgusting
  that I won't explain it, and sometimes even to edit manually some
  \texttt{store2.bmon} data file, cf §~\ref{subsubsec:filestate}.})
and interacts with \textit{Bismon} to do that. There is no really any
\index{abstract syntax tree}{abstract syntax \emph{tree}} in
\textit{Bismon}: what the developer is working on is some \emph{graph}
(with circularities), and the entire persistent state of \emph{Bismon}
could be viewed as some large graph in memory.


\subsubsection{About \emph{Bismon} as a evolving software system}

So \emph{Bismon} is better thought of as an evolving software
system. We recommend to try it. Notice that \textit{Bismon} is
\textbf{provided as free software} (unreleased in 2018, but available
on \bmurl{github.com/bstarynk/bismon} ...) in \emph{source form only}
and should be \textbf{usable} (only) \textbf{on a Linux/x86-64
  workstation}... (typically, at least 32 gigabytes of RAM and
preferably more, at least 8 cores, several hundreds gigabytes of disk
or SSD).

The \textit{Bismon} system contains \textbf{persistent data} (which is
part of the system itself and should not be considered as ``external''
data; each team using \textit{Bismon} would run its own customized
version of their \textit{Bismon monitor}.), and should be
\textbf{regularily backed up}, and preferably version controlled at
the user site. It is strongly recommended to use
\index{git}{\texttt{git}} \footnote{See \bmurl{http://git-scm.com/}}
or perhaps some other distributed \index{version control}{version
  control} system, to \texttt{git commit} its files several times a
day (probably hourly or even more frequently, as often as a developer
is committing his C++ code), and to backup the files on some external
media or server at least daily. How that is done is outside of the
scope of this document. The \emph{dump facilities} inside
\textit{Bismon} are expected to be used quite often (as often as you
would save your report in a word processor, or your source file in a
source code editor), probably several times per hour. So a developer
team using \textit{Bismon} would probably \texttt{git clone} either
\texttt{git@github.com:bstarynk/bismon.git} thru SSH or
\bmurl{https://github.com/bstarynk/bismon.git}, build it (after
downloading and building required dependencies), and work on that
\texttt{git} repository (and of course back-up it quite often).

People trying the \emph{Bismon monitor} will use some Web interface to
interact with it.

\subsubsection{About \emph{Bismon} as a static source code analyzer framework}

@@@ to be written when some static source code analysis of some C or
C++ code is possible with \emph{Bismon}.


%%%%%%%%%%%%%%%%%%%%%%%%%%%%%%%%%%%%%%%%%%%%%%%%%%%%%%%%%%%%%%%%
\subsection{Multi-threaded and distributed aspects of \textit{Bismon}}

The \textit{Bismon monitor} is by itself a multi-threaded
process\footnote{In contrast of most scripting languages
  implementations such as Python, Ocaml, Ruby, etc..., we try hard to
  avoid any ``global interpreter lock'' and strive to develop a
  genuinely multi-threaded monitor.}.
It uses a \emph{fixed} {\index{thread pool}{\emph{thread pool}}} of
{\index{worker thread}{\emph{worker threads}}} (often active)
\footnote{The number of worker threads is given by the \texttt{--job}
  program argument to \texttt{bismon}. For an 8-cores workstation, it
  is suggested to set it to 5 or 6. It should be at least 2, and at
  most 15. This number of jobs also limits the set of simultaneously
  running external processes, such as \texttt{gcc} processes started
  by \emph{Bismon}.}, and additional (generally idle) threads for web
support and other facilities. The \textit{Bismon monitor} is
occasionally starting some external processes, in particular for the
compilation of generated \emph{GCC} plugins, and for the compilation
into \index{module}{\emph{module}s} -technically ``plugins''- of
dynamically generated \emph{C} code by \textit{Bismon}; later it will
dynamically load (with \texttt{dlopen}) these modules, and thus
\textit{Bismon} can increase its code (but cannot decrease it, even if
some code becomes unused and unreachable); however such modules are
\emph{never} \index{garbage collection}{garbage collected} (so
\texttt{dlclose} is never called). So in practice, it is recommended
to restart \emph{Bismon} every day (to avoid endless growth of its
code segments).


The worker threads of \emph{Bismon} are implementing its
\index{agenda}{\textbf{agenda}} \footnote{Details about the agenda are
  subject to change.} machinery. Conceptually, the agenda is a 5-tuple
of first-in first-out queues of \index{tasklet}{\textbf{tasklets}},
each such FIFO queue is corresponding to one of the five
\index{priority}{priorities} : \emph{very high}, \emph{high},
\emph{normal}, \emph{low}, \emph{very low}. Each agenda worker thread
removes one tasklet (chossing the queue of highest possible priority
which is non empty, and picking the tasklet in front of that queue)
and runs that tasklet quickly. A tasklet should run during a few
milliseconds (e.g. with some implicit kind of non-preemptive
scheduling) at most (so cannot do any blocking IO; so input and output
happens outside of the agenda). It may add one or more tasklets
(including itself) to the agenda (either at the front, or at the end,
of a queue of given priority), and it may remove existing tasklets
from the agenda. Of course tasklets run in parallel since there are
several worker threads to run the agenda. The agenda itself is not
persisted as a whole, but tasklets\footnote{Actually tasklets are
  objects (see §\ref{subsubsec:objects} page
  \pageref{subsubsec:objects} below), and to run them, the agenda is
  sending them a message with the predefined selector
  \texttt{\emph{run\_tasklet}}.}  themselves may be
\index{persistent}{persistent} or \index{transient}{transient}.
Tasklets can also be created outside of the agenda (e.g. by incoming
HTTP requests, by completion of external processes, by timers, ...)
and added asynchronously into the agenda.

Outside of the agenda, there is an \emph{idle queue} of delayed todo
closures (a queue of closures to be run, as if it was an idle priority
queue) with some arguments to apply to them. But that \index{idle
  queue}{idle queue} don't contain directly any tasklets. That idle
queue can be filled by external events~\footnote{For example, when an
  external compilation process completes, that queue is filled with
  some closure -provided when starting that compilation- and, as
  arguments, an object with a string buffer containing the output of
  that process, and the integer status of that process.}. Of course the idle
queue is not persisted.

\bigskip

When running (in 2019) for static source code analysis purposes, the \emph{Bismon system} involves several processes:

\begin{itemize}
\item the \emph{Bismon monitor} itself, with several threads (notably for the agenda mechanism described above)
  
  \item the web \index{browser}{browsers} of developers using that
    particular \emph{Bismon monitor}; each developer probably runs
    his/her own browser. That web browser is expected to follow latest
    Web technologies and standards (HTML5, Javascript6 i.e. EcmaScript
    2016 at least, WebSockets, ...). It should probably be a Firefox
    or a Chrome browser from 2017 or after. The HTML and Javascript is
    dynamically generated by the \emph{Bismon monitor} and should
    provide (to the developer using \emph{Bismon}) some ``single-page
    application'' (cf. \cite{Atkinson:2018:webtabs,
      Queinnec:2004:ContinWeb, Graunke:2003:ModelingWeb})
    feeling~\footnote{So using your browser's backward and forward
      navigation arrows won't work well because in single-page
      applications they \emph{cannot} work reliably}.

  \item sometimes \emph{Bismon} would generate some
    \texttt{modules/*.c} file during execution, and fork a (direct)
    compilation of it (technically forking a
    \texttt{./build-bismon-persistent-module.sh} -for persistent
    modules- or a \texttt{./build-bismon-temporary-module.sh} -for
    temporary modules- shell script, which invokes \texttt{make} which
    runs some \texttt{gcc} copmmand) into a ``plugin'' module
    \texttt{modubin/*.so}, which would be \texttt{dlopen}-ed.

  \item \emph{Bismon} should also generate the C++ code of \emph{GCC
    plugins}, to be later compiled then used (with \texttt{gcc} or
    \texttt{g++} option \texttt{-fplugin}). Two kinds of \emph{GCC}
    plugins are considered to be generated:

    \begin{enumerate}
      \item usually, the GCC plugin would be generated to assist
        [cross-] compilation (e.g. of IoT software) by developers
        using \emph{Bismon}. So for an IoT developer targetting some
        RaspberryPi, it could be a GCC plugin targetting the
        \texttt{arm-linux-gnueabi-gcc-8} cross-compiler (but the C++
        code of that plugin needs to be compiled by the native
        \texttt{gcc} on the host system).
        
      \item But the GCC API is so complex (and under-documented) that
        it is worth extracting it automatically by sometimes
        generating a GCC plugin to inspect the public headers of
        GCC\footnote{In \emph{GCC MELT}, we tried to describe by
          hand-coded \emph{MELT} code a small part of that GCC API and
          its glue for \emph{MELT}. This approach is exhausting, and
          makes following the evolution of GCC very difficult and
          time-consuming, since new \emph{MELT} code should be written
          or manually adapted at each release of \emph{GCC}. Some
          partial automation is needed to ease that effort of adapting
          to successive \emph{GCC} versions and their non-compatible
          plugins API}. Even when the end-user developer is targetting
        a small IoT chip requiring a cross-compiler (like
        \texttt{arm-linux-gnueabi-gcc-8} above), these GCC inspecting
        plugins are for the native \texttt{gcc} (both
        \cite{Schafmeister:2016:CANDO} and
        \cite{Schafmeister:2015:CLASP} are inspirational for such an
        approach).
        
    \end{enumerate}

    It is yet unclear how such generated C++ code for GCC plugins is
    actually compiled and when exactly is it executed. We might
    generate (at least for the first common case of GCC plugins
    generated for developers using \emph{Bismon}) \texttt{*.shar}
    archives (obtained by Web requests, or perhaps some \texttt{wget}
    or \texttt{curl} command in some \texttt{Makefile}) for
    \emph{GNU sharutils}\footnote{See
      \bmurl{https://www.gnu.org/software/sharutils/}}. containing the
    C++ code and also the \texttt{g++} command compiling it. The rare
    second case (GCC plugin code generated to inspect the GCC API)
    could be handled thru external processes (similar to compilation
    of \emph{Bismon} modules).
    
\end{itemize}

In principle, the various facets of \emph{Bismon} can run on different
machines as \index{distributed computing}{distributed computing}
(obviously the web browser is not required to run on the same machine
as the \emph{Bismon monitor}, but even the various compilations -of
code generated by \emph{Bismon}, and the cross-compilation of IoT
code- could happen on other machines).

Conceptually, we aim for a \index{multi-tier
  programming}{\textbf{multi-tier programming}} approach (inspired by
Ocsigen~\footnote{See \bmurl{https://ocsigen.org/}} with the high-level
DSL inside \emph{Bismon} generating code: in the \emph{Bismon
  monitor}, as modules; in the \emph{web browser}, as generated
Javascript and HTML; in the \emph{GCC} compiler, as generated GCC plugins.
