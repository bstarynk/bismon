% file using-bm.tex, which is \input from bismon-chariot-doc.tex
\section{Using \emph{Bismon}}
\label{sec:using}

This section §\ref{sec:using} should become somehow a user manual, and
will be written for the final D1.3\textsuperscript{v2}. It is both for
the ordinary IoT developer just using \emph{bismon} for static
analysis of IoT source code, and for the static analysis expert
configuring and programming it.

Most of that should be generated from data persisted inside \emph{bismon}. Perhaps
should be exchanged with the ``static analysis'' chapter
(§\ref{sec:staticanalys}).


\subsection{Web interface internal design}
\label{subsec:webinterf}

The Web interface of \emph{bismon} is supposed to be used without
malice (see §\ref{subsubsec:leveraging-static-analysis} and
§\ref{subsubsec:about-bismon}). In particular, \emph{bismon} does not
handle any denial-of-service attack. The network administrator running
\emph{bismon} is expected to deploy relevant techniques (firewalls,
\texttt{iptables}, HTTP proxying, etc ...) to avoid them. The browsers
instances interacting with \emph{bismon} are few (only a dozen of
people in some IoT development team, and each uses his/her graphical
browser - a \emph{recent} Firefox or Chrome~\footnote{So no particular
  effort is even taken to support a variety of old browsers: we don't
  have any code to e.g. support Internet Explorer pecularities or
  deficiencies.}). They are physically and geographically located on
the same local area network as the machine running the \emph{bismon}
monitor. So, from web technologies perspective, \emph{bismon} is
making different trade-offs~\footnote{For example, we could accept
  making some HTTP exchange - e.g. with AJAX - on \emph{every}
  keystroke on the keyboard, but such practice won't be acceptable in
  usual web services. Also, we don't care much about minimizing the
  HTTP exchanges - no ``minification'' needed in practice!} than
``traditional'' web servers.

The web interface uses objects: each HTTP request either corresponds
to a ``static'' file path under \texttt{webroot/} or is handled
dynamically. For a static file path, that file is served directly by
\texttt{onion\_handler\_export\_local\_new} with a
\texttt{Content-Type} corresponding to its suffix; for example an HTTP
\texttt{GET} request of \texttt{favicon.ico} is answered with the
content of \texttt{webroot/favicon.ico} file, and an \texttt{GET}
request of \texttt{jscript/jquery.js} is served by the content of
\texttt{webroot/jscript/favicon.ico}. Care is taken~\footnote{In
  particular, any HTTP request containing \texttt{..} is rejected.} to
avoid serving any static file outside of \texttt{webroot/}. So the
\texttt{webroot/} directory contains static content such as images,
external JavaScript libraries, CSS stylesheets, etc...

%\bigskip

%Notice that most of \emph{GCC MELT} documentation was generated. See its
%{\small\bmurl{http://starynkevitch.net/Basile/gcc-melt/meltpluginapi.html/meltpluginapi.html}}. Likewise,
%most of this chapter will be machine generated from the persistent
%state of \emph{Bismon}
